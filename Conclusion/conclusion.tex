\clearemptydoublepage
\bookmarksetup{startatroot}
\chapter*{Conclusion}
\addcontentsline{toc}{chapter}{Conclusion}
\chaptermark{Conclusion}
% Contributions
% Context
In this thesis, we presented the need for more general string queries than classical pattern matching, and the scalability challenges that come from the large productions and archrivals of data.
%
In Section~\ref*{intro:sec:sketching} and~\ref*{intro:sec:contrib}, we detailed the sketch-based approach common to all contributions. Their use enabled processing and storing data efficiently to yield better time and space complexities.
%
The sketching techniques we use are diverse and are not immediate to apply to other problems, but my personal takeaway is the importance of thinking in terms of the key characteristic of the input for a given query. I find it very helpful in algorithmic design both for theoretical studies and applied projects.
% Open questions in many of the works
Most of the chapters leave open questions and future works to be done:
\begin{itemize}
    \item Chapter~\ref{chap:regexp}: Is it possible to search for any regular expression with $d$ $|$ and $\ast$ symbols in a stream of length $n$ in $\mathrm{poly}(d,\log n)$ space and time complexity ?
    \item Chapter~\ref{chap:gapped_index}: Can we improve the compressed index for consecutive matching to $\Ohtilde(g)$ space and $\Ohtilde(m+\occ)$ and for close consecutive occurrences, can we improve the $\Ohtilde(g^5)$ space ? Is there an efficient index for the general case where we search for consecutive occurrences separated by a distance in an interval $[a,b]$ ?
    \item Chapter~\ref{chap:squares}: Do our lowerbound of $\Omega(n\ln \sigma)$ comparisons hold for randomized algorithm ?
    \item Chapter~\ref{chap:LCS}: could we implement our $\Oh(n^{1+ 1/(1+2\eps) + o(1)})$ time and space solution using implementation of Approximate Nearest Neighbour data structure such as~\cite{} and add it to the practical evaluation ?
    \item Chapter~\ref{chap:DTW}: For a pattern $P$, a text $T$ and an integer $k$, is there a $\Oh(k(n+m))$ time algorithm that computes all positions $r$ such that the smallest DTW distance between $P$ and $T[1..r]$ is at most $k$ ? Additionally, the practical application of DWT for third generation sequencing alignment would be interesting to investigate further, but it is difficult due to many tools using an alignment based on the edit distance as their ground truth.    
    \item Chapter~\ref{chap:XBWT}: We measured the improvement of our structure in terms of the number of runs as a first step, but it remains to evaluate the full data-structure in terms of space usage and query time.
\end{itemize}
%
\todo[inline]{Bellow is a Work in Progress do not review.}
In addition to those open questions and future work, I would personally be interested in the use of spaced seeds~\cite{li2004patternhunter} in bioinformatics. A space seed is a binary sequence that describes positions that relevant (marked by a one) or irrelevant (marked by a zero). It can be used for the popular seed-and-extend approach where exact matches of a substring are extended to alignments, possibly with errors.
Spaced-seeds are an interesting alternative to $k$-mers (substrings of length $k$) as they are robust to substitutions, and are known to provide better accuracy for alignment free comparison~\cite{}.
I would be curious to know if some of the result on gapped consecutive matching could be applied to specific types of seed such as \cite{} where the seeds are just $1^k 0^\Delta 1^k$.
But I would be even more interested in whether the sketching techniques used in bioinformatics (presented at the end of Section~\ref{intro:sec:sketching}) could gain in accuracy by being based on spaced seeds rather than $k$-mers. By considering information
\backmatter