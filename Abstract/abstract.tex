% Overall context on strings
The simplicity of strings and their impactful usage puts their processing at the heart of many applications, including bioinformatics, information retrieval, and cybersecurity. Exact pattern matching has been extensively studied~\cite{charras2004handbook} as the most natural problem. However, many applications also need more complex queries. Additionally, in all those application fields, the quantity of information to process has been increasing at a staggering rate~\cite{muir2016real}, and exact queries do not always scale.
%
% First part 
In the first part of this thesis, we study complex queries such as regular expressions search, gapped consecutive matching, and square detection. 
% Regexp
For regular expression search, we provide a space-efficient algorithm in the streaming model: characters of the text arrive one at a time, and we can only access past characters if we explicitly store them. 
% Gapped
Next, gapped consecutive matching is a simpler type of query where given two patterns $P_1$, $P_2$ and a range $[a,b]$, one must report all consecutive occurrences of $P_1$ followed by $P_2$ separated by a distance in $[a,b]$. We study this problem in various models: streaming, pattern matching on a compressed text, and compressed indexing.
% Squares
For both types of queries, periodicity detection helps to avoid redundant computation. This central use of periodicity makes squares interesting to study beyond their inherent meaning~\cite{Kolpakov2003}. Thus, we investigate square detection and reporting for general alphabets (the most abstract setting where squares can be defined). We provide an optimal algorithm and answer an open question asked in 1984 by Main and Lorentz~\cite{Main1984}.
%
% Second part
The second part of this thesis proposes a few ways to use approximation to speed up the computations toward scaling up to large amounts of data in diverse applications including bioinformatics.
% LCS and DTW
We first study approximate matching, where we must report all occurrences at distance at most a threshold $k$ for a given similarity measure. Motivated by the lower bounds on the computation of the most popular similarity measures, we provide efficient parametrized algorithms for computing the length of the longest common substring with approximately $k$ mismatches and pattern matching for the dynamic time warping distance.
Due to their robustness to noise in the input, both of those measures can be applied to detecting duplicated sections in text for application such as plagiarism detection~\cite{zou2010cluster}, and alignment of biological sequences~\cite{leimeister2014kmacs,loose2016real,han2018accurate}.
% XBWT
Finally, we propose a compressed index for redundant collections of next-generation sequencing reads, which takes advantage of alignments to an assembled genome to improve the overall compression but can incur false positive occurrences.

