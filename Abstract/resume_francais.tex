%Context
La simplicité et la versatilité des chaînes de caractères rendent leur traitement crucial pour de nombreuses applications, telles que la bio-informatique, la recherche d'informations et la cybersécurité.
Le problème naturel de la recherche exact d'un motif a été largement étudié~\cite{charras2004handbook}, cependant, de nombreuses applications nécessitent également des requêtes plus complexes. Par ailleurs, dans ces domaines applicatifs, la quantité de données à traiter augmente à une vitesse stupéfiante~\cite{muir2016real}, et les requêtes exactes ne permettent pas toujours de passer à l'échelle.

%First part
Dans la première partie de cette thèse, nous étudions des requêtes complexes telles que la recherche par expressions régulières, la recherche de motifs consécutifs avec espacement et la détection de carrés.
%Regexp
Pour la recherche d'expressions régulières, nous donnons un algorithme utilisant peu d'espace une entrée sous forme de flots : les caractères du texte arrivent un par un, et nous ne pouvons accéder aux anciens que si nous les avons stockés explicitement.
%Gapped
Ensuite, la recherche de motifs consécutifs avec espacement est un type de requête plus simple, où étant donné deux motifs $P_1$, $P_2$ et un intervalle $[a, b]$, il faut renvoyer toutes les occurrences consécutives de $P_1$ suivies (sans autres occurrences des motifs entre les deux) de $P_2$ espacés d'une distance dans $[a, b]$. Nous étudions ce problème dans plusieurs modèles : algorithme de flot, la recherche de motifs sur un texte compressé, et l'indexation compressée.
%Square
Nos solutions, pour les problèmes précédant, utilisent beaucoup la détection périodicité pour s'épargner des calculs redondants. Cette utilisation de la périodicité rend les carrés (sous-chaîne immédiatement répétée) intéressants à étudier au-delà de leur inhérente signification~\cite{Kolpakov2003}. Nous étudions la détection et le calcul des carrés pour alphabets sans ordres (le cadre le plus abstrait dans lequel les carrés peuvent être définis). Nous fournissons un algorithme optimal et répondons à une question ouverte posée en 1984 par Main et Lorentz~\cite{Main1984}.

% Second part
La seconde partie de cette thèse propose quelques utilisations d'approximations pour aider à passer à l'échelle sur des grandes quantités de données, en particulier avec application à la bio-informatique.
%
Nous étudions tout d'abord la recherche approximative de motifs, où nous devons rapporter toutes les occurrences à une distance au plus égale à $k$ pour une mesure de similarité donnée.
% LCS DTW
%Motivés par les bornes inférieures pour le calcul des mesures de similarité les plus populaires,
Nous fournissons des algorithmes paramétrés efficaces pour calculer la longueur de la plus longue sous-chaîne commune avec environ $k$ différences, puis pour la recherche de motifs avec déformation temporelle dynamique. En raison de leur robustesse aux entrées bruitées, ces deux mesures peuvent être appliquées à la détection de passages répétés dans un texte, nécessaire pour des applications telles que la détection de plagiat~\cite{zou2010cluster} et l'alignement de séquences biologiques~\cite{leimeister2014kmacs,loose2016real,han2018accurate}.
% XBWT
Enfin, nous proposons un index compressé pour des collections de lectures de séquençage redondantes. Cet index tire parti d'alignements sur un génome assemblé pour améliorer la compression, mais il peut en revanche renvoyer des faux positifs dans ses requêtes.