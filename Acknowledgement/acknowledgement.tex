\chapter*{Acknowledgement - Remerciments}
\thispagestyle{empty}

I am grateful to Thierry Lecroq and Simon Puglisi for accepting to review this thesis and to Bastien Cazaux, Élise Prieur-Gaston, and Stéphane Vialette for accepting to be part of my jury. Merci aussi à Mireille Régnier et Eric Rivals pour leur participation à mon comité de thèse.


Mes remerciements vont ensuite à Pierre Peterlongo et Tatiana Starikovskaya pour leurs encadrements. Merci d’avoir été compréhensif et aidant quand j’ai exprimé mes difficultés de santé mentale. Merci de m’avoir offert un environnement scientifique très riche et libre.
Pierre, merci d’avoir accepté de m’encadrer ma thèse, un peu loin de ta recherche habituelle, j’ai beaucoup apprécié de pouvoir côtoyer ta recherche pratique, simple et très inspirante.
Tatiana, merci de m’avoir transmis un peu de tes très grandes connaissances et compétences de chercheuse et de m’avoir mené vers des projets ambitieux.
Malgré votre travail commun sur mes tics de langage, je ne vous promets pas de ne pas dire ``truc'' où ``inte\textbackslash g\textbackslash er'' pendant ma soutenance de thèse...


Merci à mes deux équipes d’accueil, Talgo à l’ENS à Paris et Symbiose (Genscale) à l’IRISA à Rennes. Il y a bien pire problème que de se retrouver à être triste de quitter un environnement et contente d’en retrouver un autre.
À Talgo, je suis reconnaissante d’avoir pu côtoyer\footnote{Et d'avoir pu partager quelques jeux de société !} Chien-Chung, Pierre A., Kevin, Monika, Guillaume, Gabriel, Quentin, et Benoit.
Plus largement, je remercie Michäel\footnote{Pour les gouters-sport du DI et les encouragements mutuels sur l'écriture !}, Yann, Yoan et Juliette d’avoir partagé de joyeuses pauses cafés avec moi pendant mon stage de M2 et doctorat, elles m’auront fait beaucoup de bien dans ces journées denses !
Merci également à Lise-Marie, Linda, Fanny et Tiffany pour leur aide administrative précieuse. Merci au DI et la Fondation de l’ENS pour avoir financé certains de mes voyages de recherche et mon extension de thèse de 2 mois.
Merci également à Laurent Boyer pour m’avoir permis d’enseigner le Python au L2 MIASH de Paris 1, ce fut un plaisir même avec de cours en distanciel et semi-distanciel.


À Symbiose je remercie toute la grande équipe, dont j'ai énormément apprécier l’ambiance joviale, solidaire, et chaleureuse.
Des mercis du fond du cœur vont à Lucas\footnote{Très fière d’être ta jumelle de thèse.}, Matthieu\footnote{Toujours à l'écoute et réconfortant!},  Nicolas\footnote{J’espère m’être un peu inspiré de ta précision pour se manuscrit.}, et Victor\footnote{J’admire énormément ton engagement politique.}  pour m’avoir intégré à leur équipe de réalisation de Science en cour[t]s\footnote{Festival de courts métrages de vulgarisation scientifique Rennais, que nous avons ensuite organisé !}, pour m’avoir hébergé, conseillé et distraite dans les moments difficiles. Merci pour ces très beaux moments d’amitiés, vous êtes tous les bienvenus à Paris pour que je vous rende un peu la pareille.
Merci à l’enthousiaste Karel pour m’avoir fait découvrir ses passionnantes utilisations de l’informatique théorique appliqué à la bio-informatique et pour tous ses encouragements qui me sont allé droit au cœur !
Un merci particulier à Marie, Jacques, Emanuelle, Clara, Téo, Camille, Olivier, Khonogan, Sandra, Kerian, Baptiste, Khodor, Rolland, Meven, Victor M, Léo et Luca, dont j’ai pu apprécié la compagnie.


I am also deeply grateful for the opportunity of working with Teresa Anna Steiner\footnote{I cherish our friendship and found our collaborations extremely motivating.}, Pawel Gawrychowski, and Jonas Ellert. I very much enjoyed our shared time in Paris and I hope I can see you all there\footnote{Possibly at the zoo.} another time.
My gratitude also goes to my former internship supervisors for their guidance: Jakub Radosweski, Travis Gagie, Gonzalo Navarro, and Roberto Grossi.
Many thanks to the co-authors who I have not mentioned already: Giulia Bernardini, Alessio Conte, Anne Driemel, Tomasz Kociumaka, Grigorios Loukides, Giovanni Manzini, Nadia Pisanti, Solon P. Pissis, Wojciech Rytter, Giulia Punzi, Arseny Shur, Leen Stougie, Michelle Sweering, and Tomasz Walen.
I am also grateful for how the community of the CPM and SPIRE conferences welcomed me with open arms as soon as I arrived.


Merci à ma famille pour tous leurs soutiens et encouragements. Merci à mes parents de m’avoir transmis un peu de leurs passions pour la recherche\footnote{Attention, vous n’allez plus pouvoir me répondre ``Passe ta thèse d’abord !''...} et tellement plus. Merci à Geneviève, ma sœur adorée pour son soutien indéfectible et pour notre sororité tellement importante. Merci à mes grands-parents, Andrée, Gwen, Marc, (dont ceux qui ne sont plus parmi nous), Daniel, Francis, Paulette, et Jean-Pierre, ils ont tant contribué à faire de moi l’adulte que je suis. Merci à mon chat turbulent, Betisou, de m’avoir inspiré la version BD pour enfant de cette thèse (en fin de manuscrit).

Merci à Éloi pour tout son soutien, son amour, et pour avoir accepté de m’épouser. Je t’aime et je souhaite que nous ayons encore de nombreuses années à se reposer l’un sur l’autre. Merci aussi à ma belle-famille pour leur accueil chaleureux, leur soutien et leur intérêt pour ma thèse qui n’aura cessé de me surprendre.


Merci à tous mes amis de m’avoir écouté, réconforté, conseillé, et distraite (et pour tout le reste) pendant 3 ans, la vie serait tellement moins douce sans vous.
Merci à Antoine P. pour m’avoir beaucoup écouté râler et pour avoir toujours été de très bon conseils. Merci à Paul d’être un ami incroyable à tout épreuve, toujours présent et calme en toute circonstances. Merci à mes amis de collège-lycée d’être toujours là après tant d’années : Diane, Enguerrand, Johanna, Mathieu (et Goshia !), Martin, malgré nos rencontres qui se font plus rares, vous comptez beaucoup. Merci à mes amis rencontrés par l’ENS, Rémi, Lucas, mon mentor Jill-Jênn, Tomoko. Merci à tous mes amis rencontrés par Prologin\footnote{Association dont je suis membre depuis 2018, qui organise un concours d’informatique à l’échelle nationale et des stages d’informatique pour les collégiennes et lycéennes.}, Antoine M., Florian, Joël, Matthieu, Maya E., Thibault, Victor, et Victoria.
