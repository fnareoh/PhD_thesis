\usepackage[english]{babel}				% Set language
\usepackage[utf8]{inputenc}			% Set encoding
\usepackage{tikz}
\usepackage{svg}

\usepackage{ifthen}

\usetikzlibrary{calc}

\mode<presentation>						% Set options
{
  \usetheme{default}					% Set theme
  \usecolortheme{default} 				% Set colors
  \usefonttheme{default}  				% Set font theme
  \setbeamertemplate{caption}[numbered]	% Set caption to be numbered
}


% emojis
\usepackage{fontawesome}
%\usepackage{fontspec}
%\newfontfamily\DejaSans{DejaVu Sans}

%\usecolortheme[named=UBCblue]{structure}
\setbeamercolor{math text}{fg=black!30!blue}

% Uncomment this to have the outline at the beginning of each section highlighted.
%\AtBeginSection[]
%{
%  \begin{frame}{Outline}
%    \tableofcontents[currentsection]
%  \end{frame}
%}

\usepackage{graphicx}					% For including figures
\usepackage{booktabs}					% For table rules
\usepackage{hyperref}					% For cross-referencing
\usepackage[backend=biber]{biblatex}
\usepackage{csquotes}
\usepackage{bookmark}
\usepackage{colortbl}                   % For color line
\usepackage{framed}                     % For frames

\beamertemplatenavigationsymbolsempty
\setbeamertemplate{footline}[frame number]

\setbeamerfont{author}{size=\large}
\setbeamerfont{date}{size=\scriptsize}
\setbeamercolor{date}{fg=gray}
\defbeamertemplate*{title page}{customized}[1][]
{
  \centering
  \vspace{1cm}
  \usebeamerfont{title}\textbf{\inserttitle}\par
  \bigskip
  \usebeamerfont{author}\insertauthor\par
  \medskip
  \usebeamercolor[fg]{date}\usebeamerfont{date}\insertdate\par
}


\definecolor{myred}{RGB}{176, 38, 0}
\definecolor{myorange}{RGB}{255, 170, 0}
\definecolor{myblue}{RGB}{0, 157, 196}
\definecolor{mygreen}{RGB}{0, 140, 65}
\newcommand{\bred}[1]{\textcolor{myred}{\textbf{#1}}}
\newcommand{\borange}[1]{\textcolor{myorange}{\textbf{#1}}}
\newcommand{\bblue}[1]{\textcolor{myblue}{\textbf{#1}}}
\newcommand{\bgreen}[1]{\textcolor{mygreen}{\textbf{#1}}}
\newcommand{\btheme}[1]{\textcolor{black!30!blue}{\textbf{#1}}}
\newcommand{\ntheme}[1]{\textcolor{black!30!blue}{{#1}}}

\newcommand{\mytitle}[1]{\centering{\Huge #1}}

\setbeamercolor{block body alerted}{bg=alerted text.fg!10}
\setbeamercolor{block title alerted}{bg=alerted text.fg!20}
\setbeamercolor{block body}{bg=white}
\setbeamercolor{block title}{bg=structure!20}
\setbeamercolor{block body example}{bg=white}
\setbeamercolor{block title example}{bg=myblue!20, fg=myblue}
\setbeamertemplate{blocks}[rounded][shadow]

\newcommand{\Oh}{\mathcal{O}}
\def\polylog{\operatorname{polylog}}


\newcommand\beamermathcolor[1]{\setbeamercolor{math text}{fg=#1}}

% Color in math mode
\makeatletter
\def\mathcolor#1#{\@mathcolor{#1}}
\def\@mathcolor#1#2#3{%
  \protect\leavevmode
  \begingroup
    \color#1{#2}#3%
  \endgroup
}
\makeatother

\usepackage{pifont}% http://ctan.org/pkg/pifont
\newcommand{\cmark}{\textcolor{mygreen}{\ding{51}}}%
\newcommand{\xmark}{\textcolor{myred}{\ding{55}}}%


\usepackage[framemethod=TikZ]{mdframed}
\newenvironment{framedenv}{%
\begin{mdframed}[roundcorner=1em, outerlinewidth=0px, innerlinecolor=gray, innerlinewidth=.1em]
}{%
\end{mdframed}
}
\newenvironment{myalertblock}[1]{%
\begin{mdframed}[roundcorner=1em, outerlinewidth=0px, innerlinecolor=gray, innerlinewidth=.1em]
\beamermathcolor{red}
\textcolor{red}{#1}
\vspace{0.3em}
\textcolor{gray}{\hrule}
\vspace{0.5em}
}{%
\end{mdframed}
}

\newenvironment{mydefblock}[1]{%
\begin{mdframed}[roundcorner=1em, outerlinewidth=0px, innerlinecolor=gray, innerlinewidth=.1em]
\beamermathcolor{black!30!blue}
\textcolor{black!30!blue}{#1}
\vspace{0.3em}
\textcolor{gray}{\hrule}
\vspace{0.5em}
}{%
\end{mdframed}
}

\newenvironment{mylemblock}[1]{%
\begin{mdframed}[roundcorner=1em, outerlinewidth=0px, innerlinecolor=gray, innerlinewidth=.1em]
\beamermathcolor{mygreen}
\textcolor{mygreen}{#1}
\vspace{0.3em}
\textcolor{gray}{\hrule}
\vspace{0.5em}
}{%
\end{mdframed}
}

\newenvironment{myblock}[1]{%
\begin{mdframed}[roundcorner=1em, outerlinewidth=0px, innerlinecolor=gray, innerlinewidth=.1em]
#1
\vspace{0.3em}
\textcolor{gray}{\hrule}
\vspace{0.5em}
}{%
\end{mdframed}
}


% Frame numbering for appendix
\newcommand{\backupbegin}{
   \newcounter{framenumberappendix}
   \setcounter{framenumberappendix}{\value{framenumber}}
}
\newcommand{\backupend}{
   \addtocounter{framenumberappendix}{-\value{framenumber}}
   \addtocounter{framenumber}{\value{framenumberappendix}} 
}

% Include pdf slides in backup
\usepackage{pdfpages}