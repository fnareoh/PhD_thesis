%à placer dans le dossier Couverture-these

%Figure Tikz pour tracer les figures de la couverture
% tikzpicture to draw the cover


%Il faut rajouter le package \usetikzlibrary{fadings}


 \begin{tikzpicture}
 	        % right triangle 
            \foreach \x in {1,...,35}{
              \draw[very thick, color = couleur-ecole-recto] (14.015,6.4+\x*4.128/35-0.025) -- (\paperwidth,4.428+\x*4.128/35-0.025);
            }
            \fill[white] (14,6.4) -- (14,8.556) -- (14, 10.8) -- (\paperwidth, 10.8) -- (\paperwidth, 8.556) -- cycle; 

	        % middle triangle
            \foreach \x in {1,...,35}{
              \draw[very thick, color = couleur-ecole-recto] (7.02,10.528 +\x*4.128/35-0.025) -- (14,8.556+\x*4.128/35-0.025);
            }
            \fill[white] (7,10.528) rectangle (14.02, 17);
            \fill[white] (13.995,8.56) rectangle (14.1, 13);
            \shade[shading=axis, top color = white, path fading = south] (7,8.556) rectangle (14,10.54);
            
          % filled hexagon
            \fill[couleur-ecole-recto, fill opacity = 1] (-1,0) -- (-1,12.7817142857) -- (14,8.556) -- (14,6.4) -- (\paperwidth,4.4281) -- (\paperwidth,0) -- cycle;
  		
		      %points
            %à placer dans le dossier Couverture-these

\node[opacity =0.9796931825625235] () at (16.372320060385647,6.070068775173975) {\textcolor{couleur-ecole-recto}{$\cdot$}};
\node[opacity =0.9208370335252852] () at (14.834873590297219,6.703526242863132) {\textcolor{couleur-ecole-recto}{$\cdot$}};
\node[opacity =0.5136601177961507] () at (15.944345950392858,5.850567509516524) {\textcolor{couleur-ecole-recto}{$\cdot$}};
\node[opacity =0.4303155914944766] () at (18.47954602260227,5.961082729003394) {\textcolor{couleur-ecole-recto}{$\cdot$}};
\node[opacity =0.11309370169213262] () at (20.550176182762364,7.122771486174603) {\textcolor{couleur-ecole-recto}{$\cdot$}};
\node[opacity =0.9902103467559072] () at (16.664753913262338,6.0245073351720055) {\textcolor{couleur-ecole-recto}{$\cdot$}};
\node[opacity =0.20659552273658088] () at (17.247468303342394,9.081283489982452) {\textcolor{couleur-ecole-recto}{$\cdot$}};
\node[opacity =0.5018437176376093] () at (15.24259008899358,6.177757881612098) {\textcolor{couleur-ecole-recto}{$\cdot$}};
\node[opacity =0.7000500790363267] () at (14.873942972530248,5.501004995413543) {\textcolor{couleur-ecole-recto}{$\cdot$}};
\node[opacity =0.8776514846552504] () at (20.267620291587445,7.283308946303215) {\textcolor{couleur-ecole-recto}{$\cdot$}};
\node[opacity =0.16933896650095948] () at (18.95293434284131,5.457697276614414) {\textcolor{couleur-ecole-recto}{$\cdot$}};
\node[opacity =0.27451716867719744] () at (18.512496721739353,6.721405384938501) {\textcolor{couleur-ecole-recto}{$\cdot$}};
\node[opacity =0.6763690263455157] () at (14.259526803678023,9.182276341430534) {\textcolor{couleur-ecole-recto}{$\cdot$}};
\node[opacity =0.8788396137601744] () at (18.1606758544879,6.8018844938931515) {\textcolor{couleur-ecole-recto}{$\cdot$}};
\node[opacity =0.16572923410595197] () at (20.194525843619395,6.465000555334003) {\textcolor{couleur-ecole-recto}{$\cdot$}};
\node[opacity =0.8919516960644462] () at (15.232700846155543,8.065443348768667) {\textcolor{couleur-ecole-recto}{$\cdot$}};
\node[opacity =0.0784237035038925] () at (17.232982714065407,4.412365837482251) {\textcolor{couleur-ecole-recto}{$\cdot$}};
\node[opacity =0.4160537611596882] () at (18.98263818987754,5.432326544170962) {\textcolor{couleur-ecole-recto}{$\cdot$}};
\node[opacity =0.49514846055971773] () at (18.569245373886673,8.340451006022745) {\textcolor{couleur-ecole-recto}{$\cdot$}};
\node[opacity =0.3872041237666599] () at (15.550243515633454,8.707357289056008) {\textcolor{couleur-ecole-recto}{$\cdot$}};
\node[opacity =0.4987716610373546] () at (15.849635170125769,5.465498659539766) {\textcolor{couleur-ecole-recto}{$\cdot$}};
\node[opacity =0.6304747871093898] () at (17.20084726905046,7.75833609234053) {\textcolor{couleur-ecole-recto}{$\cdot$}};
\node[opacity =0.05562549063315336] () at (20.426669755359537,5.766485821140674) {\textcolor{couleur-ecole-recto}{$\cdot$}};
\node[opacity =0.9684248809780561] () at (20.272184669850457,9.00451759950316) {\textcolor{couleur-ecole-recto}{$\cdot$}};
\node[opacity =0.8307462811355019] () at (15.259185173378471,5.654072023597388) {\textcolor{couleur-ecole-recto}{$\cdot$}};
\node[opacity =0.6314797249024586] () at (16.46726226812196,9.276598851690272) {\textcolor{couleur-ecole-recto}{$\cdot$}};
\node[opacity =0.2406309968721071] () at (19.847173973704678,6.007991270351617) {\textcolor{couleur-ecole-recto}{$\cdot$}};
\node[opacity =0.8245083616625892] () at (15.737522784527947,7.855863864299121) {\textcolor{couleur-ecole-recto}{$\cdot$}};
\node[opacity =0.428374722019619] () at (19.362267121371282,8.365989814238077) {\textcolor{couleur-ecole-recto}{$\cdot$}};
\node[opacity =0.5688513675485722] () at (18.28737362167469,7.761833184124338) {\textcolor{couleur-ecole-recto}{$\cdot$}};
\node[opacity =0.19941331438051635] () at (19.896384409249215,5.019255251018758) {\textcolor{couleur-ecole-recto}{$\cdot$}};
\node[opacity =0.40525573008896154] () at (20.606860365813624,4.936757295500706) {\textcolor{couleur-ecole-recto}{$\cdot$}};
\node[opacity =0.3538155975142221] () at (14.708684154353197,8.641639219888662) {\textcolor{couleur-ecole-recto}{$\cdot$}};
\node[opacity =0.2513463183837191] () at (16.68354311939399,9.349493842288515) {\textcolor{couleur-ecole-recto}{$\cdot$}};
\node[opacity =0.1862856189311668] () at (18.40793870628202,5.395717828497719) {\textcolor{couleur-ecole-recto}{$\cdot$}};
\node[opacity =0.10240997716941325] () at (16.384174501240874,8.020372633627884) {\textcolor{couleur-ecole-recto}{$\cdot$}};
\node[opacity =0.6072844238578973] () at (20.773296475293677,5.99099801429542) {\textcolor{couleur-ecole-recto}{$\cdot$}};
\node[opacity =0.47687160465586487] () at (16.390242785088176,5.951271092747615) {\textcolor{couleur-ecole-recto}{$\cdot$}};
\node[opacity =0.7989906021693857] () at (15.723236654840072,6.915394344282997) {\textcolor{couleur-ecole-recto}{$\cdot$}};
\node[opacity =0.41123142503495824] () at (17.43785506363981,5.057659817874913) {\textcolor{couleur-ecole-recto}{$\cdot$}};
\node[opacity =0.951586944274749] () at (16.090522769077584,4.857166785633124) {\textcolor{couleur-ecole-recto}{$\cdot$}};
\node[opacity =0.1745892462202564] () at (17.32197262747604,5.81556869445782) {\textcolor{couleur-ecole-recto}{$\cdot$}};
\node[opacity =0.2756485559567089] () at (15.93638714733954,4.703721642633357) {\textcolor{couleur-ecole-recto}{$\cdot$}};
\node[opacity =0.5590381493738347] () at (17.785437554411832,4.704617303514119) {\textcolor{couleur-ecole-recto}{$\cdot$}};
\node[opacity =0.6185353547655652] () at (16.129946929814086,8.232430147791757) {\textcolor{couleur-ecole-recto}{$\cdot$}};
\node[opacity =0.0616736980738003] () at (16.745652200449786,5.404632797619271) {\textcolor{couleur-ecole-recto}{$\cdot$}};
\node[opacity =0.7609531957766943] () at (16.901779395761984,9.047771951111999) {\textcolor{couleur-ecole-recto}{$\cdot$}};
\node[opacity =0.48921626567979815] () at (16.622711089431373,8.31308601164128) {\textcolor{couleur-ecole-recto}{$\cdot$}};
\node[opacity =0.6851802630083875] () at (16.60521319396007,4.795608246655445) {\textcolor{couleur-ecole-recto}{$\cdot$}};
\node[opacity =0.325555500976923] () at (14.944459180682722,4.022682812851948) {\textcolor{couleur-ecole-recto}{$\cdot$}};
\node[opacity =0.6781326169186481] () at (15.633967804868217,8.04142003347734) {\textcolor{couleur-ecole-recto}{$\cdot$}};
\node[opacity =0.12502058998129484] () at (15.300988764623536,5.090248083362985) {\textcolor{couleur-ecole-recto}{$\cdot$}};
\node[opacity =0.5301152133260676] () at (14.150592085130874,8.587877950204344) {\textcolor{couleur-ecole-recto}{$\cdot$}};
\node[opacity =0.7975966312983449] () at (18.674438185068478,8.93342642223082) {\textcolor{couleur-ecole-recto}{$\cdot$}};
\node[opacity =0.18331295613832221] () at (16.453220217090763,9.494386375783499) {\textcolor{couleur-ecole-recto}{$\cdot$}};
\node[opacity =0.3884370045393938] () at (14.76777763264414,6.121960663501486) {\textcolor{couleur-ecole-recto}{$\cdot$}};
\node[opacity =0.07658570457949954] () at (19.380559155264798,9.260321869720386) {\textcolor{couleur-ecole-recto}{$\cdot$}};
\node[opacity =0.9936680315183687] () at (14.845036189640506,4.533473303919334) {\textcolor{couleur-ecole-recto}{$\cdot$}};
\node[opacity =0.8382920220636222] () at (19.930789284263703,7.151856320239732) {\textcolor{couleur-ecole-recto}{$\cdot$}};
\node[opacity =0.834455758003474] () at (16.187361288276833,9.237466656743717) {\textcolor{couleur-ecole-recto}{$\cdot$}};
\node[opacity =0.8096866981538194] () at (20.47014861071773,8.409305814642126) {\textcolor{couleur-ecole-recto}{$\cdot$}};
\node[opacity =0.29773952464765885] () at (16.303233474146325,9.495786489421768) {\textcolor{couleur-ecole-recto}{$\cdot$}};
\node[opacity =0.32410378052349464] () at (14.535804420696236,4.052745499130585) {\textcolor{couleur-ecole-recto}{$\cdot$}};
\node[opacity =0.01965890727261599] () at (19.33144300181546,4.063024793879026) {\textcolor{couleur-ecole-recto}{$\cdot$}};
\node[opacity =0.37953868266320234] () at (17.80741768863589,5.689435818480109) {\textcolor{couleur-ecole-recto}{$\cdot$}};
\node[opacity =0.12220832558638661] () at (20.64458652569702,4.527702196093679) {\textcolor{couleur-ecole-recto}{$\cdot$}};
\node[opacity =0.2360896465304897] () at (17.079826407936032,4.660941008909482) {\textcolor{couleur-ecole-recto}{$\cdot$}};
\node[opacity =0.05301313911054484] () at (14.726639828773251,4.271369553676104) {\textcolor{couleur-ecole-recto}{$\cdot$}};
\node[opacity =0.4727317153915441] () at (16.36281583589323,8.948375775671337) {\textcolor{couleur-ecole-recto}{$\cdot$}};
\node[opacity =0.9577552379505174] () at (18.5298014790227,6.891765734175233) {\textcolor{couleur-ecole-recto}{$\cdot$}};
\node[opacity =0.23196394002415965] () at (16.10887407742683,6.369067732328084) {\textcolor{couleur-ecole-recto}{$\cdot$}};
\node[opacity =0.43813121312948533] () at (14.51752187242694,8.772040038645038) {\textcolor{couleur-ecole-recto}{$\cdot$}};
\node[opacity =0.2407100614892157] () at (19.86080968743579,6.629942854673432) {\textcolor{couleur-ecole-recto}{$\cdot$}};
\node[opacity =0.02110357214399594] () at (15.66577436947636,6.729205262713914) {\textcolor{couleur-ecole-recto}{$\cdot$}};
\node[opacity =0.4670101465690466] () at (15.485635096958326,8.526680458108096) {\textcolor{couleur-ecole-recto}{$\cdot$}};
\node[opacity =0.13623908003651097] () at (17.05374581200147,8.748089940759154) {\textcolor{couleur-ecole-recto}{$\cdot$}};
\node[opacity =0.4445040375550675] () at (17.786148985187474,5.3748456329814385) {\textcolor{couleur-ecole-recto}{$\cdot$}};
\node[opacity =0.642865616986866] () at (14.601586484016165,4.770372457960432) {\textcolor{couleur-ecole-recto}{$\cdot$}};
\node[opacity =0.16668537529569305] () at (17.241289830122277,8.611006080954414) {\textcolor{couleur-ecole-recto}{$\cdot$}};
\node[opacity =0.5706276886084265] () at (17.79431466879794,5.53048894515719) {\textcolor{couleur-ecole-recto}{$\cdot$}};
\node[opacity =0.2401182542583603] () at (20.755199205029516,9.001121358992924) {\textcolor{couleur-ecole-recto}{$\cdot$}};
\node[opacity =0.15394948645337259] () at (20.742495558134443,7.239363117005501) {\textcolor{couleur-ecole-recto}{$\cdot$}};
\node[opacity =0.13332696593549886] () at (19.521034121966547,5.624044670350754) {\textcolor{couleur-ecole-recto}{$\cdot$}};
\node[opacity =0.8708792010624076] () at (18.034587931815363,4.113105413740524) {\textcolor{couleur-ecole-recto}{$\cdot$}};
\node[opacity =0.5916884571807771] () at (20.244313706415056,6.1290207118176685) {\textcolor{couleur-ecole-recto}{$\cdot$}};
\node[opacity =0.21873358464910342] () at (17.439345834181,7.9181210795202635) {\textcolor{couleur-ecole-recto}{$\cdot$}};
\node[opacity =0.8187808536994149] () at (16.561107891379,5.69118918880493) {\textcolor{couleur-ecole-recto}{$\cdot$}};
\node[opacity =0.710641165214654] () at (17.6950091965803,6.977353885852667) {\textcolor{couleur-ecole-recto}{$\cdot$}};
\node[opacity =0.323576222897231] () at (19.829701618134596,7.325563284250256) {\textcolor{couleur-ecole-recto}{$\cdot$}};
\node[opacity =0.22720709832220798] () at (14.525951378276043,8.12405158535182) {\textcolor{couleur-ecole-recto}{$\cdot$}};
\node[opacity =0.2530603805426248] () at (14.10960406534004,6.406661881253324) {\textcolor{couleur-ecole-recto}{$\cdot$}};
\node[opacity =0.35641027441714956] () at (20.213836419074646,9.997519427143377) {\textcolor{couleur-ecole-recto}{$\cdot$}};
\node[opacity =0.2272008148223993] () at (17.441853343657918,9.1617271007412) {\textcolor{couleur-ecole-recto}{$\cdot$}};
\node[opacity =0.40168269330128825] () at (15.811852297080371,5.046933913607757) {\textcolor{couleur-ecole-recto}{$\cdot$}};
\node[opacity =0.30916710467481456] () at (20.352281796678472,7.657942915462192) {\textcolor{couleur-ecole-recto}{$\cdot$}};
\node[opacity =0.18762899174794478] () at (16.04536596708229,5.269299098745688) {\textcolor{couleur-ecole-recto}{$\cdot$}};
\node[opacity =0.8825753288753433] () at (18.290089566379898,5.297492058092459) {\textcolor{couleur-ecole-recto}{$\cdot$}};
\node[opacity =0.6096536658229053] () at (16.565284577463775,8.052741052928956) {\textcolor{couleur-ecole-recto}{$\cdot$}};
\node[opacity =0.046244487164378234] () at (17.981033832226164,4.893282432909755) {\textcolor{couleur-ecole-recto}{$\cdot$}};
\node[opacity =0.7169273915202222] () at (20.980104939299345,7.523942638554699) {\textcolor{couleur-ecole-recto}{$\cdot$}};
\node[opacity =0.6884645768521267] () at (20.631321780652968,8.965936853130236) {\textcolor{couleur-ecole-recto}{$\cdot$}};
\node[opacity =0.35716558818529154] () at (17.241826330170387,5.773110492239742) {\textcolor{couleur-ecole-recto}{$\cdot$}};
\node[opacity =0.9656618503899791] () at (17.965448796225882,4.823934653462146) {\textcolor{couleur-ecole-recto}{$\cdot$}};
\node[opacity =0.9133126386769176] () at (19.3507394866092,9.145640745618774) {\textcolor{couleur-ecole-recto}{$\cdot$}};
\node[opacity =0.18918110555819967] () at (16.776447130349958,8.745562364958156) {\textcolor{couleur-ecole-recto}{$\cdot$}};
\node[opacity =0.3973184360770541] () at (15.264457472424317,7.2280308994536755) {\textcolor{couleur-ecole-recto}{$\cdot$}};
\node[opacity =0.15290137506255774] () at (19.71115687740179,6.413582154054449) {\textcolor{couleur-ecole-recto}{$\cdot$}};
\node[opacity =0.3541556628783984] () at (15.042162514576493,9.069805387508653) {\textcolor{couleur-ecole-recto}{$\cdot$}};
\node[opacity =0.5897240611545163] () at (15.415060675201161,7.332642823864052) {\textcolor{couleur-ecole-recto}{$\cdot$}};
\node[opacity =0.6988475375118923] () at (18.288297374789494,4.108805022226196) {\textcolor{couleur-ecole-recto}{$\cdot$}};
\node[opacity =0.10011448014495761] () at (14.49829773728701,4.766420172844724) {\textcolor{couleur-ecole-recto}{$\cdot$}};
\node[opacity =0.32141332732947003] () at (18.126182839108438,4.758251525379615) {\textcolor{couleur-ecole-recto}{$\cdot$}};
\node[opacity =0.9169828507033106] () at (14.381820404446547,4.749071578842974) {\textcolor{couleur-ecole-recto}{$\cdot$}};
\node[opacity =0.15174585697439147] () at (15.024825380768927,7.42292731228749) {\textcolor{couleur-ecole-recto}{$\cdot$}};
\node[opacity =0.9289031527687274] () at (15.587010493135097,6.9335351269886125) {\textcolor{couleur-ecole-recto}{$\cdot$}};
\node[opacity =0.830538180741619] () at (20.207617931831944,5.586880499757016) {\textcolor{couleur-ecole-recto}{$\cdot$}};
\node[opacity =0.6741406460414977] () at (20.273103336055847,8.40868312635533) {\textcolor{couleur-ecole-recto}{$\cdot$}};
\node[opacity =0.92779796365497] () at (15.839057510054865,5.064668137158218) {\textcolor{couleur-ecole-recto}{$\cdot$}};
\node[opacity =0.7520852035504367] () at (18.12385417760549,8.61869877950778) {\textcolor{couleur-ecole-recto}{$\cdot$}};
\node[opacity =0.11835239290999511] () at (16.846028146898227,6.994675761800069) {\textcolor{couleur-ecole-recto}{$\cdot$}};
\node[opacity =0.24099146351294243] () at (14.03629102044326,6.3644654179977325) {\textcolor{couleur-ecole-recto}{$\cdot$}};
\node[opacity =0.5703878993463384] () at (19.817129704278024,8.83053975836247) {\textcolor{couleur-ecole-recto}{$\cdot$}};
\node[opacity =0.8943649655573458] () at (15.50892187374643,7.965093300362551) {\textcolor{couleur-ecole-recto}{$\cdot$}};
\node[opacity =0.28494000392933416] () at (16.30215852377039,9.004030268697395) {\textcolor{couleur-ecole-recto}{$\cdot$}};
\node[opacity =0.3753819572174484] () at (14.309228499838493,9.017426159517143) {\textcolor{couleur-ecole-recto}{$\cdot$}};
\node[opacity =0.7358211120105723] () at (17.64806972233145,6.6634280330427655) {\textcolor{couleur-ecole-recto}{$\cdot$}};
\node[opacity =0.01905867299414621] () at (18.24925061519407,4.2685772968229845) {\textcolor{couleur-ecole-recto}{$\cdot$}};
\node[opacity =0.02367515142653054] () at (15.052888907016609,5.131125016574433) {\textcolor{couleur-ecole-recto}{$\cdot$}};
\node[opacity =0.447397271815824] () at (14.582736441528498,4.605386016234266) {\textcolor{couleur-ecole-recto}{$\cdot$}};
\node[opacity =0.09480571357971213] () at (17.620918443046655,8.820759963400187) {\textcolor{couleur-ecole-recto}{$\cdot$}};
\node[opacity =0.6930927669367707] () at (14.587634717815323,8.654075257805985) {\textcolor{couleur-ecole-recto}{$\cdot$}};
\node[opacity =0.4995513252077738] () at (20.130252633076122,5.434230402665381) {\textcolor{couleur-ecole-recto}{$\cdot$}};
\node[opacity =0.22652004441172857] () at (15.499617341730666,6.2928139046288045) {\textcolor{couleur-ecole-recto}{$\cdot$}};
\node[opacity =0.5914295325750727] () at (18.295726527878585,7.219404022273611) {\textcolor{couleur-ecole-recto}{$\cdot$}};
\node[opacity =0.44327815683352645] () at (17.212428749283806,4.542841777731979) {\textcolor{couleur-ecole-recto}{$\cdot$}};
\node[opacity =0.09973876369512313] () at (15.25817433331226,7.50951911348562) {\textcolor{couleur-ecole-recto}{$\cdot$}};
\node[opacity =0.0687789270447382] () at (15.362171700303195,9.200980784203653) {\textcolor{couleur-ecole-recto}{$\cdot$}};
\node[opacity =0.5614193389708048] () at (14.755153968962357,7.701898566041933) {\textcolor{couleur-ecole-recto}{$\cdot$}};
\node[opacity =0.2189741429770261] () at (20.187107370814648,7.065942237397667) {\textcolor{couleur-ecole-recto}{$\cdot$}};
\node[opacity =0.7063781384929618] () at (18.47860073173873,9.358179409077753) {\textcolor{couleur-ecole-recto}{$\cdot$}};
\node[opacity =0.13441739317209522] () at (15.707854741243851,5.0596085738991174) {\textcolor{couleur-ecole-recto}{$\cdot$}};
\node[opacity =0.62272778487932] () at (17.081124431500154,8.395423871268298) {\textcolor{couleur-ecole-recto}{$\cdot$}};
\node[opacity =0.5226110421816805] () at (14.083067527930798,7.986100429766779) {\textcolor{couleur-ecole-recto}{$\cdot$}};
\node[opacity =0.8958743673200094] () at (16.805088512495033,8.559206388111697) {\textcolor{couleur-ecole-recto}{$\cdot$}};
\node[opacity =0.5003087856751963] () at (16.16326438512401,5.862020130399125) {\textcolor{couleur-ecole-recto}{$\cdot$}};
\node[opacity =0.2943119037109605] () at (18.304628475045885,5.851450831546716) {\textcolor{couleur-ecole-recto}{$\cdot$}};
\node[opacity =0.8125998688364883] () at (18.70280820957674,5.739785761262934) {\textcolor{couleur-ecole-recto}{$\cdot$}};
\node[opacity =0.12637083533173032] () at (16.50230725670743,5.504197073416781) {\textcolor{couleur-ecole-recto}{$\cdot$}};
\node[opacity =0.9367927982962532] () at (19.086388103945943,4.609804926584261) {\textcolor{couleur-ecole-recto}{$\cdot$}};
\node[opacity =0.9358494388010133] () at (20.960734658703373,7.049878457903706) {\textcolor{couleur-ecole-recto}{$\cdot$}};
\node[opacity =0.11218576449637463] () at (16.09340999357846,9.361174262521146) {\textcolor{couleur-ecole-recto}{$\cdot$}};
\node[opacity =0.7881799863743518] () at (18.365294193165802,9.566767839320985) {\textcolor{couleur-ecole-recto}{$\cdot$}};
\node[opacity =0.43873407484953064] () at (17.06789370573639,5.153561348152994) {\textcolor{couleur-ecole-recto}{$\cdot$}};
\node[opacity =0.7942729909893242] () at (16.97046689679624,8.146167100050189) {\textcolor{couleur-ecole-recto}{$\cdot$}};
\node[opacity =0.12541894417853494] () at (19.194251486002795,6.565146094334681) {\textcolor{couleur-ecole-recto}{$\cdot$}};
\node[opacity =0.11518756031008781] () at (20.220755117980495,7.038344097477156) {\textcolor{couleur-ecole-recto}{$\cdot$}};
\node[opacity =0.7638216939121224] () at (17.3698571526273,9.752560383475632) {\textcolor{couleur-ecole-recto}{$\cdot$}};
\node[opacity =0.7502889664538207] () at (16.01776485758696,7.06746793463445) {\textcolor{couleur-ecole-recto}{$\cdot$}};
\node[opacity =0.4618792964810431] () at (17.81040430075205,8.821579117647486) {\textcolor{couleur-ecole-recto}{$\cdot$}};
\node[opacity =0.3944700218626459] () at (14.494152047119648,4.30664097989834) {\textcolor{couleur-ecole-recto}{$\cdot$}};
\node[opacity =0.4212851520602726] () at (15.26089191697167,5.67876394417792) {\textcolor{couleur-ecole-recto}{$\cdot$}};
\node[opacity =0.6544908325374684] () at (18.774808028698672,5.556924902412889) {\textcolor{couleur-ecole-recto}{$\cdot$}};
\node[opacity =0.8490367638703993] () at (17.520255412825296,5.5673245123511235) {\textcolor{couleur-ecole-recto}{$\cdot$}};
\node[opacity =0.2746091132402819] () at (20.358704026326055,8.457452344739094) {\textcolor{couleur-ecole-recto}{$\cdot$}};
\node[opacity =0.1324060272249965] () at (19.367469383669807,4.946796949923427) {\textcolor{couleur-ecole-recto}{$\cdot$}};
\node[opacity =0.6788431485006402] () at (18.54810409226008,7.043590977011134) {\textcolor{couleur-ecole-recto}{$\cdot$}};
\node[opacity =0.5313581580763369] () at (17.08484681232014,9.053404164171313) {\textcolor{couleur-ecole-recto}{$\cdot$}};
\node[opacity =0.7384038820883927] () at (18.459562644737808,6.07660813213549) {\textcolor{couleur-ecole-recto}{$\cdot$}};
\node[opacity =0.7185563379821461] () at (20.01077479470258,8.784132251653263) {\textcolor{couleur-ecole-recto}{$\cdot$}};
\node[opacity =0.031193661185080268] () at (18.163274018140516,6.219981302962004) {\textcolor{couleur-ecole-recto}{$\cdot$}};
\node[opacity =0.165704805457227] () at (14.11927435433906,6.872398841231613) {\textcolor{couleur-ecole-recto}{$\cdot$}};
\node[opacity =0.827641196155955] () at (17.327149785054157,9.25894013370499) {\textcolor{couleur-ecole-recto}{$\cdot$}};
\node[opacity =0.042339718353376576] () at (16.559017740981663,8.530268008604805) {\textcolor{couleur-ecole-recto}{$\cdot$}};
\node[opacity =0.8886986890373825] () at (19.055997974120487,8.636498129368114) {\textcolor{couleur-ecole-recto}{$\cdot$}};
\node[opacity =0.8984904255928452] () at (17.304365739616838,5.710630440726623) {\textcolor{couleur-ecole-recto}{$\cdot$}};
\node[opacity =0.4802021726749818] () at (15.506085495161479,9.100323372562624) {\textcolor{couleur-ecole-recto}{$\cdot$}};
\node[opacity =0.9346820298002791] () at (14.640182834649476,5.860887825159201) {\textcolor{couleur-ecole-recto}{$\cdot$}};
\node[opacity =0.6339455917738341] () at (18.538734880310713,5.0783331656968755) {\textcolor{couleur-ecole-recto}{$\cdot$}};
\node[opacity =0.2514748352664349] () at (14.610487658660897,7.811814606391765) {\textcolor{couleur-ecole-recto}{$\cdot$}};
\node[opacity =0.3696777857746468] () at (16.68306014581451,7.8451340367498545) {\textcolor{couleur-ecole-recto}{$\cdot$}};
\node[opacity =0.5575030672068306] () at (18.56682566284268,8.025226323738762) {\textcolor{couleur-ecole-recto}{$\cdot$}};
\node[opacity =0.8043099850489975] () at (20.860031094402242,7.203773420801249) {\textcolor{couleur-ecole-recto}{$\cdot$}};
\node[opacity =0.7225536647194963] () at (20.403107705938336,9.740655438154674) {\textcolor{couleur-ecole-recto}{$\cdot$}};
\node[opacity =0.14960040872007552] () at (19.756141151912587,5.5077253810264875) {\textcolor{couleur-ecole-recto}{$\cdot$}};
\node[opacity =0.5139100375199614] () at (18.385205023699335,7.319961635774244) {\textcolor{couleur-ecole-recto}{$\cdot$}};
\node[opacity =0.7224692002803207] () at (15.131312984538912,7.795736022307575) {\textcolor{couleur-ecole-recto}{$\cdot$}};
\node[opacity =0.6341048121071295] () at (20.8822828656765,7.5735332014300205) {\textcolor{couleur-ecole-recto}{$\cdot$}};
\node[opacity =0.8079492127725805] () at (17.504978249202097,9.84479331600695) {\textcolor{couleur-ecole-recto}{$\cdot$}};
\node[opacity =0.24216728958792932] () at (14.658289622701245,7.908245919243781) {\textcolor{couleur-ecole-recto}{$\cdot$}};
\node[opacity =0.512596476786809] () at (17.946136352127766,6.583740644873211) {\textcolor{couleur-ecole-recto}{$\cdot$}};
\node[opacity =0.8549804253116241] () at (20.695802941566228,7.944115830315649) {\textcolor{couleur-ecole-recto}{$\cdot$}};
\node[opacity =0.1733890982847326] () at (20.826841676934336,9.491498825713489) {\textcolor{couleur-ecole-recto}{$\cdot$}};
\node[opacity =0.042592676647320915] () at (15.78329542618224,5.7255148652021415) {\textcolor{couleur-ecole-recto}{$\cdot$}};
\node[opacity =0.22717735053061472] () at (16.053675073916345,9.594118364187654) {\textcolor{couleur-ecole-recto}{$\cdot$}};
\node[opacity =0.2698873758185243] () at (16.31880448669635,9.559056295286798) {\textcolor{couleur-ecole-recto}{$\cdot$}};
\node[opacity =0.027944753463205907] () at (18.94239986810642,6.303602884348939) {\textcolor{couleur-ecole-recto}{$\cdot$}};
\node[opacity =0.41990560467052696] () at (20.239457356662193,9.108857924183564) {\textcolor{couleur-ecole-recto}{$\cdot$}};
\node[opacity =0.9090787660235413] () at (16.71745076434647,4.961753920108986) {\textcolor{couleur-ecole-recto}{$\cdot$}};
\node[opacity =0.3850776651864517] () at (16.613420492019713,6.854280736742227) {\textcolor{couleur-ecole-recto}{$\cdot$}};
\node[opacity =0.03982699435882431] () at (16.439912050688584,4.075695434445467) {\textcolor{couleur-ecole-recto}{$\cdot$}};
\node[opacity =0.04517090751577302] () at (20.72778988561538,7.118374171720718) {\textcolor{couleur-ecole-recto}{$\cdot$}};
\node[opacity =0.06286557277608529] () at (18.332892501421107,6.425627688729129) {\textcolor{couleur-ecole-recto}{$\cdot$}};
\node[opacity =0.8970264327378205] () at (19.783407745133324,4.848740101544452) {\textcolor{couleur-ecole-recto}{$\cdot$}};
\node[opacity =0.035500874604242116] () at (16.640582040279384,7.204295636782276) {\textcolor{couleur-ecole-recto}{$\cdot$}};
\node[opacity =0.48318582868894744] () at (17.03078702985343,4.207527101638141) {\textcolor{couleur-ecole-recto}{$\cdot$}};
\node[opacity =0.9878736416366737] () at (20.427701984456835,9.500645488284258) {\textcolor{couleur-ecole-recto}{$\cdot$}};
\node[opacity =0.8394625044322385] () at (18.354521890424742,7.203325134039919) {\textcolor{couleur-ecole-recto}{$\cdot$}};
\node[opacity =0.8372999158318132] () at (17.53528822410459,9.411512737925118) {\textcolor{couleur-ecole-recto}{$\cdot$}};
\node[opacity =0.2496653874601541] () at (19.287630168497166,8.95079307691612) {\textcolor{couleur-ecole-recto}{$\cdot$}};
\node[opacity =0.5039661593946442] () at (16.785472261318844,6.046488651403887) {\textcolor{couleur-ecole-recto}{$\cdot$}};
\node[opacity =0.2308812222410611] () at (16.77910276143333,5.740163282786003) {\textcolor{couleur-ecole-recto}{$\cdot$}};
\node[opacity =0.8218997201972819] () at (19.384252113368678,4.572092867136752) {\textcolor{couleur-ecole-recto}{$\cdot$}};
\node[opacity =0.5862178835780848] () at (14.233535513482058,8.841033052779576) {\textcolor{couleur-ecole-recto}{$\cdot$}};
\node[opacity =0.9733457839399245] () at (14.777150542962326,8.97292745115066) {\textcolor{couleur-ecole-recto}{$\cdot$}};
\node[opacity =0.9829894583171159] () at (17.568803923687376,9.914280667435442) {\textcolor{couleur-ecole-recto}{$\cdot$}};
\node[opacity =0.703148043286015] () at (18.163134244017407,9.959986547931507) {\textcolor{couleur-ecole-recto}{$\cdot$}};
\node[opacity =0.010027514955258998] () at (15.228726402278362,7.596975244344468) {\textcolor{couleur-ecole-recto}{$\cdot$}};
\node[opacity =0.11552256298071928] () at (18.901260189360215,8.391233886492147) {\textcolor{couleur-ecole-recto}{$\cdot$}};
\node[opacity =0.16890647790611735] () at (14.492430232027015,9.072467812376972) {\textcolor{couleur-ecole-recto}{$\cdot$}};
\node[opacity =0.3443614219901747] () at (19.442540024445087,5.795259403965114) {\textcolor{couleur-ecole-recto}{$\cdot$}};
\node[opacity =0.6421126530482439] () at (16.14289047979787,4.954576245555598) {\textcolor{couleur-ecole-recto}{$\cdot$}};
\node[opacity =0.3033843089914312] () at (17.412610792500093,9.441843314323515) {\textcolor{couleur-ecole-recto}{$\cdot$}};
\node[opacity =0.647112539921562] () at (16.850599081419,8.621263960151607) {\textcolor{couleur-ecole-recto}{$\cdot$}};
\node[opacity =0.21433099844976933] () at (16.655201361981433,9.894461800147013) {\textcolor{couleur-ecole-recto}{$\cdot$}};
\node[opacity =0.5493759953804124] () at (20.317098937124676,6.480059793464935) {\textcolor{couleur-ecole-recto}{$\cdot$}};
\node[opacity =0.7856078876225798] () at (16.558650580477565,4.2687817073626375) {\textcolor{couleur-ecole-recto}{$\cdot$}};
\node[opacity =0.7603943167302147] () at (18.786647475164088,4.4810218339585255) {\textcolor{couleur-ecole-recto}{$\cdot$}};
\node[opacity =0.15906106201840886] () at (19.745700790478683,5.689168674193958) {\textcolor{couleur-ecole-recto}{$\cdot$}};
\node[opacity =0.619869266584996] () at (14.06284292788918,4.810812876434992) {\textcolor{couleur-ecole-recto}{$\cdot$}};
\node[opacity =0.2396058902804986] () at (19.10300429893106,9.893736744457103) {\textcolor{couleur-ecole-recto}{$\cdot$}};
\node[opacity =0.6736524940920003] () at (16.25150647933507,9.271097283051311) {\textcolor{couleur-ecole-recto}{$\cdot$}};
\node[opacity =0.6038657759140548] () at (14.513204955254771,9.237675606999057) {\textcolor{couleur-ecole-recto}{$\cdot$}};
\node[opacity =0.801757336552046] () at (19.699331771195816,6.840116517166072) {\textcolor{couleur-ecole-recto}{$\cdot$}};
\node[opacity =0.6625855978135446] () at (20.336342936790864,8.6494153824439) {\textcolor{couleur-ecole-recto}{$\cdot$}};
\node[opacity =0.5848865457892654] () at (15.633889037739097,9.82123274389733) {\textcolor{couleur-ecole-recto}{$\cdot$}};
\node[opacity =0.7613360288585682] () at (14.02946871317655,8.038634924436048) {\textcolor{couleur-ecole-recto}{$\cdot$}};
\node[opacity =0.36685683568147875] () at (20.096666399046793,7.742019278184882) {\textcolor{couleur-ecole-recto}{$\cdot$}};
\node[opacity =0.40991562762907685] () at (20.856816471611605,7.777373389509458) {\textcolor{couleur-ecole-recto}{$\cdot$}};
\node[opacity =0.40901524620829455] () at (18.028442334708473,9.024163389626153) {\textcolor{couleur-ecole-recto}{$\cdot$}};
\node[opacity =0.10709507241597604] () at (14.619683564815398,7.41335684992457) {\textcolor{couleur-ecole-recto}{$\cdot$}};
\node[opacity =0.6763038876395914] () at (15.778710187069791,4.831145471811843) {\textcolor{couleur-ecole-recto}{$\cdot$}};
\node[opacity =0.6761909863425569] () at (19.975381858331144,7.263339419444231) {\textcolor{couleur-ecole-recto}{$\cdot$}};
\node[opacity =0.7176167548878658] () at (18.792830519370742,7.571046039780939) {\textcolor{couleur-ecole-recto}{$\cdot$}};
\node[opacity =0.11411753633933774] () at (18.791319554701822,4.518090331096216) {\textcolor{couleur-ecole-recto}{$\cdot$}};
\node[opacity =0.44738672722802697] () at (14.155488602496536,7.5060730002501685) {\textcolor{couleur-ecole-recto}{$\cdot$}};
\node[opacity =0.10248232814315839] () at (18.07790894207417,5.096353409215549) {\textcolor{couleur-ecole-recto}{$\cdot$}};
\node[opacity =0.21772429855051056] () at (15.123973706501328,6.877133723822726) {\textcolor{couleur-ecole-recto}{$\cdot$}};
\node[opacity =0.03960556156354722] () at (14.61003871327969,4.894322399195459) {\textcolor{couleur-ecole-recto}{$\cdot$}};
\node[opacity =0.3847211722113604] () at (19.54615110995216,9.203519434752181) {\textcolor{couleur-ecole-recto}{$\cdot$}};
\node[opacity =0.23998127000886527] () at (19.907463052418528,6.575700219034994) {\textcolor{couleur-ecole-recto}{$\cdot$}};
\node[opacity =0.7016216680795135] () at (18.542222144424713,4.413931112771266) {\textcolor{couleur-ecole-recto}{$\cdot$}};
\node[opacity =0.45243959384478627] () at (14.442404203911309,5.1578682439789105) {\textcolor{couleur-ecole-recto}{$\cdot$}};
\node[opacity =0.3448862596436264] () at (17.10315239606215,9.680735521183571) {\textcolor{couleur-ecole-recto}{$\cdot$}};
\node[opacity =0.5096338986073244] () at (16.77177135796072,8.086081153598286) {\textcolor{couleur-ecole-recto}{$\cdot$}};
\node[opacity =0.26044345845956707] () at (14.159573072985776,7.226269695051208) {\textcolor{couleur-ecole-recto}{$\cdot$}};
\node[opacity =0.2948357119324122] () at (18.05943819375431,7.108991098270463) {\textcolor{couleur-ecole-recto}{$\cdot$}};
\node[opacity =0.46564334989397393] () at (19.168521777082937,9.876568577362931) {\textcolor{couleur-ecole-recto}{$\cdot$}};
\node[opacity =0.19751924758899386] () at (16.451846449827556,6.206485668159195) {\textcolor{couleur-ecole-recto}{$\cdot$}};
\node[opacity =0.7719762084160228] () at (15.830091087132836,8.97249997194805) {\textcolor{couleur-ecole-recto}{$\cdot$}};
\node[opacity =0.4516198460097709] () at (14.91358133762263,6.178241693288587) {\textcolor{couleur-ecole-recto}{$\cdot$}};
\node[opacity =0.699436849532013] () at (20.803924393264854,7.829657816454686) {\textcolor{couleur-ecole-recto}{$\cdot$}};
\node[opacity =0.31933267086550543] () at (15.449589217952948,8.514844862208571) {\textcolor{couleur-ecole-recto}{$\cdot$}};
\node[opacity =0.20942467447852253] () at (17.77190208118208,9.945484224462703) {\textcolor{couleur-ecole-recto}{$\cdot$}};
\node[opacity =0.0005214557762991401] () at (15.374157056008393,5.1978936969324385) {\textcolor{couleur-ecole-recto}{$\cdot$}};
\node[opacity =0.6722396173588048] () at (18.424429349701036,8.008067661399288) {\textcolor{couleur-ecole-recto}{$\cdot$}};
\node[opacity =0.7835294274349922] () at (19.011828553818777,5.242532124096962) {\textcolor{couleur-ecole-recto}{$\cdot$}};
\node[opacity =0.4796810349416242] () at (20.206675581554812,6.407156736049188) {\textcolor{couleur-ecole-recto}{$\cdot$}};
\node[opacity =0.08670312258444679] () at (18.638721985188113,8.26580833748471) {\textcolor{couleur-ecole-recto}{$\cdot$}};
\node[opacity =0.28842876510476523] () at (18.787697474682968,9.916811768243871) {\textcolor{couleur-ecole-recto}{$\cdot$}};
\node[opacity =0.999436849532013] () at (18.9042934970107474682,5.04571353544970) {\textcolor{couleur-ecole-recto}{$\cdot$}};
\node[opacity =0.94796810349416242] () at (18.820141290931844970,5.0616490091087) {\textcolor{couleur-ecole-recto}{$\cdot$}};
\node[opacity =0.71933267086550543] () at (18.5065198215591874,5.161893696915705) {\textcolor{couleur-ecole-recto}{$\cdot$}};
\node[opacity =0.74731121690022] () at (18.6065198290161874,5.128794388337833764) {\textcolor{couleur-ecole-recto}{$\cdot$}};
\node[opacity =0.7071618531837844] () at (18.4078824126734,5.1926098168337432119) {\textcolor{couleur-ecole-recto}{$\cdot$}};
\node[opacity =0.9667064948676725] () at (18.109816373469,5.2677098167833732523) {\textcolor{couleur-ecole-recto}{$\cdot$}};
\node[opacity =0.9428519005895681] () at (18.0098163734609816,5.29977833732523) {\textcolor{couleur-ecole-recto}{$\cdot$}};
\node[opacity =0.3217233951827446] () at (16.678824126734817,4.8237325231387835) {\textcolor{couleur-ecole-recto}{$\cdot$}};
\node[opacity =0.6108885775927801] () at (19.405383316155703,6.128635708365742) {\textcolor{couleur-ecole-recto}{$\cdot$}};
\node[opacity =0.4876225603383253] () at (17.42036322378725,7.051396761899676) {\textcolor{couleur-ecole-recto}{$\cdot$}};
\node[opacity =0.6650765677589171] () at (20.024895751229444,8.61133373896888) {\textcolor{couleur-ecole-recto}{$\cdot$}};
\node[opacity =0.9485338414640323] () at (19.88907961415554,9.664790650983662) {\textcolor{couleur-ecole-recto}{$\cdot$}};
\node[opacity =0.64731121690022] () at (17.732056575882556,4.610409637682883) {\textcolor{couleur-ecole-recto}{$\cdot$}};
\node[opacity =0.46675776331243457] () at (15.315403309039084,9.835943211791815) {\textcolor{couleur-ecole-recto}{$\cdot$}};
\node[opacity =0.5116814137579008] () at (17.751287732423847,8.724111008949093) {\textcolor{couleur-ecole-recto}{$\cdot$}};
\node[opacity =0.438577973748585] () at (15.892065198291874,4.79514540555577) {\textcolor{couleur-ecole-recto}{$\cdot$}};
\node[opacity =0.5513108230294121] () at (15.625151133161644,8.841627634699012) {\textcolor{couleur-ecole-recto}{$\cdot$}};
\node[opacity =0.27542999180514727] () at (15.415938027650558,9.352756838342275) {\textcolor{couleur-ecole-recto}{$\cdot$}};
\node[opacity =0.5854005504204473] () at (18.660731081755365,9.146375584754878) {\textcolor{couleur-ecole-recto}{$\cdot$}};
\node[opacity =0.772536106543066] () at (17.685753140704513,4.638641242918491) {\textcolor{couleur-ecole-recto}{$\cdot$}};
\node[opacity =0.33460324390496476] () at (16.52101206254435,8.526177403467658) {\textcolor{couleur-ecole-recto}{$\cdot$}};
\node[opacity =0.5364261548591664] () at (14.264981637346978,8.171050823047137) {\textcolor{couleur-ecole-recto}{$\cdot$}};
\node[opacity =0.4689626040790503] () at (20.74248560412295,9.97491279250994) {\textcolor{couleur-ecole-recto}{$\cdot$}};
\node[opacity =0.5071618531837844] () at (20.826447539062737,6.57354175721661) {\textcolor{couleur-ecole-recto}{$\cdot$}};
\node[opacity =0.9667064948676725] () at (19.714226934498917,4.141378232053148) {\textcolor{couleur-ecole-recto}{$\cdot$}};
\node[opacity =0.6428519005895681] () at (17.12277147673514,9.88583777835037) {\textcolor{couleur-ecole-recto}{$\cdot$}};
\node[opacity =0.13246268619617219] () at (15.873286447126047,9.532347978798956) {\textcolor{couleur-ecole-recto}{$\cdot$}};
\node[opacity =0.34832520946186574] () at (18.955903639167875,4.9060787164423365) {\textcolor{couleur-ecole-recto}{$\cdot$}};
\node[opacity =0.08632943363734291] () at (17.736778208313947,5.389041847733635) {\textcolor{couleur-ecole-recto}{$\cdot$}};
\node[opacity =0.7912451615588657] () at (16.814780278069613,9.593631980765846) {\textcolor{couleur-ecole-recto}{$\cdot$}};
\node[opacity =0.8519801090621839] () at (16.13385662372224,5.333726795747808) {\textcolor{couleur-ecole-recto}{$\cdot$}};
\node[opacity =0.6312247051487918] () at (18.827661371097783,8.25386669150998) {\textcolor{couleur-ecole-recto}{$\cdot$}};
\node[opacity =0.6842282740778207] () at (20.72680793550869,7.334916310883552) {\textcolor{couleur-ecole-recto}{$\cdot$}};
\node[opacity =0.08527750579338145] () at (20.58846473228755,9.968522930798345) {\textcolor{couleur-ecole-recto}{$\cdot$}};
\node[opacity =0.32473155446881186] () at (18.709247015917303,9.843808740045354) {\textcolor{couleur-ecole-recto}{$\cdot$}};
\node[opacity =0.7229963705778092] () at (14.54914647392254,4.900365151623441) {\textcolor{couleur-ecole-recto}{$\cdot$}};
\node[opacity =0.9677421640730162] () at (15.676497959297965,9.840284417437445) {\textcolor{couleur-ecole-recto}{$\cdot$}};
\node[opacity =0.8029800244420149] () at (16.84838388663242,4.537584484575623) {\textcolor{couleur-ecole-recto}{$\cdot$}};
\node[opacity =0.25761390210949886] () at (19.5295808774465,7.865595436048888) {\textcolor{couleur-ecole-recto}{$\cdot$}};
\node[opacity =0.12125547435246031] () at (15.08376578145286,7.750717744993258) {\textcolor{couleur-ecole-recto}{$\cdot$}};
\node[opacity =0.18411983688724876] () at (20.2398594126608,6.3583228852168725) {\textcolor{couleur-ecole-recto}{$\cdot$}};
\node[opacity =0.8484053682765387] () at (20.432698355166043,4.07100281771601) {\textcolor{couleur-ecole-recto}{$\cdot$}};
\node[opacity =0.9015927291251976] () at (15.463649482658862,7.008195470021003) {\textcolor{couleur-ecole-recto}{$\cdot$}};
\node[opacity =0.21198820863648382] () at (18.051149645265312,9.609753993367802) {\textcolor{couleur-ecole-recto}{$\cdot$}};
\node[opacity =0.2012799326547261] () at (17.625852900734113,9.835669910235671) {\textcolor{couleur-ecole-recto}{$\cdot$}};
\node[opacity =0.7196828461608665] () at (14.994021611191723,8.807186877425552) {\textcolor{couleur-ecole-recto}{$\cdot$}};
\node[opacity =0.8580496676163882] () at (18.602967973552154,4.465837588253154) {\textcolor{couleur-ecole-recto}{$\cdot$}};
\node[opacity =0.4481895468173963] () at (17.502260946666542,9.106876883708047) {\textcolor{couleur-ecole-recto}{$\cdot$}};
\node[opacity =0.7970099899768165] () at (17.986979754009177,9.760406489993976) {\textcolor{couleur-ecole-recto}{$\cdot$}};
\node[opacity =0.661101833903077] () at (16.95305286367596,6.4213216334619645) {\textcolor{couleur-ecole-recto}{$\cdot$}};
\node[opacity =0.8453343488168257] () at (19.52120454751572,4.663593637660377) {\textcolor{couleur-ecole-recto}{$\cdot$}};
\node[opacity =0.24455144500482606] () at (20.22108560575997,9.76059014052998) {\textcolor{couleur-ecole-recto}{$\cdot$}};
\node[opacity =0.6433692319412432] () at (20.36996896114677,5.99464171810216) {\textcolor{couleur-ecole-recto}{$\cdot$}};
\node[opacity =0.4880751857678558] () at (16.699246168012344,8.660865027681231) {\textcolor{couleur-ecole-recto}{$\cdot$}};
\node[opacity =0.5985594969899842] () at (20.977382971744643,4.994787853605995) {\textcolor{couleur-ecole-recto}{$\cdot$}};
\node[opacity =0.804465182357859] () at (17.538232716112894,8.501744087588696) {\textcolor{couleur-ecole-recto}{$\cdot$}};
\node[opacity =0.03517465280856913] () at (14.802279250015731,4.819450177441892) {\textcolor{couleur-ecole-recto}{$\cdot$}};
\node[opacity =0.6349111130483185] () at (15.883982058608758,8.793953232658287) {\textcolor{couleur-ecole-recto}{$\cdot$}};
\node[opacity =0.18596086329654282] () at (16.363819985212107,5.911952219291189) {\textcolor{couleur-ecole-recto}{$\cdot$}};
\node[opacity =0.44359611517029096] () at (20.31025546863428,7.580968713964811) {\textcolor{couleur-ecole-recto}{$\cdot$}};
\node[opacity =0.20516972125259703] () at (20.314319988636598,8.83912178689017) {\textcolor{couleur-ecole-recto}{$\cdot$}};
\node[opacity =0.8023688506353308] () at (14.40852763140171,4.76794286820205) {\textcolor{couleur-ecole-recto}{$\cdot$}};
\node[opacity =0.2824073599172142] () at (16.432064747117135,7.625619137505396) {\textcolor{couleur-ecole-recto}{$\cdot$}};
\node[opacity =0.5061108781693211] () at (15.640184176674914,7.725939071719996) {\textcolor{couleur-ecole-recto}{$\cdot$}};
\node[opacity =0.11249681461640348] () at (15.760628505877296,9.483860049541033) {\textcolor{couleur-ecole-recto}{$\cdot$}};
\node[opacity =0.08798777304183159] () at (18.306977034464005,9.201193776157147) {\textcolor{couleur-ecole-recto}{$\cdot$}};
\node[opacity =0.051646657726702405] () at (19.839223338315254,7.147790576455525) {\textcolor{couleur-ecole-recto}{$\cdot$}};
\node[opacity =0.9572026212518111] () at (19.04008978864681,9.944266733900445) {\textcolor{couleur-ecole-recto}{$\cdot$}};
\node[opacity =0.07838234294599888] () at (14.96303452774687,7.306980144450965) {\textcolor{couleur-ecole-recto}{$\cdot$}};
\node[opacity =0.542210606595412] () at (17.871980714216548,6.013583110578299) {\textcolor{couleur-ecole-recto}{$\cdot$}};
\node[opacity =0.23398293940748038] () at (18.082065213982037,7.707063023164087) {\textcolor{couleur-ecole-recto}{$\cdot$}};
\node[opacity =0.09765015864734805] () at (15.63843967837693,9.366264115726523) {\textcolor{couleur-ecole-recto}{$\cdot$}};
\node[opacity =0.484802087437615] () at (15.981216535822599,5.076950082128828) {\textcolor{couleur-ecole-recto}{$\cdot$}};
\node[opacity =0.8723102680873833] () at (19.440186167873343,5.198873426548269) {\textcolor{couleur-ecole-recto}{$\cdot$}};
\node[opacity =0.33143678573347846] () at (20.90385703383403,5.105138137082958) {\textcolor{couleur-ecole-recto}{$\cdot$}};
\node[opacity =0.19352056185998623] () at (18.10105723811951,4.347693605597457) {\textcolor{couleur-ecole-recto}{$\cdot$}};
\node[opacity =0.1474983398725377] () at (14.045180841154377,8.156329718457746) {\textcolor{couleur-ecole-recto}{$\cdot$}};
\node[opacity =0.3802689938270132] () at (15.342360190899294,5.743857721653955) {\textcolor{couleur-ecole-recto}{$\cdot$}};
\node[opacity =0.8038583090587035] () at (18.219236531402686,4.266596246308708) {\textcolor{couleur-ecole-recto}{$\cdot$}};
\node[opacity =0.9878278025247105] () at (20.765871494366504,5.4469809619416925) {\textcolor{couleur-ecole-recto}{$\cdot$}};
\node[opacity =0.21945429795961813] () at (16.589194467917665,6.0030546792364365) {\textcolor{couleur-ecole-recto}{$\cdot$}};
\node[opacity =0.45869928829958206] () at (18.07378611694309,4.194784821523232) {\textcolor{couleur-ecole-recto}{$\cdot$}};
\node[opacity =0.3327478937273487] () at (17.79946479913839,5.758642593848986) {\textcolor{couleur-ecole-recto}{$\cdot$}};
\node[opacity =0.5996585619605412] () at (16.88986097453272,6.336758337541622) {\textcolor{couleur-ecole-recto}{$\cdot$}};
\node[opacity =0.7821765132781843] () at (16.691529529403088,8.383996374496874) {\textcolor{couleur-ecole-recto}{$\cdot$}};
\node[opacity =0.7173188812947754] () at (18.566303370059106,5.207245740446669) {\textcolor{couleur-ecole-recto}{$\cdot$}};
\node[opacity =0.32003553192410816] () at (16.783784347300678,8.19394553697761) {\textcolor{couleur-ecole-recto}{$\cdot$}};
\node[opacity =0.9857892348703694] () at (17.267795664449753,4.59901357379384) {\textcolor{couleur-ecole-recto}{$\cdot$}};
\node[opacity =0.6302374762317279] () at (18.56730857231469,7.573338974978834) {\textcolor{couleur-ecole-recto}{$\cdot$}};
\node[opacity =0.2520304711920126] () at (16.227130801797614,5.468096568088043) {\textcolor{couleur-ecole-recto}{$\cdot$}};
\node[opacity =0.9300496588933451] () at (20.6493805573043,9.290796677027888) {\textcolor{couleur-ecole-recto}{$\cdot$}};
\node[opacity =0.8407388362895625] () at (15.405119744852577,7.218600544603402) {\textcolor{couleur-ecole-recto}{$\cdot$}};
\node[opacity =0.1417131490819029] () at (18.025306360543034,8.918319442532843) {\textcolor{couleur-ecole-recto}{$\cdot$}};
\node[opacity =0.9845013103098854] () at (19.39355814455727,8.81959527480564) {\textcolor{couleur-ecole-recto}{$\cdot$}};
\node[opacity =0.6250718012352485] () at (19.778531441777687,7.453759777433487) {\textcolor{couleur-ecole-recto}{$\cdot$}};
\node[opacity =0.9509267020152721] () at (17.090391799431586,4.193585013324995) {\textcolor{couleur-ecole-recto}{$\cdot$}};
\node[opacity =0.8821058778205187] () at (14.969688480444988,7.221854204603193) {\textcolor{couleur-ecole-recto}{$\cdot$}};
\node[opacity =0.3400674943262745] () at (17.744728646626616,7.8878061782039905) {\textcolor{couleur-ecole-recto}{$\cdot$}};
\node[opacity =0.5397150370606227] () at (18.793762381778766,9.992667992312857) {\textcolor{couleur-ecole-recto}{$\cdot$}};
\node[opacity =0.22762193328392644] () at (16.917535657866075,6.244665654347923) {\textcolor{couleur-ecole-recto}{$\cdot$}};
\node[opacity =0.6574325083191407] () at (14.06559569088408,8.234574253953465) {\textcolor{couleur-ecole-recto}{$\cdot$}};
\node[opacity =0.47322889678769575] () at (15.8289943269764,7.007980129358559) {\textcolor{couleur-ecole-recto}{$\cdot$}};
\node[opacity =0.37905820315564387] () at (15.54708673945646,4.802023359533297) {\textcolor{couleur-ecole-recto}{$\cdot$}};
\node[opacity =0.7681236511904871] () at (20.20032024931824,5.442654095295626) {\textcolor{couleur-ecole-recto}{$\cdot$}};
\node[opacity =0.017828808754723813] () at (19.681451693381337,4.639757737243726) {\textcolor{couleur-ecole-recto}{$\cdot$}};
\node[opacity =0.7748758883227734] () at (14.90363735597842,6.677153025744921) {\textcolor{couleur-ecole-recto}{$\cdot$}};
\node[opacity =0.436710056059616] () at (18.807749433618852,5.941462515874488) {\textcolor{couleur-ecole-recto}{$\cdot$}};
\node[opacity =0.8495165331066818] () at (20.11112782414275,5.668929151281064) {\textcolor{couleur-ecole-recto}{$\cdot$}};
\node[opacity =0.9866012042671526] () at (18.736955198340418,9.121528139373293) {\textcolor{couleur-ecole-recto}{$\cdot$}};
\node[opacity =0.4243836990050872] () at (19.26696567397478,7.372774138008602) {\textcolor{couleur-ecole-recto}{$\cdot$}};
\node[opacity =0.7037999662847438] () at (14.952358721789066,5.112277549801123) {\textcolor{couleur-ecole-recto}{$\cdot$}};
\node[opacity =0.9126655785760455] () at (15.55574097820276,5.895702460319856) {\textcolor{couleur-ecole-recto}{$\cdot$}};
\node[opacity =0.3274117884899227] () at (15.388490893006047,9.565058433475269) {\textcolor{couleur-ecole-recto}{$\cdot$}};
\node[opacity =0.6180423416009839] () at (14.861201545963246,6.9435703182174935) {\textcolor{couleur-ecole-recto}{$\cdot$}};
\node[opacity =0.042032253284733145] () at (15.909513554231932,4.950201847559102) {\textcolor{couleur-ecole-recto}{$\cdot$}};
\node[opacity =0.8755992163074154] () at (16.072764205143564,7.884679148745286) {\textcolor{couleur-ecole-recto}{$\cdot$}};
\node[opacity =0.8803573505912455] () at (20.575266191883923,8.094303757734107) {\textcolor{couleur-ecole-recto}{$\cdot$}};
\node[opacity =0.7553462921332263] () at (18.123152325691194,8.046712792745577) {\textcolor{couleur-ecole-recto}{$\cdot$}};
\node[opacity =0.3917774986343978] () at (17.717465346515432,9.920497342086932) {\textcolor{couleur-ecole-recto}{$\cdot$}};
\node[opacity =0.47261818048445836] () at (14.514674976321922,5.431231446207875) {\textcolor{couleur-ecole-recto}{$\cdot$}};
\node[opacity =0.39674305177147795] () at (15.705837743790504,7.514307325778811) {\textcolor{couleur-ecole-recto}{$\cdot$}};
\node[opacity =0.37474758338895164] () at (18.364706627699306,5.417212612810902) {\textcolor{couleur-ecole-recto}{$\cdot$}};
\node[opacity =0.7807072894469819] () at (19.26328755400055,7.962361802838793) {\textcolor{couleur-ecole-recto}{$\cdot$}};
\node[opacity =0.943162941892976] () at (16.166838366637496,5.52505082660393) {\textcolor{couleur-ecole-recto}{$\cdot$}};
\node[opacity =0.16225853308791738] () at (20.50087341336137,7.723311133840416) {\textcolor{couleur-ecole-recto}{$\cdot$}};
\node[opacity =0.33059456173422364] () at (16.03097855684041,4.312079684709053) {\textcolor{couleur-ecole-recto}{$\cdot$}};
\node[opacity =0.6028593725155637] () at (14.220508138517276,7.08394025534949) {\textcolor{couleur-ecole-recto}{$\cdot$}};
\node[opacity =0.0872445604924984] () at (14.96779414621071,6.59306606626702) {\textcolor{couleur-ecole-recto}{$\cdot$}};
\node[opacity =0.6434025293459054] () at (17.029282225890984,7.0495676254813855) {\textcolor{couleur-ecole-recto}{$\cdot$}};
\node[opacity =0.8474436064537126] () at (16.884077139871824,5.307067622483025) {\textcolor{couleur-ecole-recto}{$\cdot$}};
\node[opacity =0.9968472499512296] () at (14.883221400829864,4.388792561687777) {\textcolor{couleur-ecole-recto}{$\cdot$}};
\node[opacity =0.9103381014604596] () at (15.450901791833742,5.668182058466763) {\textcolor{couleur-ecole-recto}{$\cdot$}};
\node[opacity =0.026501801525729896] () at (14.181252271412715,9.891119213284005) {\textcolor{couleur-ecole-recto}{$\cdot$}};
\node[opacity =0.37426766176477066] () at (15.98455663560376,8.303184557725274) {\textcolor{couleur-ecole-recto}{$\cdot$}};
\node[opacity =0.8104770112999968] () at (19.20233751623351,5.97921044480968) {\textcolor{couleur-ecole-recto}{$\cdot$}};
\node[opacity =0.43167345821672964] () at (19.774494322152783,7.599109961381958) {\textcolor{couleur-ecole-recto}{$\cdot$}};
\node[opacity =0.8961295753774409] () at (14.590035919891056,5.6307302429553925) {\textcolor{couleur-ecole-recto}{$\cdot$}};
\node[opacity =0.11209350083128855] () at (18.753099520342616,4.907496414390944) {\textcolor{couleur-ecole-recto}{$\cdot$}};
\node[opacity =0.7542771376907837] () at (16.582907225415653,4.978522904521558) {\textcolor{couleur-ecole-recto}{$\cdot$}};
\node[opacity =0.658321713863943] () at (14.212457170490767,6.9611296962507945) {\textcolor{couleur-ecole-recto}{$\cdot$}};
\node[opacity =0.10371324157189632] () at (19.42449631907045,7.303997895239533) {\textcolor{couleur-ecole-recto}{$\cdot$}};
\node[opacity =0.41918343056056606] () at (18.772102265551812,5.664322413763346) {\textcolor{couleur-ecole-recto}{$\cdot$}};
\node[opacity =0.40009445309081615] () at (19.21323843447788,9.0264919684763) {\textcolor{couleur-ecole-recto}{$\cdot$}};
\node[opacity =0.534268621974034] () at (19.74002391580199,9.891402094370363) {\textcolor{couleur-ecole-recto}{$\cdot$}};
\node[opacity =0.7745391158203067] () at (17.90782936953889,9.359725920881639) {\textcolor{couleur-ecole-recto}{$\cdot$}};
\node[opacity =0.7850636519774813] () at (15.561492503051264,4.422690563536663) {\textcolor{couleur-ecole-recto}{$\cdot$}};
\node[opacity =0.1253894327383822] () at (14.05509125607344,4.529406399813086) {\textcolor{couleur-ecole-recto}{$\cdot$}};
\node[opacity =0.8239314433202959] () at (17.42044372769277,4.216463557673169) {\textcolor{couleur-ecole-recto}{$\cdot$}};
\node[opacity =0.2529951378658041] () at (20.908538687068514,6.852984247179856) {\textcolor{couleur-ecole-recto}{$\cdot$}};
\node[opacity =0.4386111835733183] () at (17.68883028208949,4.908320812683366) {\textcolor{couleur-ecole-recto}{$\cdot$}};
\node[opacity =0.6625399470174906] () at (19.986105744117353,5.9748442010020035) {\textcolor{couleur-ecole-recto}{$\cdot$}};
\node[opacity =0.3444937236531628] () at (19.054766417183334,7.11788637310053) {\textcolor{couleur-ecole-recto}{$\cdot$}};
\node[opacity =0.14527636558515444] () at (16.154557713930675,7.073532219891491) {\textcolor{couleur-ecole-recto}{$\cdot$}};
\node[opacity =0.16457455889807093] () at (20.5672921491154,9.459609333263725) {\textcolor{couleur-ecole-recto}{$\cdot$}};
\node[opacity =0.43904936441804565] () at (16.22441839529001,9.973341479803427) {\textcolor{couleur-ecole-recto}{$\cdot$}};
\node[opacity =0.001445996665629945] () at (17.107643776047325,9.109114043381604) {\textcolor{couleur-ecole-recto}{$\cdot$}};
\node[opacity =0.6423311033654768] () at (17.679036876731885,9.267287997243749) {\textcolor{couleur-ecole-recto}{$\cdot$}};
\node[opacity =0.32375647585808753] () at (18.06511410041419,6.216689584133854) {\textcolor{couleur-ecole-recto}{$\cdot$}};
\node[opacity =0.11512131900449774] () at (16.12848356843717,4.988711060388486) {\textcolor{couleur-ecole-recto}{$\cdot$}};
\node[opacity =0.5207286490091627] () at (14.110697964130349,5.91056281221479) {\textcolor{couleur-ecole-recto}{$\cdot$}};
\node[opacity =0.6944181338771236] () at (16.112643362656357,5.55724246777817) {\textcolor{couleur-ecole-recto}{$\cdot$}};
\node[opacity =0.4249828043174185] () at (18.61951443699614,5.278321739329618) {\textcolor{couleur-ecole-recto}{$\cdot$}};
\node[opacity =0.384298181736593] () at (20.221274824335786,5.669544355593008) {\textcolor{couleur-ecole-recto}{$\cdot$}};
\node[opacity =0.46043647463670634] () at (15.151383647835125,9.44327873381879) {\textcolor{couleur-ecole-recto}{$\cdot$}};
\node[opacity =0.6306889056383533] () at (16.570871334687432,6.714902617305518) {\textcolor{couleur-ecole-recto}{$\cdot$}};
\node[opacity =0.4283337846542019] () at (15.781920328976774,7.002135677302556) {\textcolor{couleur-ecole-recto}{$\cdot$}};
\node[opacity =0.6606252380251993] () at (15.633597339000623,4.6621669081628685) {\textcolor{couleur-ecole-recto}{$\cdot$}};
\node[opacity =0.05524646305255376] () at (17.735743619789098,9.828474965819927) {\textcolor{couleur-ecole-recto}{$\cdot$}};
\node[opacity =0.24640896373867915] () at (15.701642437401182,4.177688572765655) {\textcolor{couleur-ecole-recto}{$\cdot$}};
\node[opacity =0.14355901272447502] () at (19.8047484554267,8.335035541151012) {\textcolor{couleur-ecole-recto}{$\cdot$}};
\node[opacity =0.04455048566841646] () at (14.849613995512927,7.203716488555431) {\textcolor{couleur-ecole-recto}{$\cdot$}};
\node[opacity =0.7520884487451112] () at (14.015780506808648,8.353362417626293) {\textcolor{couleur-ecole-recto}{$\cdot$}};
\node[opacity =0.49717854392484573] () at (17.545040270417598,8.246550310607752) {\textcolor{couleur-ecole-recto}{$\cdot$}};
\node[opacity =0.8604996931043978] () at (16.495316526640153,7.173513917537136) {\textcolor{couleur-ecole-recto}{$\cdot$}};
\node[opacity =0.38341279287471186] () at (20.703165978913017,8.349324921687789) {\textcolor{couleur-ecole-recto}{$\cdot$}};
\node[opacity =0.24661061790940297] () at (16.22488928342095,8.452016517933979) {\textcolor{couleur-ecole-recto}{$\cdot$}};
\node[opacity =0.5783205212738334] () at (17.441857329098088,6.989112222548011) {\textcolor{couleur-ecole-recto}{$\cdot$}};
\node[opacity =0.7996910498533416] () at (14.963001920792367,6.365552854093442) {\textcolor{couleur-ecole-recto}{$\cdot$}};
\node[opacity =0.35901703481155556] () at (16.979277790400662,4.302175156120643) {\textcolor{couleur-ecole-recto}{$\cdot$}};
\node[opacity =0.9726863419254484] () at (20.184977457512336,9.556817722384354) {\textcolor{couleur-ecole-recto}{$\cdot$}};
\node[opacity =0.4574611696277654] () at (15.266847592440202,7.453420370734181) {\textcolor{couleur-ecole-recto}{$\cdot$}};
\node[opacity =0.23769418016443467] () at (17.305510681112878,9.628334947558539) {\textcolor{couleur-ecole-recto}{$\cdot$}};
\node[opacity =0.24272342311819795] () at (14.157151071459488,4.362307274658505) {\textcolor{couleur-ecole-recto}{$\cdot$}};
\node[opacity =0.45395667118742733] () at (20.80379598017759,9.689110233183596) {\textcolor{couleur-ecole-recto}{$\cdot$}};
\node[opacity =0.9840437202904443] () at (14.735014289899368,5.640863651695066) {\textcolor{couleur-ecole-recto}{$\cdot$}};
\node[opacity =0.7450583241654153] () at (17.053906652364496,5.461952413267759) {\textcolor{couleur-ecole-recto}{$\cdot$}};
\node[opacity =0.4468330065446928] () at (20.224276207347007,7.448056563358503) {\textcolor{couleur-ecole-recto}{$\cdot$}};
\node[opacity =0.8775673585888051] () at (20.194230130201035,7.657206570282354) {\textcolor{couleur-ecole-recto}{$\cdot$}};


            \shade[shading=axis, top color = white, path fading = south] (14,8.556) rectangle (\paperwidth,10);
\end{tikzpicture}