\usepackage{amsthm,amsfonts,amssymb,amsmath} %symboles maths
\usepackage{tabularx} %depassement des tableau hors page
\usepackage{float} %position des figures
\usepackage[utf8]{inputenc}
\usepackage{todonotes}
\usepackage{multirow}
\usepackage{tikz} %figures
\usetikzlibrary{automata,positioning,calc}
\usetikzlibrary{decorations.pathreplacing}
%\usepackage[noend,boxed]{algorithm2e}
\usepackage{enumitem}
\usepackage[capitalize]{cleveref} % square cref


\newcommand{\A}{\mathcal{A}}
\newcommand{\Oh}{\mathcal{O}}
\newcommand{\Ohtilde}{\tilde{\Oh}}
\newcommand{\RLE}{\textsf{RLE}}
\newcommand{\LCP}{\textsf{LCP}}
\newcommand{\poly}{\mathrm{poly}}
\newcommand{\per}{\mathrm{per}}
\def\polylog{\operatorname{polylog}}
\newcommand{\Mod}[1]{\ (\mathrm{mod}\ #1)}

\newcommand{\ed}{\mathsf{ed}}
\newcommand{\gr}{\mathsf{GR}}
\newcommand{\ga}{\mathsf{GA}}

\newtheorem{thm}{Theorem}[section] % reset theorem numbering for each section

\newtheorem{definition}[thm]{Definition}
\newtheorem{observation}[thm]{Observation}
\newtheorem{lemma}[thm]{Lemma}
\newtheorem{theorem}[thm]{Theorem}
\newtheorem{corollary}[thm]{Corollary}
\newtheorem{example}[thm]{Example}
\newtheorem{proposition}[thm]{Proposition}
\newtheorem{fact}[thm]{Fact}
\newtheorem{claim}[thm]{Claim}

\makeatletter
\let\c@proposition\c@thm
\let\c@corollary\c@thm
\let\c@theorem\c@thm
\let\c@lemma\c@thm
\let\c@fact\c@thm
%\let\c@definition\c@thm
%\let\c@example\c@theorem
\makeatother

\renewcommand{\paragraph}[1]{\vspace{1mm}\noindent\textbf{#1}}

   \newcommand{\defproblem}[3]{
  \vspace{2mm}
\noindent\fbox{
  \begin{minipage}{0.96\textwidth}
  #1\\
  #2
  \end{minipage}
  }
  \vspace{2mm}
}

\usepackage{makecell} % NEw line in tabular cell

\renewcommand\theadalign{bc}
\renewcommand\theadfont{\bfseries}
\renewcommand\theadgape{\Gape[4pt]}
\renewcommand\cellgape{\Gape[4pt]}


%Tikzit for figures - it requires compiling with pdflatex
\usepackage{tikzit}
\input{simple.tikzstyles}

\usepackage{caption}
\usepackage{subcaption}


% Squares

\usepackage{thmtools}
\usepackage{thm-restate}
\usepackage{mathtools}
\usetikzlibrary{patterns,calc,arrows,arrows.meta,positioning,fit,shapes}
\usepackage[outline]{contour}

\newtheorem{question}{Question}[section]

% Approximate LCS
\newtheorem{problem}{Problem}{\bfseries}{\itshape}

\usepackage{xspace}
\usepackage[noend]{algpseudocode}
\usepackage{algorithm,algorithmicx}
