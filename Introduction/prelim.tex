\section{Preliminaries}\label{sec:prelim}

This section aims at presenting the basic concepts that are needed to understand the following chapters, as well as presenting the different form of compression used in this work and how they relate to each other.

A string of length $n$ is a sequence $T[1] \dots T[n]$ of characters from a finite alphabet $\Sigma$ of size $\sigma$. The substring $T[i..j]$ is the string $T[i] \cdots T[j]$, whereas the fragment $T[i..j]$ refers to the specific occurrence of $T[i..j]$ starting at position $i$ in $T$. If $i > j$, then $T[i..j]$ is the empty string.

\subsection{Tries and Suffix trees}


\begin{forest}
    for tree={circle,draw, l sep=25pt, inner sep=1pt, minimum size=1.5em, edge={->}}
    % Trie for {CAT, CODE, CODED, COTTAGE, COTTON, TON}
    [
        [,edge label={node[midway,fill=white] {C}} 
            [,edge label={node[midway,fill=white] {A}} 
                [\$, edge label={node[midway,fill=white] {T}}]
            ]
            [, edge label={node[midway,fill=white] {O}}
                [, edge label={node[midway,fill=white] {D}}
                    [\$, edge label={node[midway,fill=white] {E}}
                        [\$, edge label={node[midway,fill=white] {D}}]
                    ]
                ]
                [, edge label={node[midway,fill=white] {T}}
                    [, edge label={node[midway,fill=white] {T}}
                        [, edge label={node[midway,fill=white] {A}}
                            [, edge label={node[midway,fill=white] {G}}
                                [\$, edge label={node[midway,fill=white] {E}}]
                            ]
                        ]
                        [, edge label={node[midway,fill=white] {O}}
                            [\$, edge label={node[midway,fill=white] {N}}]
                        ]
                    ]
                ]
            ]
        ]
        [, edge label={node[midway,fill=white] {T}}
            [, edge label={node[midway,fill=white] {O}}
                [\$, edge label={node[midway,fill=white] {N}}]
            ]
        ] 
    ]
\end{forest}

\subsection{Karp Rabin fingerprint} 
\subsection{Periodicity and Fine-wilf lemma}
Compact run representation of occ crossing a given position

\subsection{Streaming pattern matching}
Application to the explanation of Breslauer \& Gallil (needed for both streaming)

\subsection{Heavy path decomposition}


\subsection{Compression techniques}
What type of compression is used on what problem and why ?
\begin{itemize}
\item RLE
\item Grammar
\item Lempel Ziv
\item BWT
\item State of the art on the equivalence through string attractors
\item Sketches
\end{itemize}