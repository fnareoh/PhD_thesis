\section{Preliminaries}\label{sec:prelim}

This section aims at presenting the basic concepts that are needed to understand the following chapters, as well as presenting the different sketches used in this work and their connections.

A string of length $n$ is a sequence $T[0] \dots T[n-1]$ of characters from a finite alphabet $\Sigma$ of size $\sigma$. The substring $T[i..j]$ is the string $T[i] \cdots T[j]$ (not linked to a specific position), whereas the fragment $T[i..j]$ refers to the specific occurrence of $T[i..j]$ starting at position $i$ in $T$. If $i > j$, then $T[i..j]$ is the empty string. We additionally use the notation $T[i..j)$ and $T(i..j]$ which stand for $T[i..j-1]$ and $T[i+1..j]$ respectively. We call the substrings of the form $T[0..i]$ \emph{prefixes}  and use the notation $T[..i]$, analogously \emph{suffixes} refer to substrings $T[j..n-1]$ and are denoted $T[j..]$.

\subsection{Tries and Suffix Trees}\label{sec:prelim:tries}
Let $\mathcal{S} = {S_1,S_2, ..., S_k}$ be a collection of strings over the alphabet $\Sigma$. Its trie~\cite{thue1912gegenseitige,de1959file,fredkin1960trie} is a rooted tree with edge labels in $\Sigma$.
For any path, we say the path \emph{spells} the string obtained by concatenating the label of the edges of the path.
Likewise, when referring to a node within the tree, we define it as spelling the string spelled by the unique path from the root to that node.
The trie is a tree such that no two nodes spell the same string, each leaf spells a string of $\mathcal{S}$ and each $T_i \in \mathcal{S}$ is spelled by a leaf or an internal node. In both cases, we mark the ending node with a special character $\$$ outside the alphabet $\Sigma$.

The more efficient compacted trie is obtained by contracting  into a single edge any path from a node that is either marked by a \$ or branching (with at least two children) to its lowest ancestor that is either branching or marked by a \$.
This edge is then labelled with the string spelled by the path it replaced.
Let $S$ be a string spelled by a node of the non-compacted trie, if it is also spelled by a node of the compacted trie we say the node is \emph{explicit} else it is \emph{implicit}. Those definitions are illustrated in Figure~\ref{fig:intro:tries_suffix_tree}. Note that for any two nodes $i$ and $j$, their lowest common ancestor spells the longest common prefix between the strings spelled by $i$ and $j$.
% dynamic z-fast tries 
Dynamic z-fast tries~\cite{belazzougui2010dynamic} are not needed in this manuscript but might be interesting for the reader to know that as they have been specifically designed to optimize predecessor and successor (lexicographic order) queries and longest common prefix queries.
%
Notice that all nodes are either branching or marked by a \$ and that all node marked by a \$ spell a different string of $\mathcal{S}$. 
Hence, there are $\Oh(k)$ nodes in the compacted trie.
For each edge, there exists $(i,s,e)$ such that its label is equal to $S_i[s .. e]$, thus we can avoid storing the label explicitly and just store the reference $(i,s,e)$ to the input strings of $\mathcal{S}$.
Thus, $\Oh(k)$ words are sufficient to store the compacted trie.
%
%With the addition of failure links\todo{undef} to each node (which requires a preprocessing time in $\Oh(|S_1| + ... +|S_k|$)), the compacted trie enables the reporting of all occurrences of strings in $\mathcal{S}$ in a text $T$ in time $\Oh(|T|+occ)$ where $occ$ is the total number of occurrences. For a detailed explanation see Chapter 3.4 of~\cite{Gusfield1997}. 


A suffix tree of a string $T$ of length $n$ is the compacted trie containing the suffixes of $T$ i.e. over the collection $\{T[j..]$ for $ j \in [0,n-1] \}$. It can be stored in $O(n)$ space and for linearly-sortable alphabet constructed in $O(n)$ time~\cite{Farach1997}. For more details on the suffix tree see Chapters 5 to 9 of~\cite{Gusfield1997}. We illustrate this definition in Figure~\ref{subfig:suffixtree}. Tries and Suffix Trees are used in Chapter~\ref{chap:regexp} to~\ref{chap:squares}.

\begin{figure}
\centering
\begin{subfigure}[t]{0.25\textwidth}
\centering
\caption{Trie}
\label{subfig:trie}
\footnotesize
\tikzset{centerlabel/.style={midway,fill=white}}
\begin{forest}
    for tree={
            circle,
            draw,
            l sep=18pt,
            minimum width=1.5em,
            minimum height=1.5em,
            inner sep=0pt,
            edge={->},
            font=\strut\footnotesize\sffamily,
        },
    % Trie for {CAT, CODE, CODED, COTTAGE, COTTON, TON}
    [
        [,edge label={node[centerlabel] {C}} 
            [,edge label={node[centerlabel] {A}} 
                [\$, edge label={node[centerlabel] {T}}]
            ]
            [, edge label={node[centerlabel] {O}}
                [, edge label={node[centerlabel] {D}}
                    [\$, edge label={node[centerlabel] {E}}
                        [\$, edge label={node[centerlabel] {D}}]
                    ]
                ]
                [, edge label={node[centerlabel] {T}}
                    [, edge label={node[centerlabel] {T}}
                        [, edge label={node[centerlabel] {A}}
                            [, edge label={node[centerlabel] {G}}
                                [\$, edge label={node[centerlabel] {E}}]
                            ]
                        ]
                        [, edge label={node[centerlabel] {O}}
                            [\$, edge label={node[centerlabel] {N}}]
                        ]
                    ]
                ]
            ]
        ]
        [, edge label={node[centerlabel] {T}}
            [, edge label={node[centerlabel] {O}}
                [\$, edge label={node[centerlabel] {N}}]
            ]
        ] 
    ]
\end{forest}
\end{subfigure}
%
\begin{subfigure}[t]{0.325\textwidth}
\centering
\caption{Compacted trie}
\label{subfig:compacttrie}
\footnotesize
\tikzset{centerlabel/.style={midway,font=\footnotesize}}
\begin{forest}
    for tree={
            circle,
            draw,
            l sep=25pt,
            s sep=20pt,
            minimum width=1.5em,
            minimum height=1.5em,
            inner sep=0pt,
            edge={->},
            font=\strut\footnotesize\sffamily,
        },
    % Compacted Trie for {CAT, CODE, CODED, COTTAGE, COTTON, TON}
    [,
        [,edge label={node[centerlabel,left] {$(1,0,0)$}} 
            [\$, edge label={node[centerlabel,left] {$(1,1,2)$}}]
            [, edge label={node[centerlabel,right] {$(2,1,1)$}}, 
                [\$, edge label={node[centerlabel,left] {$(2,2,3)$}}
                    [\$, edge label={node[centerlabel,left] {$(3,4,4)$}}]
                ]
                [, edge label={node[centerlabel,right] {$(4,2,3)$}}, 
                    [\$, edge label={node[centerlabel,left] {$(4,4,6)$}}]
                    [\$, edge label={node[centerlabel,right] {$(5,4,5)$}}]
                ]
            ]
        ]
        [\$, edge label={node[centerlabel,right] {$(6,0,2)$}}] 
    ]
\end{forest}
\end{subfigure}
%
\tikzset{mylabel/.style={midway,font=\footnotesize,fill=white,sloped}}
\begin{subfigure}[t]{0.4\textwidth}
    \centering
    \caption{Suffix tree}
    \label{subfig:suffixtree}
    \footnotesize
    \begin{forest}
        for tree={
            circle,
            draw,
            l sep=22mm,
            s sep=5mm,
            minimum width=1em,
            minimum height=1em,
            inner sep=0pt,
            edge={->},
            font=\strut\footnotesize\sffamily,
        },
        % suffix tree for 
    [,l sep=18mm,s sep=5mm,
            [,edge label={node[mylabel] {A}}
                [\$, edge label={node[mylabel,rotate=-180] {C}}]
                [\$, edge label={node[mylabel]{GATAC}}]
                [, edge label={node[mylabel] {T}}
                    [\$, edge label={node[mylabel,rotate=-180]{AC}}]
                    [\$, edge label={node[mylabel]{TAGATAC}}]
                ]
            ]
            [\$,edge label={node[mylabel] {C}}]
            [,edge label={node[mylabel] {GAT}}
                [\$,edge label={node[mylabel,rotate=-180] {AC}}]
                [,edge label={node[mylabel] {TAGATAC\$}}]
            ]
            [,edge label={node[mylabel] {T}}
                [,edge label={node[mylabel,rotate=-180] {A}}
                    [\$,edge label={node[mylabel,rotate=-180] {C}}]
                    [\$,edge label={node[mylabel] {GATAC}}]
                ]
                [\$,edge label={node[mylabel] {TAGATAC}}]
            ]
        ]
    \end{forest}
    \end{subfigure}
\caption{Subfigure (\ref{sub@subfig:trie}) shows the trie of $\mathcal{S}=$\{CAT, CODE, CODED, COTTAGE, COTTON, TON\}, (\ref{sub@subfig:compacttrie}) its compacted form and (\ref{sub@subfig:suffixtree}) represents the suffix tree of the string GATTAGATACAT (with full strings instead of references). In (\ref{sub@subfig:trie}) and (\ref{sub@subfig:compacttrie}) the nodes labeled with a \$ spell a string from $\mathcal{S}$. Note that the string COT is spelled by an implicit node of the trie of $\mathcal{S}$.}
\label{fig:intro:tries_suffix_tree}
\end{figure}

\subsection{Heavy Path Decomposition}\label{sec:prelim:HP}

\subsection{Karp--Rabin Fingerprint}\label{sec:prelim:KR}

For $p$ is a prime number and $r \in \mathbb{F}_p$ (the finite field of integer modulo $p$), the Karp--Rabin fingerprint~\cite{DBLP:journals/ibmrd/KarpR87} of a string $S$ is defined as
$\varphi_{p,r}(S) = \sum_{i = 0}^{|S|-1}  S[i]r^{i} \bmod p$.

It is easy to see that if two strings $S_1$ and $S_2$ are equal then $\varphi_{p,r}(S_1) = \varphi_{p,r}(S_2)$. However, the reverse is not true, two distinct string $S_1$ and $S_2$ can have colliding fingerprints.
% Collision
From a theoretical point of view, for a given integer $n$, if $p$ is chosen to be $\Theta(n^k)$ and $r$ is chosen at random in $\mathbb{F}_p$, $|S_1|\leq n$, $|S_2|\leq n$, then $\varphi_{p,r}(S_1) - \varphi_{p,r}(S_2) = 0$ can be seen as a polynomial over $\mathbb{F}_p$. This polynomial has degree greater than zero (because $S_1 \neq S_2$), and at most $\max(|S_1|,|S_2|) < n$ thus it can have at most $n$ roots thus the probability of the fingerprints $S_1$ and $S_2$ colliding is less than $\frac{n}{n^k} = \frac{1}{n^{k-1}}$.
% Practice
In practice $p$ is often chosen as a large primary number around $2^{27}$ to still fit on a 32-bit integer and $r$ is often also hard coded as a parameter (this is the case in the implementation of Chapter~\ref{chap:LCS}), sometimes to a power of two in order to speed up the computation (this is the case in the implementation of Chapter~\ref{chap:XBWT}).

% Sliding window property
This fingerprint has a nice ``sliding'' property in the sense that for a window of a text $T$ and two integers $i,m$ such that $i+m < |T|$, we can express efficiently the fingerprint of $T[i+1..i+m]$ as a function of the fingerprint of $T[i..i+m-1]$, $T[i]$ and $T[i+m]$:
$$ r \varphi_{p,r}(T[i+1..i+m]) = \varphi_{p,r}(T[i..i+m-1])\times r - T[i] + T[i+m]r^k \mod p$$
This allows Karp--Rabin fingerprints to be used for an online matching algorithm running in $\Oh(|P|+|T|)$ that can err with small probability (Monte-Carlo).
% Differences in definitions Gapped
The precise definition can vary slightly depending on the specific property needed. In Chapter~\ref{chap:gapped_index}, the fingerprint of a string $S$ are defined as a triple $(r^{|S|-1} \mod p, r^{-|S|+1} \mod p, \varphi_{p,r}(S))$, to have the property that given three string $X,Y,Z$ such that $XY =Z$, and the fingerprint of any two strings we can deduce in constant time the third fingerprint (Fact~\ref{fact:fingerprints}). In other definitions such as the one in~\cite{Gusfield1997}, $r=2$ is fixed and $p$ is a random prime number chosen bellow a fixed parameter $I$, the collision-probability  analysis is therefore also different.

\subsection{Periodicity and Fine-Wilf Lemma}\label{sec:prelim:FW}
Compact run representation of occ crossing a given position


\subsection{Compression Techniques}\label{sec:prelim:compress}
What type of compression is used on what problem and why ?
\begin{itemize}
\item RLE
\item Grammar
\item Lempel Ziv
\item BWT
\item State of the art on the equivalence through string attractors
\item Sketches
\end{itemize}