\chapter*{Introduction}\label{chap:intro}
\addcontentsline{toc}{chapter}{Introduction}
\chaptermark{Introduction}

When answering the classic question "What is your PhD about?" to family and friends, I always start form the "Ctrl + F" function in their favorite text editor or web browser. This quickly highlights one of the applications of the exact pattern matching problem. If I am feeling ambitious in my explanations I will try to give the intuition of the naive $\Oh(nm)$ algorithm by having them picture a young child, aligning the word against every position of the text and comparing character by character because he has yet to learned how to read. To give a glimpse at more complex solution, I proceed to comment on how, depending on the word, the child may try to skip portions of the text.

At the same time, they immediately know searching in text quickly has been possible for decades and that it cannot be my real research subject. Indeed, exact pattern matching has been long studied, with in particular the famous Knuth-Morris-Pratt algorithm\footnote{The elegance of this algorithm is what first drew me in this area of research as a bachelor student!} published in 1977~\cite{KMP} after being independently discovered by Morris-Pratt in a technical report in 1970 and Knuth in 1973. Since, this has become one of the classic text book algorithm and Charras and Lecroq published a detailed handbook~\cite{charras2004handbook} on the various solutions to exact pattern matching.

% Define center type column
\newcolumntype{Y}{>{\centering\arraybackslash}X}
\newcolumntype{P}[1]{>{\centering\arraybackslash}p{#1}}

%spacing
\renewcommand{\arraystretch}{2}
\begin{figure}[h]
    \begin{tabularx}{\textwidth}{l P{4.5cm}  Y }
        Matching model & Pattern & Text with occurences underlined \\
        \hline
        Error bound (for ED) & $P=$ GATTACAT & $T=$ AT\underline{GATTAACAT}ATA, $\mathrm{ED}(P,T[2..10])=1$ \\
        Don't care & $P=$ GAT**CAT & \underline{GATTACAT}A\underline{GATOACAT}AC\\
        %
        Elastic Degenerate  & $P=$ GATTACAT &  $T=$ {\renewcommand{\arraystretch}{1} AT\underline{GAT}$\left\{
            \begin{array}{l}
                \mathrm{\underline{TA}}  \\
                \mathrm{O}
            \end{array}\right\} \mathrm{\underline{CAT}A}$} \\
        %
        Abelian/Jumbled & $P=$ GATTACAT & $T=$ AGAG\underline{TATGATC}AGT\\
        %
        Parametrized & $P=$ GATTACAT & $T=$ OPO\underline{POGGODOG}O, {\footnotesize A:O, C:D, G:P, T:G} \\
        %
        Regular Expression & $P=$ GAT$(\mathrm{TA}\mid \mathrm{O})(\mathrm{CAT})^*$ & $T=$ \underline{GATTA}AT\underline{GATOCATCATCATCAT}A \\
        Gapped consecutive & $P_1=$ GATTA $P_2=$ TAC  $a=2$, $b=6$ & $T=$ AGG\underline{GATTAC}TAC, $d=3 \in [a,b]$\\
        Order preserving & $P = 1 5 3 4 6 2$ & $T= 2 7 4 5 8 3 1 20 15 16 25 6 $  \\
    \end{tabularx}
    \todo[inline]{For order preserving have a drawing of the shapes. Add citations for each model.}
    \caption{Example of various model of matching on strings.}
    \label{fig:intro:match_model}
\end{figure}

However, the need for text processing goes far beyond exact pattern matching. Here after I give an overview of various text processing problems and models of matching along with their motivations. Figure~\ref{fig:intro:match_model} also provides an example for each matching model.
% Regular expression
One of the oldest and most classic model for more complex queries are regular expression introduced by Kleene in 1951~\cite{RM-704}.
% Briefly explain the formalism
This formalism compactly describes recursively a set of strings starting from three operators, concatenation union and Kleene star.
% Application and Limitations
It has deep connection with automatons~\cite{Thompson_automaton} and its versatile nature makes it a key tool in many fields such as internet traffic analysis~\cite{4221791,4579527}, databases and data mining~\cite{1000341,10.5555/645927.672035,10.1145/375551.375569}, computer networks~\cite{10.1145/1159913.1159952}, protein search~\cite{10.1145/369133.369220}. Chapter~\ref{chap:regexp} provides a new algorithm for Regular expression and pattern matching.

% Similarity measures
Although regular expression is a powerful framework, Bioinformatics\cite{Gusfield1997}, music analysis~\cite{mongeau1990comparison} and plagiarism detection~\cite{lukashenko2007computer} also need relevant and efficient similarity measures such as the Levenshtein distance~\cite{levenshtein1966binary} or Dynamic Time warping distance~\cite{sakoe1978dynamic}. They also often need to report all occurrences with an error bound\cite{landau1986efficient,landau1989fast}: at distance at most a threshold $\tau$.
We contribute to this line of research in Chapter~\ref{chap:LCS} and~\ref{chap:DTW}.
% Don't care
As an alternative, Fischer and Paterson~\cite{fischer1974string} introduced "don't care" matching where a don't care symbol denoted * can occur in both the pattern and the text which match to any other character of the alphabet (but only one).
% Gapped matching
This "don't care" matching model is sometimes referred to as "gapped" matching, however it is not to be confused with gapped consecutive matching~\cite{bille2022gapped} where we are given two patterns $P_1$ and $P_2$ as well as an interval $[a,b]$ and must report all occurrences of $P_1$ and $P_2$ with the distance in $[a,b]$ and no occurrences in between. This model also has connections to spaced seeds~\cite{burkhardt2003better}, and we study it in Chapter~\ref{chap:gapped_pm} and~\ref{chap:gapped_index}.

% (Elastic) Degenerate strings
The modelization of flexible and diverse DNA sequences (such as the IUPAC encoding~\cite{comm1970iupac}) lead to the model of degenerate string~\cite{abrahamson1987generalized} and more recently elastic degenerate strings~\cite{iliopoulos2021efficient}.
% Abelian/jumbled/many other name 
The automatic discovery of clusters of genes in genomes (where they can occur in a different order but still be linked to the same function) motivated~\cite{eres2004permutation} the model of Abelian matching (also known as jumbled, permutation, compomers matching and many other names). In this model a string (or a substring) is entirely identified by the letter it contains (with multiplicities), disregarding their order. The thesis~\cite{ejaz2010abelian} of Tahir Ejaz was dedicated to this model.


% Order preserving
The order preserving model~\cite{kim2014order,kubica2013linear} takes a somewhat opposite approach and considers that two strings match if they have the same relative shape: $\forall i,j \in [0,n-1], X[i] < X[j] \leftrightarrow Y[i] < Y[j]$. This matching model naturally captures the trend detection in stock market and music melody matching problems.
%
% Parametrized matching
Another application driven model is the one of parametrized strings or "p-string" introduced by Baker~\cite{baker1993theory}, where two string match if they can be transformed into the other by applying a one-to-tone function renaming the parameters, meant to detect code duplication.


I hope that this incomplete list matching model on string can illustrate their flexibility and the versatility their applications. However, a major issue in many of those applications is scalability. In particular in Bioinformatics, due to the cost of next generation sequencing (NGS) dropping faster than it would be expected by Moore's law~\cite{muir2016real}, efficient algorithms and data structures that can leverage the redundant nature of DNA sequences are badly needed. We explore this specific subject in Chapter~\ref{chap:XBWT}. A good indicator of this phenomenon is the size of the Sequence Read Archive shown in Figure~\ref{fig:intro:sra} since this database tries to capture the major part of the sequencing data produced.

\begin{figure}
    \begin{center}
        \includegraphics[width=0.8\textwidth]{Introduction/sra_database_growth.png}
        \caption{Evolution of the number of DNA bases in the Sequence Read Archive (SRA), captured from \href{https://www.ncbi.nlm.nih.gov/sra/docs/sragrowth/}{[the dynamic version on the SRA website]}.}
        \label{fig:intro:sra}
    \end{center}
\end{figure}

\todo[inline]{Try to give other example of scalibility issues in other domains}

\subsection*{Approaches to handle massive string data}

\begin{itemize}
\item Data structures and Indexing
\item Streaming algorithm
\item Compressed input
\item Approximation algorithms
\end{itemize}

\begin{itemize}
    \item Detail how we can combine the need for complex queries and scalability
    \item Periodicity
\end{itemize}

\section*{Contributions}

Extend the abstract description of how the different project relate to each other.
