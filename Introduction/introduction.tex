\chapter*{Introduction}\label{chap:intro}
\addcontentsline{toc}{chapter}{Introduction}
\chaptermark{Introduction}

When answering the classic question "What is your Ph.D. about?" to family and friends, I always start with the "Ctrl + F" function in their favorite text editor or web browser. This quickly highlights one of the applications of the exact pattern-matching problem. If I feel especially ambitious in my explanations, I will attempt to give the intuition of the naive $\Oh(nm)$ algorithm. Picture a young child, aligning the word against every position of the text and comparing character by character because he has yet to learn how to read. To give a glimpse at a more complex solution, I comment on how, depending on the word, the child may try to skip portions of the text. But even my grandparents immediately know that searching in a text has been possible for decades and that it cannot be my real research subject.

\section{Context}
\subsection{The need for complex queries} 

Indeed, exact pattern matching has been long studied, with in particular the famous Knuth-Morris-Pratt algorithm\footnote{The elegance of this algorithm is what first drew me in this area of research as a bachelor student!} published in 1977~\cite{KMP} after being independently discovered by Morris-Pratt in a technical report in 1970 and Knuth in 1973. Since then, this has become one of the classic textbook algorithms, and Charras and Lecroq published a detailed handbook~\cite{charras2004handbook} on the various solutions to exact pattern matching.

% Define center type column
\newcolumntype{Y}{>{\centering\arraybackslash}X}
\newcolumntype{P}[1]{>{\centering\arraybackslash}p{#1}}

%spacing
\renewcommand{\arraystretch}{2}
\begin{figure}[h]
    \begin{tabularx}{\textwidth}{l P{4.5cm}  Y }
        Matching model & Pattern & Text with occurences underlined \\
        \hline
        Error bound (for ED) & $P=$ GATTACAT & $T=$ AT\underline{GATTAACAT}ATA, $\mathrm{ED}(P,T[2..10])=1$ \\
        Don't care & $P=$ GAT**CAT & \underline{GATTACAT}A\underline{GATOACAT}AC\\
        %
        Elastic Degenerate  & $P=$ GATTACAT &  $T=$ {\renewcommand{\arraystretch}{1} AT\underline{GAT}$\left\{
            \begin{array}{l}
                \mathrm{\underline{TA}}  \\
                \mathrm{O}
            \end{array}\right\} \mathrm{\underline{CAT}A}$} \\
        %
        Abelian/Jumbled & $P=$ GATTACAT & $T=$ AGAG\underline{TATGATC}AGT\\
        %
        Parametrized & $P=$ GATTACAT & $T=$ OPO\underline{POGGODOG}O, {\footnotesize A:O, C:D, G:P, T:G} \\
        %
        Regular Expression & $P=$ GAT$(\mathrm{TA}\mid \mathrm{O})(\mathrm{CAT})^*$ & $T=$ \underline{GATTA}AT\underline{GATOCATCATCATCAT}A \\
        Gapped consecutive & $P_1=$ GATTA $P_2=$ TAC  $a=2$, $b=6$ & $T=$ AGG\underline{GATTAC}TAC, $d=3 \in [a,b]$\\
        Order preserving & $P = 1 5 3 4 6 2$ & $T= 2 7 4 5 8 3 1 20 15 16 25 6 $  \\
    \end{tabularx}
    \todo[inline]{For order preserving have a drawing of the shapes. Add citations for each model.}
    \caption{Example of various model of matching on strings.}
    \label{fig:intro:match_model}
\end{figure}

However, the need for text processing goes far beyond exact pattern matching. To illustrate this claim, here is an overview of various text-processing problems and their motivations. Figure~\ref{fig:intro:match_model} also provides an example for each matching model.
% Regular expression
One of the oldest and most classic models for more complex queries is regular expressions introduced by Kleene in 1951~\cite{RM-704}.
% Briefly explain the formalism
This formalism compactly describes a set of strings recursively starting from three operators, concatenation union and Kleene star.
% Application and Limitations
It has deep connections with automatons~\cite{Thompson_automaton}, and its versatile nature makes it a crucial tool in many fields such as internet traffic analysis~\cite{4221791,4579527}, databases, data mining~\cite{1000341,10.5555/645927.672035,10.1145/375551.375569}, computer networks~\cite{10.1145/1159913.1159952}, and protein search~\cite{10.1145/369133.369220}. Chapter~\ref{chap:regexp} provides a new algorithm for Regular expression and pattern matching.

% Similarity measures
Although regular expressions are powerful, Bioinformatics\cite{Gusfield1997}, music analysis~\cite{mongeau1990comparison} and plagiarism detection~\cite{lukashenko2007computer} also need relevant and efficient similarity measures such as the Levenshtein distance~\cite{levenshtein1966binary} or Dynamic Time warping distance~\cite{sakoe1978dynamic}. They also often need to report all occurrences with an error bound\cite{landau1986efficient,landau1989fast}: at a distance at most a threshold $\tau$.
We contribute to this line of research in Chapter~\ref{chap:LCS} and~\ref{chap:DTW}.
% Don't care
As an alternative, Fischer and Paterson~\cite{fischer1974string} introduced "don't care" matching where a don't care symbol denoted * can occur in both the pattern and the text, matches to any other character of the alphabet (but only one).
% Gapped matching
The "don't care" matching model is sometimes referred to as "gapped" matching; however, it is not to be confused with gapped consecutive matching~\cite{bille2022gapped} where we are given two patterns $P_1$ and $P_2$ as well as an interval $[a,b]$ and must report all occurrences of $P_1$ and $P_2$ with the distance in $[a,b]$ and no occurrences in between. This model also has connections to spaced seeds~\cite{burkhardt2003better}, and we study it in various settings in Chapter~\ref{chap:gapped_stream}, ~\ref{chap:gapped_pm} and~\ref{chap:gapped_index}.

% (Elastic) Degenerate strings
The modelization of flexible and diverse DNA sequences~\cite{comm1970iupac} lead to the model of degenerate string~\cite{abrahamson1987generalized} and more recently elastic degenerate strings~\cite{iliopoulos2021efficient}.
% Abelian/jumbled/many other names 
In the model of Abelian matching, a string (or a substring) is entirely identified by the letter it contains (with multiplicities), disregarding their order. It stems from the automatic discovery of clusters of genes in genomes where they can occur in a different order but still linked to the same function~\cite{eres2004permutation}. This model is also known as jumbled, permutation, compomers matching, and many other names, and Tahir Ejaz dedicated his thesis~\cite{ejaz2010abelian} to this model.
% order preserving
The order-preserving model~\cite{kim2014order,kubica2013linear} takes a somewhat opposite approach and considers that two strings match if they have the same relative shape: $\forall i,j \in [0,n-1], X[i] < X[j] \leftrightarrow Y[i] < Y[j]$. This matching model naturally captures the trend detection in the stock market and music melody matching problems.
%
% Parametrized matching
Another application-driven model is parametrized strings or "p-string" introduced by Baker~\cite{baker1993theory}, where two strings match if we can transform one into the other by applying a one-to-tone function renaming the parameters, meant to detect code duplication.

So far we focussed on matching models where we are given a pattern and text as well as conditions that define a match of the pattern in the text. But another central task in text processing is repetitions detection. By repetitions, we refer to consecutive occurrences of the same fragment. They can be repeated twice (a square), three times (a cube) or more, then represented as a run: a maximal periodic substring. They are needed as a theoretical tool to avoid needless repetitive computations, but they also naturally occur in DNA with an important role in genomic fingerprinting~\cite{Kolpakov2003}.
The study of squares in strings goes back to 1906 with the work of Thue~\cite{thue1906} on the construction of an infinite square free word, in Chapter~\ref{chap:squares} we provide an optimal algorithm for square detection.

\subsection{Scalability Issues}

%%%%%%%%%%%%%%%%%%%%%% Intro scalibility %%%%%%%%%%%%%%%%%%%%%%%%%%%%
 
% But it is not just about the specific model also about scalibility
We detailed how specific applications can motivate particular string processing tasks, but another major challenge in most applications is the scalability to large datasets.
% Wikipedia
Highly curated datasets generally remain quite small, for example the English pages of Wikipedia (just the text and metadata) take up 20~gigabytes in a compressed format as of 2022\footnote{https://web.archive.org/web/20230228010327/https://dumps.wikimedia.org/enwiki/20221201/}. In comparison, any form of archival and version history tends to grow much bigger. Just the metadata of revision's history (without the content of the articles) for the Wikipedia English pages takes up 75~gigabytes still as of 2022.
% Software Heritage
It is sometimes possible to limit the redundancy in the archive data, for example by using a graph which tracks where the data is repeated multiple times. This is the approach taken by the Software Heritage\footnote{https://www.softwareheritage.org/} project, which aims at keeping an archive of entirety of the software code produced by humanity. The graph structure is especially necessary in this project to reduce code redundancy and reflect the standard use of version history management in software development. Through significant research and engineering efforts~\cite{DBLP:phd/hal/Pietri21} the graph is can be explored efficiently, but the code repository are index through their URLs and metadata, it is not possible to search of the occurrences of a specific snippet of code. As of 2023, the graph is limited to 7~terabytes but with the source files the space usage approaches 1~petabyte\footnote{https://www.polytechnique-insights.com/en/columns/economy/source-code-building-a-universal-software-archive/}.
% Internet archive
The internet archive is a non-profit which started saving web pages in 1996 and now holds the history of more than 800 billion web pages through their program: the wayback machine\footnote{https://web.archive.org/}. This archive takes up more than 70~petabytes, however, one of the limitations are the search options limited to the metadata of the websites and not content of the webpages themselves.

%%%%%%%%%% Bioinformatics
%Intro DNA representation
In bioinformatics, DNA is most commonly seen as a string over the nucleotide alphabet \texttt{\{A,T,C,G\}} which can be stored using just 2 bits per base. However, this type of storage does not provide fast queries. At the opposite of the spectrum, a suffix tree allows efficient sequence analysis but requires 10 bytes per base~\cite{navarro2016compact}, which adds up to 30 gigabytes for a human genome containing 3.3 billion bases. A trade-off between those two extreme has been found through the development of compact data structures that exploit redundancy to decrease space usage. For a human genome it allows to represent the sequence and its suffix tree using just 4 gigabytes.\todo{Find a more specific reference/example.} 
% Reads
However, for DNA, another classic data model is the one of reads: when DNA is sequenced, the output is a set of fragments (called \emph{reads}) of the original sequence. The length of those fragments and error rate vary depending on the sequencing techniques. To be able to reconstruct the original sequence, reads are extracted in such quantities that each base is covered multiple times. This means readsets are larger than the assembled genome and even more redundant. In Chapter~\ref{chap:XBWT} we explore this topic for Illumina reads and propose a data structure specifically tailored to sequencing technique's specificities. 
% Cheaper = more sequencing
Additionally, drastic decrease in sequencing cost since 2008 (faster than expected by Moore's law~\cite{muir2016real}) lead higher volumes of DNA being sequenced. 
% ENA
So far, the European Nucleotide Archive has accumulated more than 47 petabytes\footnote{https://www.ebi.ac.uk/ena/browser/about/statistics} of read data.
% SRA
While the NCBI Sequence Read Archive has more than 73 petabases\footnote{https://www.ncbi.nlm.nih.gov/sra/docs/sragrowth/} of archive including 38 petabases in open access. Again, in those datasets the data is indexed by its metadata only.\\
\todo[inline]{Look for more reasonable size example ? It's unclear if anybody would want to index the entire SRA}

%%%%%%%%%%%%%%%%%%%%%% Approaches to adress scalibility %%%%%%%%%%%%%%%%%%%%%%%%%%%%
It goes without saying that algorithms with quadratic time complexity cannot scale to terabytes and petabytes of inputs. Consequently, the choice of query is oriented by the specific needs of the application and the typical scale of the input.
Several (overlapping) approaches are generally used to cope with large amounts of data:
\begin{itemize}
\item Compressed input: as we mentioned several times already, highly redundant data can sometimes be compressed to a manageable size. Then the goal is to design algorithms and data structures that can directly work on the compressed data. This is possibly the most natural approach as the data is almost always shared in a compressed format. Chapter~\ref{chap:gapped_pm} and~\ref{chap:gapped_index} take this approach by working on a grammar compressed text.
\item Distributed algorithms: those are designed to work on several computers sharing a network. The data is processed in parallel with limited transfer over the network (those transfer can be 10 times slower compared to disk transfer). This direction is a research field of its own which is out of the scope of this thesis.
\item Data structures: there is first a processing of the input and build of the data structures in a given \emph{construction} time and space. Then the data structures can be queried on various query input. This preprocessing can generally allow for a small \emph{query} time and space of the data structure. One of the use case is to construct on a powerful server and then send the small data structure to users that can loaded it in main memory and query it repeatedly. Chapter~\ref{chap:gapped_index} and ~\ref{chap:XBWT} are examples of that.
\item Efficient second-memory algorithms: It is  common knowledge that random access to disk are very inefficient, however contiguous reads are only 100 time slower than reading from main memory~\cite{navarro2016compact}. Therefore, an algorithm that uses few random reads can be executed directly on disk and scale much more easily. The main issue with this approach is that not all problems can avoid random access. We partially use this approach to limit main memory usage in chapter~\ref{chap:XBWT}. The construction of the index is split in phases that read contiguously from disk, process the information (for the next phase or final output) and write to disk.
\item Streaming algorithms: Here the reasoning is that the data is so large it can only be handled as a stream that must be processed on the fly without the possibility to go back. Generally the pattern and the length of the text are known in advance and can be preprocessed, then the characters arrive one by one and can only be accessed later if they have been explicitly stored. This model optimizes for both the time per character during the streaming phase and the space complexity which accounts for storing the result of the preprocessing and any space necessary to process the characters. Chapter~\ref{chap:regexp} and~\ref{chap:gapped_stream} are set in this model.
\item Approximation algorithms: On very large dataset it is not always relevant to answer queries exactly. Allowing some approximation in the result can allow to cut-corner and circumvent lower bounds. The entire second section of this thesis is related to this approach but Chapter~\ref{chap:LCS} is perhaps most characteristic. We treat the problem of the Longest Common Subsequence with approximately $k$ mismatches with a probabilistic algorithm that answer correctly with high probability ($\geq 1 - \frac{1}{n}$ for an input of size $n$).
\end{itemize}

\section{Contributions}\label{intro:sec:contrib}

So far, we presented the two core challenges at the heart of text processing: enabling relevant (and sometimes complex) queries suited to specific applications and with performances that can scale to the large volumes of input data.
%
This thesis makes theoretical and practical contributions to address both needs. 
Each contribution is presented as independent chapter corresponding to a publication. This choice was motivated by the variety of subjects and techniques as well as how they were conceived as independent projects (with varying sets of co-authors) during the Ph.D. However, this section is meant to give an overview of the main contributions of the thesis and how they relate.

\todo[inline,color=teal]{Extend the abstract description of how the different project relate to each other.}
