\chapter{Pattern Matching under DTW Distance}
\contextbox{The following chapter is a publication}{
This chapter corresponds to the following publication: %\fullcite{}\\
}

%%%%%%%%%% Math notations %%%%%%%%%%%%
\providecommand{\norm}[1]{\ensuremath{\lVert#1\rVert}}
\providecommand{\ceil}[1]{\ensuremath{\lceil#1\rceil}}
\newcommand{\dtw}{\mathrm{DTW}}
\newcommand{\ham}{\mathrm{HAM}}
\providecommand{\dd}{\mathinner{.\,.\allowbreak}}
\providecommand{\RLE}{\mathrm{RLE}}
\providecommand{\Oh}{\mathcal{O}}
\providecommand{\eps}{\varepsilon}
\newcommand{\invacker}{\alpha}

\newcommand{\inputDTW}[1]{\input{Part_Two/dtw/#1}}
\newcommand{\figDTW}[2]{\includegraphics[scale=#1]{Part_Two/dtw/#2}}

\begin{small}
In this work, we consider the problem of pattern matching under the dynamic time warping ($\dtw$) distance motivated by potential applications in the analysis of biological data produced by the third generation sequencing. To measure the $\dtw$ distance between two strings, one must ``warp'' them, that is, double some letters in the strings to obtain two equal-lengths strings, and then sum the distances between the letters in the corresponding positions. When the distances between letters are integers, we show that for a pattern $P$ with $m$ runs and a text~$T$ with $n$ runs:
\begin{enumerate}
\item There is an $\Oh(m+n)$-time algorithm that computes all locations where the $\dtw$ distance from $P$ to $T$ is at most $1$;
\item There is an $\Oh(kmn)$-time algorithm that computes all locations where the $\dtw$ distance from $P$ to $T$ is at most $k$.
\end{enumerate}
As a corollary of the second result, we also derive an approximation algorithm for general metrics on the alphabet. 
\end{small}

\section{Introduction}
\inputDTW{intro}

\section{Preliminaries}\label{dtw:sec:prelim}
\inputDTW{prelim}

\section{Linear algorithm for \texorpdfstring{$k=1$}{kone}}
\label{dtw:sec:lce}
\inputDTW{lce}

\section{Main result: \texorpdfstring{$\Oh(kmn)$}{Okm}-time algorithm}
\label{dtw:sec:block}
\inputDTW{block_new}

\section{Approximation algorithm}
\label{dtw:sec:approx}
\inputDTW{approx}

\section{Experiments}\label{dtw:sec:experiments}
\inputDTW{experiments}


\BiblatexSplitbibDefernumbersWarningOff

\backmatter
\todo[inline]{Fix links to some DOI not working}
\printbibliography[heading=subbibintoc]