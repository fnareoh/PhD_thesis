In this section, we prove Theorem~\ref{th:klcs_upper}. Let us first introduce a decision variant of the \kApproxLCS problem. 

\begin{problem}\label{pr:LCS'k-decision}
Two strings $X, Y$ of length at most $n$, integers $k, \ell$, and a constant $\eps > 0$ are given. We must return:
%
\begin{enumerate}
\item YES if $\ell \le \lcsk(X,Y)$;
\item Anything if $\lcsk(X,Y) < \ell \le \mathrm{LCS}_{(1+\eps)k}(X,Y)$; 
\item NO if $\mathrm{LCS}_{(1+\eps)k}(X,Y) < \ell$.
\end{enumerate}
%
If we return YES, we must also give a \emph{witness pair} of length-$\ell$ substrings $S_1$ and $S_2$ of $X$ and $Y$, respectively, such that $\HD(S_1,S_2)\le (1+\eps)k$. 
\end{problem}

The decision variant of the \kApproxLCS problem can be reduced to the following $(c,r)$-\NN problem. 

\begin{problem}\label{pr:NN}
In the $(c,r)$-\NN problem with failure probability~$f$, the aim is, given a set $P$ of $n$ points in $\mathbb{R}^d$, to construct a data structure supporting the following queries: given any point $q\in \mathbb{R}^d$, if there exists $p \in P$ such that $\norm{p-q} \leq r$, then return some point $p' \in P$ such that $\norm{p'-q} \leq cr$ with probability at least $1-f$.
\end{problem}

Using the reduction, we will show our first solution to the \kApproxLCS decision problem based on the result of Andoni and Razenshteyn~\cite{DBLP:conf/stoc/AndoniR15}, who showed that for any constant $f$, there is a data structure for the $(c,r)$-\NN problem that has $\Oh(n^{1+\rho+o(1)}+d \cdot n)$ size, $\Oh(d\cdot n^{\rho+o(1)})$ query time, and $\Oh(d \cdot n^{1+\rho+o(1)})$ preprocessing time, where $\rho = 1/(2c^2-1)$.

\begin{lemma}\label{lm:opt_NN}
Assume an alphabet of constant size $\sigma$. The decision variant of the \kApproxLCS problem can be solved in space $\Oh(n^{1+ 1/(1+2\eps) + o(1)})$ and time $\Oh(n^{1+ 1/(1+2\eps) + o(1)})$. The answer is correct with constant probability. 
\end{lemma}
\begin{proof}
Let $P$ be the set of all length-$\ell$ substrings of $X$ and $Q$ be the set of all length-$\ell$ substrings of $Y$,
all encoded in binary using the morphism $\mu$ (see Section~\ref{sec:prelim}). We start by applying the dimension reduction procedure of Corollary~\ref{cor:dim_reduction} to $P$ and $Q$ with $\alpha = 1/(\log\log n)^{\Theta(1)}$ and $\beta = 2$ to obtain sets $P'$ and $Q'$. We can implement the procedure in $\Oh(\sigma n \log^2 n (\log\log n)^{\Theta(1)}) = \Oh(n \log^{2+o(1)} n)$ time by encoding $X, Y$ using $\mu$ and running the FFT algorithm~\cite{FischerPaterson} for each of the $\Oh(\log^{1+o(1)} n)$ rows of the matrix and $\mu(X), \mu(Y)$. 

To solve the decision variant of \kApproxLCS, we build the data structure of Andoni and Razenshteyn~\cite{DBLP:conf/stoc/AndoniR15} for $(\sqrt{(1+\eps)(1-\alpha)}, \sqrt{(1+\alpha)k})$-\NN over~$Q'$. We make a query for each string in $P'$. If, queried for $\sk_{\alpha}(S_1)\in P'$, where $S_1$ is a length-$\ell$ substring of $X$, the data structure outputs $\sk_{\alpha}(S_2)\in Q'$, where $S_2$ is a length-$\ell$ substring of $Y$, then we compute $\norm{\sk_{\alpha}(S_1)- \sk_{\alpha}(S_2)}^2$. If it is at most $(1+\eps)k$, we output YES and the witness pair $(S_1,S_2)$ of substrings. As the length of vectors in $P'$, $Q'$ is $d = \Oh(\log^{1+o(1)} n)$, we obtain the desired complexity. 

To show that the algorithm is correct, suppose that there are length-$\ell$ substrings $S_1$ and $S_2$ of $X$ and $Y$, respectively, with $\HD(S_1, S_2) \le k$. By Corollary~\ref{cor:dim_reduction}, $\norm{\sk_\alpha(S_1),\sk_\alpha(S_2)}\le \sqrt{(1+\alpha)k}$ holds with probability at least $1-1/n$. Then, when querying for $\sk_\alpha(S_1)$, with constant probability the data structure will output a string $\sk_\alpha(S'_2)$ such that $\norm{\sk_\alpha(S_1) - \sk_\alpha(S'_2)}^2 \le (1+\eps)(1-\alpha^2) k \le (1+\eps) k$. Then, our algorithm will return YES. 

On the other hand, if we output YES with a witness pair $(S_1,S_2)$, then $\norm{\sk_\alpha(S_1) - \sk_\alpha(S_2)}^2\le (1+\eps)k$ implies $\HD(S_1,S_2)\le (1+\eps)k$ with high probability by Corollary~\ref{cor:dim_reduction}.
\end{proof}

While this solution is very fast, it uses quite a lot of space. Furthermore, the data structure of~\cite{DBLP:conf/stoc/AndoniR15} that we use as a black box applies highly non-trivial techniques. To overcome these two disadvantages, we will show a different solution based on a careful implementation of ideas first introduced in~\cite{substringNN} that showed a data structure for approximate text indexing with mismatches. In~\cite{DBLP:journals/algorithmica/KociumakaRS19}, the authors developed these ideas further to show an algorithm that solves the \kApproxLCS problem in $\Oh(n^{1+1/(1+\eps)} )$ space and $\Oh (n^{1+1/(1+\eps)} \log^2 n)$ time for $\eps\in(0,2)$ with constant error probability. In this work, we significantly improve and simplify the approach to show the following result:  

\begin{theorem}\label{th:LCS'k-decision}
Assume an alphabet of arbitrary size $\sigma = n^{\Oh(1)}$. The decision variant of \kApproxLCS can be solved in $\Oh(n^{1+1/(1+\eps)} \log^2 n)$ time and $\Oh(n)$ space. The answer is correct with constant probability. 
\end{theorem}

Let us defer the proof of the theorem until Section~\ref{sec:decision} and start by explaining how we use Lemma~\ref{lm:opt_NN} and Theorem~\ref{th:LCS'k-decision} and the \twentyquestions game to show Theorem~\ref{th:klcs_upper}.

\begin{proof}[Proof of Theorem~\ref{th:klcs_upper}]
We will rely on the modified version of the \twentyquestions game that we
  described in Section~\ref{sec:20questions}. In our case, $A = \lcsk(X,Y)$ and
  $B = \lcske(X,Y)$. For Carole, we use either the algorithm of
  Lemma~\ref{lm:opt_NN}, or the algorithm of Theorem~\ref{th:LCS'k-decision}, 
  with an additional procedure verifying the witness pair $(S_1,S_2)$ character by character to check that it indeed satisfies $\HD(S_1,S_2)\le (1+\eps)k$.
  We output the longest pair of (honest) witness
  substrings found across all iterations. We will return a correct answer
  assuming that the fraction of errors is $\rho <\frac13$. Recall that the algorithm solves the decision variant of the \kApproxLCS problem incorrectly with probability not exceeding a constant $\delta$, and we can ensure $\delta < \frac13$ by repeating it a constant number of times. It means that Carole can answer an individual question erroneously with probability less than~$\frac13$. Therefore, for a sufficiently large constant in the number of queries $Q = \Theta(\log n)$, the fraction of erroneous answers is $\rho < \frac13$ with high probability by Chernoff--Hoeffding bounds. The claim of the theorem follows immediately from Lemma~\ref{lm:opt_NN} and Theorem~\ref{th:LCS'k-decision}.
\end{proof}

\subsection{Proof of Theorem~\ref{th:LCS'k-decision}}\label{sec:decision}
We first give an algorithm for the decision version of the \kApproxLCS problem that uses $\Oh(n \log n)$ space and $\Oh(n^{1+1/(1+\eps)} \log n + \sigma n \log^2 n)$ time, and then we improve the space and time complexity. 

We assume to have fixed a Karp--Rabin fingerprinting function $\varphi$ for a prime $q = \Omega(\max\{n^5, \sigma\})$ and an integer $r \in \mathbb{Z}_q$. With error probability inverse polynomial in $n$, we can find such $q$ in $\Oh(\log^{\Oh(1)}n)$ time;
see~\cite{DBLP:journals/moc/TaoCH12,agrawal2004primes}. 

Let~$\Projections$ be the set of all projections of strings of length $\ell$ onto a single position, i.e., the value $\pi_i(S)$ of the $i$-th projection on a string $S$ of length $\ell$ is simply its $i$-th character $S[i]$. More generally, for a length-$\ell$ string $S$ and a function $h=(\pi_{a_1},\ldots,\pi_{a_m}) \in \Projections^m$, we define $h(S)$ as $S[a_{1}] S[a_{2}] \cdots S[a_{m}]$.

Let $p_1 = 1 - k / \ell$ and $p_2 = 1 - (1+\eps) k / \ell$. We assume that $(1+\eps) k< \ell$ in order to guarantee $p_1>p_2>0$; the problem is trivial if $(1+\eps)k \ge \ell$. 
Further, let $m = \ceil { \log_{p_2}{\tfrac{1}{n}} }$.

We choose a set $\Hashes$ of $L = \Theta(n^{1/(1+\eps)})$ hash functions in $\Projections^m$ uniformly at random. Let $\Collisions^{\Hashes}_{\ell}$ be the mutliset of all collisions of length-$\ell$ substrings of $X$ and $Y$ under the functions from $\Hashes$, i.e. $\Collisions^{\Hashes}_{\ell} = \{(X[i,i+\ell-1], Y[j,j+\ell-1],h) : \varphi(h(X[i,i+\ell-1])) = \varphi(h(Y[j,j+\ell-1])), 1 \le i \le |X|-\ell, 1 \le j \le |Y|-\ell\}$. 

We will perform two tests. The first test chooses an arbitrary subset $\Collisions'\subseteq \Collisions^{\Hashes}_{\ell}$ of size $|\Collisions'|=\min\{4nL, |\Collisions^{\Hashes}_{\ell}|\}$ and, for each collision $(S_1, S_2, h)\in \Collisions'$, computes $\norm{\sk_\eps(S_1) - \sk_\eps (S_2)}^2$. If this value is at most $(1+\eps) k$, then the algorithm returns YES and the pair $(S_1, S_2)$ as a witness. The second test chooses a collision $(S_1, S_2, h) \in \Collisions^{\Hashes}_{\ell}$ uniformly at random and computes the Hamming distance between $S_1$ and $S_2$ character by character in $\Oh(\ell) = \Oh(n)$ time. If the Hamming distance is at most $(1+\eps)k$, the algorithm returns YES and the witness pair $(S_1, S_2)$. Otherwise, the algorithm returns NO. See Algorithm~\ref{alg:LSH}.

\begin{algorithm}[ht]
\caption{\kApproxLCS (decision variant)}
\begin{algorithmic}[1]
\State Choose a set $\Hashes$ of $L$ functions from $\Projections^m$ uniformly at random
\State $\Collisions^{\Hashes}_{\ell}\!=\!\{(S_1,S_2, h) : S_1, S_2\text{---length-$\ell$ substrings of $X,Y$ resp. and }\varphi(h(S_1)) = \varphi(h(S_2))\}$\label{ln:hashes}

\State Choose an arbitrary subset $\Collisions' \subseteq \Collisions^{\Hashes}_{\ell}$ of size $\min\{4nL,|\Collisions^{\Hashes}_{\ell}|\}$
\State Compute $\sk_\eps(\cdot)$ sketches for all length-$\ell$ substrings of $X, Y$  \label{ln:sketches}
\For {$(S_1, S_2, h) \in \Collisions'$} \label{ln:test1}
	\If {$\norm{\sk_\eps (S_1) - \sk_\eps (S_2)}^2 \le (1+\eps) k$} \label{ln:Hamming_dist}\Return $(\text{YES},(S_1,S_2))$
	\EndIf
\EndFor

\State Draw a collision $(S_1, S_2, h) \in \Collisions^{\Hashes}_{\ell}$ uniformly at random \label{ln:test2}
\If {$\HD (S_1, S_2) \le (1+\eps) k$} \label{ln:test3}\Return $(\text{YES},(S_1,S_2))$
\EndIf\\
\Return NO
\end{algorithmic}
\label{alg:LSH}
\end{algorithm}

We must explain how we compute $\Collisions^{\Hashes}_{\ell}$ and choose the collisions that we test. We consider each hash function $h \in \Hashes$ in turn. Let $h = (\pi_{a_1},\ldots,\pi_{a_m})$. Recall that for a string $S$ of length~$\ell$ we define $h(S)$ as $S[a_{1}] S[a_{2}] \cdots S[a_{m}]$. Consequently, $\varphi(h(S)) =  (\sum_{i=1}^m r^{i-1} S[a_{i}]) \bmod q$. We create a vector $U$ of length $\ell$ where each entry is initialised with $0$. For each $i$, we add $r^{i-1} \bmod q$ to the $a_{i}$-th entry of $U$. Finally, we run the FFT algorithm~\cite{FischerPaterson} for $U$ and $X, Y$ in the field $\mathbb{Z}_q$, and sort the resulting values. We obtain a list of sorted values that we can use to generate the collisions. Namely, consider some fixed value $z$. Assume that there are $x$ substrings of $X$ and $y$ substrings of $Y$ of length $\ell$ such that the fingerprint of their projection is equal to $z$. The value $z$ then gives $xy$ collisions, and we can generate each one of them in constant time. This explains how to choose the subset $\Collisions'$ in $\Oh(nL \log n)$ time.

To draw a collision from $\Collisions^{\Hashes}_{\ell}$ uniformly at random, we could simply compute the total number of collisions across all functions $h \in \Hashes$, draw a number in $[1,  |\Collisions^{\Hashes}_{\ell}|]$, and generate the corresponding collision. However, this would require to generate the collisions twice. Instead, we use the weighted reservoir sampling algorithm~\cite{reservoir}. We divide all collisions into subsets according to the values of fingerprints. We assume that the weighted reservoir sampling algorithm receives the fingerprint values one-by-one, as well as the number of corresponding collisions. At all times, the algorithm maintains a ``reservoir'' containing one fingerprint value and a random collision corresponding to this value. When a new value $z$ with $xy$ collisions arrives, the algorithm replaces the value in the reservoir with $z$ and a random collision with some probability. Note that to select a random collision it suffices to choose a pair from $[1,x] \times [1,y]$ uniformly at random. It is guaranteed that if for a value $z$ we have $xy$ collisions, the algorithm will select $z$ with probability $xy/|\Collisions^{\Hashes}_{\ell}|$. Consequently, after processing all values, the reservoir will contain a collision chosen from $\Collisions^{\Hashes}_{\ell}$ uniformly at random.

\begin{lemma}\label{lm:complexity}
Algorithm~\ref{alg:LSH} uses $\Oh(n^{1+1/(1+\eps)} \log n + \sigma n \log^2 n)$ time and $\Oh(n \log n)$ space. 
\end{lemma}
\begin{proof}
Computing the sketches (Line~\ref{ln:sketches}) takes $\Oh(\sigma n \log^2 n)$ time and $\Oh(n \log n)$ space. Computing the collisions and choosing the collisions to test takes $\Oh(n^{1+1/(1+\eps)} \log n)$ time and $\Oh(n)$ space in total. Testing $\min \{4nL, |\Collisions^{\Hashes}_{\ell}|\}$ collisions (Line~\ref{ln:test1}) takes $\Oh(n^{1+1/(1+\eps)} \log n)$ time and constant space. Computing the Hamming distance for a random collision (Line~\ref{ln:test3}) takes $\Oh(\ell) = \Oh(n)$ time and constant space.
\end{proof}

\begin{lemma}\label{lm:hash_function_exists}
Let $S_1$ and $S_2$ be two length-$\ell$ substrings of $X$ and $Y$, respectively, with $\HD(S_1, S_2) \le k$. 
If $L=\Theta(n^{1/(1+\eps)})$ is large enough, then, with probability at least $3/4$, there exists a function $h\in \Hashes$ such that $h(S_1)=h(S_2)$.
\end{lemma}
\begin{proof}
Consider a function $h=(\pi_{a_1},\ldots,\pi_{a_m})$ drawn from $\Projections^m$ uniformly at random. The probability of $h(S_1)=h(S_2)$ is at least $p_1^m$.
Due to $p_1 \le 1$, we have \[p_1^m = p_1^{\ceil{\log_{p_2}\frac1n}} \ge p_1^{1+\log_{p_2}\frac1n}=p_1\cdot n^{-\frac{\log p_1}{\log p_2}}.\]
Moreover, $p_1 = 1-\frac{k}{\ell}$ and $(1+\eps)k < \ell$ yield $p_1 > 1-\frac{1}{1+\eps}=\frac{\eps}{1+\eps}$,
whereas Bernoulli's inequality implies $p_2 = 1-(1+\eps)\frac{k}{\ell} \le (1-\frac{k}{\ell})^{1+\eps}=p_1^{1+\eps}$,
i.e., $\log p_2  \le (1+\eps)\log p_1$.
Therefore, \[p_1^m \ge p_1\cdot n^{-\frac{\log p_1}{\log p_2}}\ge\tfrac{\eps}{1+\eps}\cdot n^{-\frac{1}{1+\eps}}.\]
Hence, we can choose the constant in $L=|\Hashes|$ so that the claim of the lemma holds.
\end{proof}

\begin{lemma}\label{lm:bad_collisions}
If $|\Collisions^{\Hashes}_{\ell}| > 4 nL$ and $(S_1,S_2, h)$ is a uniformly random element of $\Collisions^{\Hashes}_{\ell}$, then $\Prob[\HD(S_1,S_2) \ge (1+\eps) k] \le \frac12$.
\end{lemma}
\begin{proof}
Consider length-$\ell$ substrings $S_1, S_2$ of $X, Y$, respectively, such that $\HD(S_1, S_2) \ge (1+\eps)k$, and a hash function $h$. Let us bound the probability of $(S_1,S_2, h) \in \Collisions^{\Hashes}_{\ell}$. There two possible cases: either $h(S_1) \neq h(S_2)$ but $\varphi(h(S_1)) = \varphi(h(S_2))$, or $h(S_1) = h(S_2)$. The probability of the first event is bounded by the collision probability of Karp--Rabin fingerprints, which is at most $1/n$. Let us now bound the probability of the second event. Since $\HD (S_1,S_2)\ge (1+\eps)k$, we have $\Prob [h (S_1) = h (S_2)] \le p_2^{m} \le 1/n$,  
where the last inequality follows from the definition of $m$. Therefore, the probability that for some function $h \in \Hashes$ we have $\varphi(h(S_1)) = \varphi(h(S_2))$ is at most $2/n$. 

In total, we have $n^2 |\Hashes|$ possible triples $(S_1, S_2 ,h)$ so by linearity of expectation, we conclude that the expected number of such triples is at most $\frac{2}{n} n^2 L =2n L$. Therefore the probability to hit a triple $(S_1, S_2, h)$ such that $\HD(S_1, S_2) \ge (1+\eps)k$ when drawing from $\Collisions^{\Hashes}_{\ell}$ uniformly at random is at most $2nL / |\Collisions^{\Hashes}_{\ell}| \le 2nL / 4nL = 1/2$.
\end{proof}

Below, we combine the previous results to prove that, with constant probability, Algorithm~\ref{alg:LSH} correctly solves
the decision variant of the \kApproxLCS problem.
Note that we can reduce the error probability to an arbitrarily small constant $\delta>0$: it suffices to repeat the algorithm a constant number of times. 

\begin{corollary}
With non-zero constant probability, Algorithm~\ref{alg:LSH} solves the decision variant of \kApproxLCS correctly.
\end{corollary}
\begin{proof}
Suppose first that $\ell \le \lcsk(X,Y)$, which means that there are two length-$\ell$ substrings $S_1, S_2$ of $X, Y$ such that $\HD(S_1, S_2) \le k$. By Lemma~\ref{lm:hash_function_exists}, with probability at least $3/4$, there exists a function $h\in \Hashes$ such that $h(S_1)=h(S_2)$.
In other words, $(S_1, S_2, h) \in \Collisions^{\Hashes}_{\ell}$ with probability at least $\frac34$. If $|\Collisions^{\Hashes}_{\ell}| < 4nL$, we will find this triple and it will pass the test with probability at least $1-n^{-6}$.
If $|\Collisions^{\Hashes}_{\ell}| \ge 4nL$, then by Lemma~\ref{lm:bad_collisions} the Hamming distance between $S_1, S_2$, where $(S_1, S_2, h)$ was drawn from $\Collisions^{\Hashes}_{\ell}$ uniformly at random, is at most $(1+\eps)k$ with probability $\ge 1/2$, and therefore this pair will pass the test with probability $\ge 1/2$. It follows that in this case the algorithm outputs YES with constant probability.

Suppose now that $\ell > \lcske(X,Y)$. In this case, the Hamming distance between any pair of length-$\ell$ substrings of $X$ and $Y$ is at least $(1+\eps)k$, so none of them will ever pass the second test and none of them will pass the first test with constant probability.
\end{proof}

We now improve the space of the algorithm to linear. Note that the only reason why we needed $\Oh(n \log n)$ space is that we precompute and store the sketches for the Hamming distance. Below we explain how to overcome this technicality.

First, we do not precompute the sketches. Second, we process the collisions in $\Collisions'$ in batches of size $n$. Consider one of the batches, $\mathcal{B}$. For each collision $(S_1,S_2, h) \in \mathcal{B}$ we must compute $\norm{\sk_\eps(S_1) - \sk_\eps(S_2)}^2$. 
We initialize a counter for every collision, setting it to zero initially. The number of rounds in the algorithm will be equal to the length of the sketches, and, in round $i$, the counter for a collision $(S_1, S_2, h) \in \mathcal{B}$ will contain the squared $L_2$ distance between the length-$i$ prefixes of $\sk_\eps(S_1)$ and $\sk_\eps(S_2)$. In more detail, let $\mathcal{S}$ be the set of all substrings of $X, Y$ that participate in the collisions in $\mathcal{B}$. Recall that all these substrings have length $\ell$. At round $i$, we compute the $i$-th coordinate of the sketches of the substrings in $\mathcal{S}$. By definition, the $i$-th coordinate is the dot product of the $i$-th row of $c \cdot M$, where $c$ and $M$ are as in Corollary~\ref{cor:dim_reduction}, and a substring encoded using $\mu$. Hence, we can compute the coordinate using the FFT algorithm~\cite{FischerPaterson} in $\Oh(\sigma n \log n)$ time and $\Oh(n)$ space. When we have the coordinate $i$ computed, we update the counters for the collisions and repeat.

At any time, the algorithm uses $\Oh(n)$ space.
Compared to the time consumption proven in Lemma~\ref{lm:complexity}, the algorithm spends an additional $\Oh(\sigma n^{1+1/(1+\eps)} \log^2 n)$ time for computing the coordinates of the sketches.
Therefore, in total the algorithm uses $\Oh(\sigma n^{1+1/(1+\eps)} \log^2 n) = \Oh(n^{1+1/(1+\eps)} \log^2 n)$ time and $\Oh(n)$ space. 
For constant-size alphabets, this completes the proof of Theorem~\ref{th:LCS'k-decision}. For alphabets of arbitrary size, we replace the sketches from Section~\ref{sec:prelim} with the sketches defined in~\cite{DBLP:journals/algorithmica/KociumakaRS19} to achieve the desired complexity.
We note that we could use the sketches~\cite{DBLP:journals/algorithmica/KociumakaRS19} for small-size alphabets as well, but their lengths hide a large constant. 



