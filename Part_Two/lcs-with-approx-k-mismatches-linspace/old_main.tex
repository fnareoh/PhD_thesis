\documentclass[a4paper,UKenglish,cleveref,autoref]{lipics-v2019}
\usepackage{xspace,enumerate}
\usepackage[noend]{algpseudocode}
\usepackage{algorithm,algorithmicx}
\usepackage{amsfonts,amsmath}
\usepackage{microtype}
\usepackage{xspace}
\usepackage{todonotes}
\usepackage[draft]{fixme}
\usetikzlibrary{decorations.pathreplacing}
\usepackage{multirow}
\usepackage{diagbox}
\usepackage{array}

\newcommand{\Oh}{\mathcal{O}}
\newcommand{\eps}{\varepsilon}
\newcommand{\lcpe}{\mathrm{LCP}_{(1+\eps)k}}
\newcommand{\lcp}{\mathrm{LCP}_{\tilde{k}}}
\newcommand{\lcpk}{\mathrm{LCP}_{k}}
\newcommand{\lcsk}{\mathrm{LCS}_{k}}
\newcommand{\lcske}{\mathrm{LCS}_{(1+\eps)k}}
\newcommand{\lcsak}{\mathrm{LCS}_{\tilde{k}}}
\newcommand{\sk}{\mathrm{sk}}
\newcommand{\Prob}{\mathrm{Pr}}
\newcommand{\LCSp}{\textsf{LCS}\xspace}
\newcommand{\kLCS}{\textsf{LCS with $k$ Mismatches}\xspace}
\newcommand{\kApproxLCS}{\textsf{LCS with Approximately $k$ Mismatches}\xspace}
\newcommand{\Bichromatic}{\textsf{$(1+\gamma)$-approximate Bichromatic Closest Pair}\xspace}
\newcommand{\NN}{\textsf{Approximate Near Neighbour}\xspace}
\newcommand{\twentyquestions}{\textsf{Twenty Questions}\xspace}
\newcommand{\Carole}{\mathit{Carole}}
\newcommand{\pop}{\mathit{pop}}
\newcommand{\push}{\mathit{push}}
\newcommand{\ttop}{\mathit{top}}
\newcommand{\mmid}{\mathit{mid}}

\newcommand{\Hashes}{\mathcal{H}}
\newcommand{\Collisions}{C}
\newcommand{\Bad}{B}
\newcommand{\Projections}{\Pi}
\newcommand{\Pos}{\mathsf{P}}
\newcommand{\HD}{d_H}
\newcommand{\LCP}{\mathrm{LCP}}

\newtheorem{fact}[theorem]{Fact}{\bfseries}{\itshape}
\newtheorem{observation}[theorem]{Observation}{\bfseries}{\itshape}
\newtheorem{problem}[theorem]{Problem}{\bfseries}{\itshape}
\newtheorem{hypothesis}[theorem]{Hypothesis}{\bfseries}{\itshape}

\newcommand{\norm}[1]{\ensuremath{\lVert#1\rVert}}

\newcommand{\ceil}[1]{\ensuremath{\lceil#1\rceil}}

\newcommand\restr[2]{{
  \left.\kern-\nulldelimiterspace
  #1 
  \vphantom{\big|} 
  \right|_{#2} 
}}

\bibliographystyle{plainurl}% the recommended bibstyle

\nolinenumbers

% Author macros::begin %%%%%%%%%%%%%%%%%%%%%%%%%%%%%%%%%%%%%%%%%%%%%%%%
\author{Garance Gourdel}{ENS Paris Saclay, France}{garance.gourdel@ens-paris-saclay.fr}{}{}
\author{Tomasz Kociumaka}{Bar-Ilan University, Ramat Gan, Israel}{kociumaka@mimuw.edu.pl}{https://orcid.org/0000-0002-2477-1702}{Supported by ISF grants no. 1278/16 and 1926/19, a BSF grant no. 2018364, and an ERC grant MPM (no.\ 683064) under the EU's Horizon 2020 Research and Innovation~Programme.}
\author{Jakub Radoszewski}{Institute of Informatics, University of Warsaw, Warsaw, Poland \and Samsung R\&D Institute, Warsaw, Poland}{jrad@mimuw.edu.pl}{https://orcid.org/0000-0002-0067-6401}{Supported by the Polish National Science Center, grant number 2018/31/D/ST6/\allowbreak03991.}
\author{Tatiana Starikovskaya}{DIENS, \'{E}cole normale sup\'{e}rieure, PSL Research University,  France}{tat.starikovskaya@gmail.com}{}{}

\title{\texorpdfstring{Approximating longest common substring with $k$~mismatches: Theory and practice}{Approximating longest common substring with k mismatches: Theory and practice}}
\titlerunning{Approximating longest common substring with $k$ mismatches}
\authorrunning{G. Gourdel, T. Kociumaka, J. Radoszewski, and T. Starikovskaya}
\Copyright{Garance Gourdel, Tomasz Kociumaka, Jakub Radoszewski, and Tatiana Starikovskaya}
\ccsdesc[500]{Theory of computation~Pattern matching}
\keywords{approximation algorithms, string similarity, LSH, conditional lower bounds}
\acknowledgements{}
% Author macros::end %%%%%%%%%%%%%%%%%%%%%%%%%%%%%%%%%%%%%%%%%%%%%%%%%

%Editor-only macros:: begin (do not touch as author)%%%%%%%%%%%%%%%%%%%%%%%%%%%%%%%%%%
\EventEditors{Inge Li G{\o}rtz and Oren Weimann}
\EventNoEds{2}
\EventLongTitle{31th Annual Symposium on Combinatorial Pattern Matching (CPM 2020)}
\EventShortTitle{CPM 2020}
\EventAcronym{CPM}
\EventYear{2020}
\EventDate{June 17--19, 2020}
\EventLocation{Copenhagen, Denmark}
\EventLogo{}
\SeriesVolume{161}
\ArticleNo{18}
% Editor-only macros::end %%%%%%%%%%%%%%%%%%%%%%%%%%%%%%%%%%%%%%%%%%%%%%%

\begin{document}
\maketitle

\begin{abstract}
In the problem of the longest common substring with $k$ mismatches we are given two strings $X, Y$ and must find the maximal length $\ell$ such that there is a length-$\ell$ substring of $X$ and a length-$\ell$ substring of $Y$ that differ in at most $k$ positions. The length $\ell$ can be used as a robust measure of similarity between $X, Y$. In this work, we develop new approximation algorithms for computing $\ell$ that are significantly more efficient that previously known solutions from the theoretical point of view. Our approach is simple and practical, which we confirm via an experimental evaluation, and is probably close to optimal as we demonstrate via a conditional lower bound.
\end{abstract}

\section{Introduction}\label{sec:intro}
\section{Introduction}

The notion of repetition is a central concept in combinatorics on words and algorithms on strings. In this context,
a word or a string is simply a sequence of characters from some finite alphabet $\Sigma$. In the most basic version,
a repetition consists of two (or more) consecutive occurrences of the same fragment. Repetitions are interesting not
only from a purely theoretical point of view, but are also very relevant in bioinformatics~\cite{Kolpakov2003}.
A repetition could be a square, defined as two consecutive occurrences of the same fragment, a higher power (for example, a cube), or a run, which is a length-wise maximal periodic substring.
For example, both \texttt{anan} and \texttt{nana} are squares with two occurrences each in \texttt{banananas}, and they belong to the same run \texttt{ananana}.
In this paper, we start by focusing on squares, then generalize our results for runs.

The study of squares in strings goes back
to the work of Thue published in 1906~\cite{thue1906}, who considered the question of constructing an infinite word
with no squares. It is easy to see that any sufficiently long binary word must contain a square, and Thue proved that
there exists an infinite ternary word with no squares. His result has been rediscovered multiple times, and in 1979
Bean, Ehrenfeucht and McNulty~\cite{BeanEM1979} started a systematic study of the so-called avoidable repetitions,
see for example the survey by Currie~\cite{Currie05}.

\paragraph{Combinatorics on words.} The basic tool in the area of combinatorics on words is the so-called
periodicity lemma. A period of a string $T[1..n]$ is an integer $d$ such that $T[i]=T[i+d]$ for every $i\in [1,n-d]$,
and the periodicity lemma states that if $p$ and $q$ are both such periods and $p+q\leq n+\gcd(p,q)$ then 
$\gcd(p,q)$ is also a period~\cite{Fine1965}. 
This was generalised in a myriad of ways, for strings~\cite{Castelli1999,Justin2000,Tijdeman2003},
partial words (words with don't cares)~\cite{Berstel1999,Blanchet-Sadri2008,Blanchet-Sadri2002,Shur2004,Shur2001,Idiatulina2014,Kociumaka2022},
Abelian periods\cite{Constantinescu2006,Blanchet-Sadri2013}, parametrized periods~\cite{Apostolico2008},
order-preserving periods~\cite{Matsuoka2016,Gourdel2020}, approximate periods~\cite{Amir2010,Amir2012,Amir2015}.
Now, a square can be defined as a fragment of length twice its period. 
The string $\texttt{a}^{n}$ contains $\Omega(n^{2})$ such fragments,
thus from the combinatorial point of view it is natural to count only distinct squares.
Fraenkel and Simpson~\cite{Fraenkel1998} showed an upper bound of $2n$ and a lower bound of $n-\Theta(\sqrt{n})$ for the maximum number of distinct squares in a length-$n$ string.
After a sequence of improvements~\cite{Ilie2007,Deza2015,Thierry2020}, the upper bound was very recently improved to $n$~\cite{Brlek2022}.
The last result was already generalised to higher powers~\cite{Li2022}.
Another way to avoid the trivial examples such as $\texttt{a}^{n}$ is to count only maximal periodic fragments,
that is, fragments with period at most half of their length and that cannot be extended to the left or to the right without
breaking the period. Such fragments are usually called runs. Kolpakov and Kucherov~\cite{Kolpakov1999} showed
an upper bound of $\Oh(n)$ on their number, and this started a long line of work on determining the exact
constant~\cite{Rytter2006,Puglisi2008,Crochemore2008,Giraud2008,Giraud2009,Crochemore2011}, culminating
in the paper of Bannai et al.~\cite{Bannai2017} showing an upper bound of $n$, and followed by even better upper bounds
for binary strings~\cite{Fischer2015,Holub2017}. This was complemented by a sequence of 
lower bounds~\cite{Franek2008,Matsubara2008,Matsubara2009,Simpson2010}.

\paragraph{Algorithms on strings.}
In this paper, we are interested in the algorithmic aspects of detecting repetitions in strings. The most basic question
in this direction is checking if a given length-$n$ string contains at least one square,
while the most general version asks for computing all the runs.
Testing square-freeness was
first considered by Main and Lorentz~\cite{Main1984}, who designed an $\Oh(n\log n)$ time algorithm based on
a divide-and-conquer approach and a linear-time procedure for finding all new squares obtained when concatenating
two strings. In fact, their algorithm can be used to find (a compact representation of) all squares in a given string
within the same time complexity. They also proved that any algorithm based on comparisons of characters needs
$\Omega(n\log n)$ such operations to test square-freeness in the worst case. Here, comparisons of characters means
checking if characters at two positions of the input string are equal. However, to obtain the lower bound they
had to consider instances consisting of even up to $n$ distinct characters, that is, over alphabet of size $n$.
This is somewhat unsatisfactory, and motivates the following open question that was explicitly asked by Main and Lorentz~\cite{Main1984}:

\begin{question}
Is there a faster algorithm to determine if a string is square-free if we restrict the size of the alphabet?
\end{question}

Crochemore~\cite{Crochemore1981} gave another $\Oh(n\log n)$ time algorithm for finding all repetitions,
and also showed that for constant-size alphabets testing square-freeness can be done in  $\Oh(n)$ time~\cite{Crochemore1986}.
In fact, the latter algorithm works in $\Oh(n\log \sigma)$ time for alphabets of size $\sigma$ with a linear order on the characters.
That is, it needs to test if the character at some position is smaller than the character at another position.
In the remaining part of the paper, we will refer to this model as general ordered alphabet, while the model
in which we can only test equality of characters will be called general (unordered) alphabet.
Later, Kosaraju~\cite{Kosaraju1994} showed that in fact, assuming constant-size alphabet, $\Oh(n)$ time is enough
to find the shortest square starting at each position of the input string.
Apostolico and Preparata~\cite{Apostolico1983} provide another $\Oh(n\log n)$ time algorithm assuming a general ordered alphabet,
based more on data structure considerations than combinatorial properties of words.
Finally, a number of alternative $\Oh(n\log n)$ and $\Oh(n\log \sigma)$ time algorithms (respectively, for general unordered
and general ordered alphabets) can be obtained from the work on online~\cite{Hong2008,Kosolobov2014,Kosolobov2015a}
and parallel~\cite{Apostolico1996} square detection (interestingly, this cannot be done efficiently in the related
streaming model~\cite{Merkurev2019,Merkurev2022}).

Faster algorithms for testing square-freeness of strings over general ordered alphabets  were obtained as a byproduct of
the more general results on finding all runs. Kolpakov and Kucherov~\cite{Kolpakov1999} not only proved that any length-$n$
string contains only $\Oh(n)$ runs, but also showed how to find them in the same time assuming
linearly-sortable alphabet. Every square is contained in a run, and every run contains at least one square, thus this
in particular implies a linear-time algorithm for testing square-freeness over such alphabets. For general ordered alphabets,
Kosolobov~\cite{Kosolobov2015} showed that the decision tree complexity of this problem is only $\Oh(n)$, and later complemented this with an efficient $\Oh(n(\log n)^{2/3})$ time algorithm~\cite{Kosolobov2016}
(still using only $\Oh(n)$ comparisons). The time complexity was then improved to $\Oh(n\log\log n)$ by providing a general
mechanism for answering longest common extension (LCE) queries for general ordered alphabets~\cite{Gawrychowski2016},
and next to $\Oh(n\alpha(n))$ by observing that the LCE queries have additional structure~\cite{CrochemoreIKKPR16}.
Finally, Ellert and Fischer provided an elegant $\Oh(n)$ time algorithm, thus fully resolving the complexity of square detection
for general ordered alphabets. However, for general (unordered) alphabets the question of Main and Lorentz remains
unresolved, with the best upper bound being $\Oh(n\log n)$, and only known to be asymptotically tight for alphabets of
size $\Theta(n)$.

\paragraph{General alphabets.} While in many applications one can without losing generality assume some
ordering on the characters of the alphabet, no such ordering is necessary for defining what a square is. Thus, it is natural
from the mathematical point of view to seek algorithms that do not require such an ordering to efficiently test square-freeness.
Similar considerations have lead to multiple beautiful results concerning the pattern matching problem, such as constant-space algorithms~\cite{Galil1983,Breslauer1992},
or the works on the exact number of required equality comparisons ~\cite{Cole1995,Cole1997} 
 More recent examples include the work of Duval, Lecroq, and Lefebvre~\cite{Duval2014}
on computing the unbordered conjugate/rotation, and Kosolobov~\cite{Kosolobov2016a} on finding the leftmost critical point.

\paragraph{Main results.}

We consider the complexity of checking if a given string $T[1..n]$ containing $\sigma$ distinct characters is square-free. 
The input string can be only accessed by issuing comparisons $T[i]\stackrel{?}{=} T[j]$, and the value of $\sigma$ is not
assumed to be known. We start by analysing the decision tree complexity of the problem. That is, we only
consider the required and necessary number of comparisons, without worrying about an efficient implementation.
We show that, even if the value of $\sigma$ is assumed to be known, $\Omega(n\log \sigma)$ comparisons are required. 

\begin{restatable}{theorem}{lowerbound}
\label{thm:lowerbound}
For any integers $n$ and $\sigma$ with $8 \leq \sigma \leq n$, there is no deterministic algorithm that performs at most $n \ln \sigma - 3.6n = \Oh(n \ln \sigma)$ comparisons in the worst case, and determines whether a length-$n$ string with at most $\sigma$ distinct symbols from a general unordered alphabet is square-free.
\end{restatable}

Next, we show that $\Oh(n\log \sigma)$ comparisons are sufficient. We stress that the value of $\sigma$ is not assumed to
be known. In fact, as a warm-up for the above theorem, we first prove that finding a sublinear multiplicative approximation
of this value requires $\Omega(n\sigma)$ comparisons. This does not contradict the claimed upper bound, as we are only saying
that the number of comparisons used on a particular input string is at most $\Oh(n\log \sigma)$, but might actually be smaller.
Thus, it is not possible to extract any meaningful approximation of the value of $\sigma$ from the number of used comparisons.

\begin{restatable}{theorem}{upperbound}
\label{thm:upperbound}
Testing square-freenes of a length-$n$ string that contains $\sigma$ distinct symbols from a general unordered alphabet can be done with $\Oh(n \log \sigma)$ comparisons.
\end{restatable}

The proof of the above result is not efficient in the sense that it only restricts the overall number of comparisons, and not the time
to actually figure out which comparisons should be used.  A direct implementation results in a quadratic time algorithm. We first
show how to improve this to $\Oh(n\log \sigma+n\log^{*}n)$ time (while still keeping the asymptotically optimal $\Oh(n\log \sigma)$ number
of comparisons), and finally to $\Oh(n\log \sigma)$. In this part of the paper, we assume the Word RAM model with word of length $\Omega(\log n)$.
We stress that the input string is still assumed to consist of characters that can be only tested for equality, that is, one should
think that we are given oracle access to a functions that, given $i$ and $j$, checks whether $T[i]=T[j]$.

\begin{restatable}{theorem}{upperbound2}
\label{thm:upperbound2}
Testing square-freeness of a length-$n$ string that contains $\sigma$ distinct symbols from a general unordered alphabet can be implemented in $\Oh(n\log \sigma)$ comparisons
and time.
\end{restatable}

Finally, we also generalize this result to the computation of runs.

\begin{restatable}{theorem}{upperbound2runs}
\label{thm:upperbound:runs}
Computing all runs in a length-$n$ string that contains $\sigma$ distinct symbols from a general unordered alphabet can be implemented in $\Oh(n\log \sigma)$ comparisons
and time.
\end{restatable}

Altogether, our results fully resolve the open question of Main and Lorentz for the case of general unordered alphabets and deterministic algorithms.
We leave extending our lowerbound to randomised algorithms as an open question.

\paragraph{Overview of the methods.}

As mentioned before, Main and Lorentz~\cite{Main1984} designed an $\Oh(n\log n)$ time algorithm for testing square-freeness of
length-$n$ strings over general alphabets. The high-level idea of their algorithm goes as follows. They first designed a procedure for checking,
given two strings $x$ and $y$, if their concatenation contains a square that is not fully contained in $x$ nor $y$ in $\Oh(\absolute{x}+\absolute{y})$ time.
Then, a divide-and-conquer approach can be used to detect a square in the whole input string in $\Oh(n\log n)$ total time.
For general alphabets of unbounded size this cannot be improved, but Crochemore~\cite{Crochemore1986} showed that, for general
ordered alphabets of size $\sigma$, a faster $\Oh(n\log \sigma)$ time algorithm exists. The gist of his approach is to first obtain the so-called
$f$-factorisation of the input string (related to the well-known Lempel-Ziv factorisation), that in a certain sense ``discovers'' repetitive
fragments. Then, this factorisation can be used to apply the procedure of Main and Lorentz on appropriately selected fragments of the input
strings in such a way that the leftmost occurrence of every distinct square is detected, and the total length of the strings on which we apply the
procedure is only $\Oh(n)$. The factorisation can be found in $\Oh(n\log \sigma)$ time for general ordered alphabets of size $\sigma$ by,
roughly speaking, constructing some kind of suffix structure (suffix array, suffix tree or suffix automaton).

For general (unordered)
alphabets, computing the $f$-factorisation (or anything similar) seems problematic, and in fact we show (as a corollary of our lower
bound on approximating the alphabet size) that computing the $f$-factorisation or Lempel-Ziv-factorisation (LZ-factorisation) of a given length-$n$
string containing $\sigma$ distinct characters requires $\Omega(n\sigma)$ equality tests. Thus, we need another approach.
Additionally, the $\Oh(n)$ time algorithm of Ellert and Fischer~\cite{Ellert2021} hinges on the notion of Lyndon words, which is simply
not defined for strings over general alphabets. Thus, at first glance it might seem that $\Theta(n\sigma)$ is the right time complexity
for testing square-freeness over length-$n$ strings over general alphabets of size $\sigma$. However, due to the $\Omega(n\log n)$
lower bound of Main and Lorentz for testing square-freeness of length-$n$ string consisting of up to $n$ distinct characters,
one might hope for an $\Oh(n\log \sigma)$ time algorithm when there are only $\sigma$ distinct characters.

We begin our paper with a lower bound of $\Theta(n\log \sigma)$ for such strings. Intuitively, we show that testing square-freeness
has the direct sum property: $\frac{n}{\sigma}$ instances over length-$\sigma$ strings can be combined into a single instance over length-$n$ string.
As in the proof of Main and Lorentz, we use the adversarial method. While the underlying calculation is essentially the same,
we need to appropriately combine the smaller instances, which is done using the infinite square-free Prouhet-Thue-Morse sequence, and use significantly
more complex rules for resolving the subsequent equality tests. As a warm-up for the adversarial method, we prove that computing
any meaningful approximation of the number of distinct characters requires $\Omega(n\sigma)$ such tests, and that this implies
the same lower bound on computing the $f$-factorisation and the Lempel-Ziv factorisation (if the size of the alphabet is unknown in advance).

We then move to designing an approach that uses $\Oh(n\log \sigma)$ equality comparisons to test square-freeness.
As discussed earlier, one way of detecting squares uses the $f$-factorisation of the string, which is similar to its LZ factorisation.
However, as we prove in Corollary~\ref{cor:f-facto} and \ref{cor:LZ}, we cannot compute either of these factorisations over a general unordered alphabet in $o(n\sigma)$ comparisons.
Therefore, we will instead use a novel type of factorisation, $\Delta$-approximate LZ factorisation, that can be seen as an approximate
version of the LZ factorisation.
Intuitively, its goal is to ``capture'' all sufficiently long squares, while the original LZ factorisation (or $f$-factorisation)
captures all squares. Each phrase in a $\Delta$-approximate LZ factorisation consists of a head of length at most $\Delta$ and a tail
(possibly empty) that must occur at least once before, such that the whole phrase is at least as long as the classical LZ phrase starting
at the same position. Contrary to the classical LZ factorisation, this factorisation is not unique. 
The advantage of our modification is that there are fewer phrases (and there is more flexibility as to what they should be), and
hence one can hope to compute such factorisation more efficiently.

To design an efficient construction method for $\Delta$-approximate LZ factorisation, we first show how to compute a sparse
suffix tree while trying to use only a few symbol comparisons. This is then applied on a set of positions from a so-called
difference cover with some convenient synchronizing properties.
Then, a $\Delta$-approximate LZ factorisation allows us to detect squares of length $\geq 8\Delta$.

The first warm-up algorithm fixes $\Delta$ depending on $n$ and $\sigma$ (assuming that $\sigma$ is known), and uses
the approximate LZ factorisation to find all squares of length at least $8\Delta$. It then finds all the shorter squares by dividing
the string in blocks of length $8\Delta$, and applying the original algorithm by Main and Lorentz on each
block pair. Our choice of $\Delta$ leads to $\Oh(n (\lg \sigma + \lg \lg n))$ comparisons.

The improved algorithm does not need to know $\sigma$, and instead starts with a large $\Delta = \Omega(n)$, and then
progressively decreases $\Delta$ in at most $\Oh(\lg \lg n)$ phases, where later phases detect shorter squares.
As soon as we notice that there are many distinct characters in the alphabet, by carefully adjusting the parameters
we can afford switching to the approach of Main and Lorentz on sufficiently short fragments of the input string. 
Since we cannot afford $\Omega(n)$ comparisons per phase, we use a deactivation technique, where whenever we perform a large
number of comparisons in a phase, we will discard a large part of the string in all following phases.
More precisely, during a given phase, we avoid looking for squares in a fragment fully contained in a tail from an earlier phase.
This leads to optimal $\Oh(n \lg \sigma)$ comparisons.

The above approach uses an asymptotically optimal number of equality tests in the worst case, but does not result in an efficient
algorithm. The main bottleneck is constructing the sparse suffix trees. However, it is not hard to provide an efficient implementation
using the general mechanism for answering LCE queries for strings over general alphabets~\cite{Gawrychowski2016}.
Unfortunately, the best known approach for answering such queries incurs an additional $\Oh(n\log^{*}n)$ in the time complexity,
even if the size of the alphabet is constant. We overcome this technical hurdle by carefully deactivating fragments
of the text to account for the performed work.

Many of our techniques can easily be modified to compute all runs rather than detecting squares. We exploit that the approximate
factorisation reveals long substrings with an earlier occurrence. Hence we compute runs only for the first occurrence of such substrings,
while for later occurrences we simply copy the already computed runs.
By carefully arranging the order of the computation, we ensure that the total time for copying is bounded by the number
of runs, which is known to be $\Oh(n)$.
This way, we achieve $\Oh(n \lg \sigma)$ time and comparisons to compute all runs.

	
\section{Preliminaries}\label{sec:prelim}
 
\section{Preliminaries}
\label{pmgapped:sec:prelim}
A \emph{string} $s$ of length $|s| = n$ is a sequence $s[0]s[1]\dots s[n-1]$ of characters from an integer alphabet~$\Sigma$ that can be sorted in $O(m+g)$ time.
A \emph{substring} of a string $s$ is a pair $(i,j)$ where  $0 \le i \le j < |s|$ and is identified with the string $s[i \dots j]=s[i] \dots s[j]$. We also use the notation $s[i \dots j)$ and $s(i \dots j]$ which stand for the substring $s[i \dots j-1]$ and $s[i-1 \dots j]$ respectively.
We say that a substring $s[i \dots j]$ is fully contained in another substring $s[i' \dots j']$ if $i' \le i \le j \le j'$. 
We call a substring $s[0 \dots i]$ \emph{a prefix} of $s$ and use a simplified notation $s[\dots i]$, and a substring $s[i \dots n-1]$ \emph{a suffix} of $s$ denoted by $s[i \dots]$. 
We say that $x$ occurs in $s$ at position $i$ if $x = s[i \dots i+|x|)$, alternatively we say $i$ is an occurence of $x$ in $s$.
Additionaly, an occurence $i$ is fully included in a substring $f$ of $s$ if $s[i \dots i+|x|)$ is fully included in $f$.
%

An occurrence $k_1$ of $p_1$ together with an occurrence $k_2$ of $p_2$ form a \emph{consecutive occurrence (co-occurrence)} $(k_1,k_2)$ of the strings $p_1,p_2$ in a string $s$ if there are no occurrences of $p_1$ in $(k_1,k_2]$ and no occurrences of $p_2$ in $[k_1,k_2)$. The distance $k_2-k_1$ is sometimes referred to as a \emph{gap}.

An integer $\pi$ is a \emph{period} of a string $s$ of length $n$ if $s[i]=s[i+\pi]$ for all $i=0,\dots, n-1-\pi$. We say that $s$ is \emph{periodic} if its smallest period, referred to as \emph{the period} of $s$, is at most $|s|/2$. We also exploit the well-known corollary of the Fine and Wilf's periodicity lemma~\cite{Fine1965}:


\begin{corollary}\label{pmgapped:cor:arithmetic_progression}
Let $x, y$ be strings such that $|x|\leq 2|y|$. If there are at least three occurrences of $y$ in $x$, then all occurrences of $y$ in $x$ form an arithmetic progression with difference equal to the period of $y$. 
\end{corollary}


\begin{proposition}
\label{prop:suffix_tree}
One can preprocess a string $p$ of length $m$ in $O(m)$ time and space to maintain the following queries in constant time: Given a substring $(i,j)$, find the leftmost and the rightmost occurrences of $p[i...j]$ in $p$, as well as the total number of occurrences. Given two substrings $(i,j)$ and $(k,l)$, compute the longest common prefix and longest common suffix between $p[i...j]$ and $p[k...l]$.
\end{proposition}
\begin{proof}
We assume the reader to be familiar with suffix trees. We build the suffix tree of $p$ in $O(m)$ time and space. Belazzougui et al.~\cite{belazzougui_et_al:LIPIcs.CPM.2021.8} showed that the suffix tree can be preprocessed in linear time so that, given a substring $(i,j)$, one can find the node $u$ of the suffix tree labeled by $p[i...j]$ in constant time. The leaves of the subtree of $u$ correspond to the occurrences of $p[i...j]$ in $p$.
%By maintaining the range minimum and the range maximum query data structures on the leaves of the suffix tree, that can be built in linear time and occupy linear space~\cite{10.1007/11780441_5}, one can find the leftmost and the rightmost occurrences of $x$ in $p$ in constant time.
For each node, we can precompute in linear time, the size of its subtree, the lefmost and rightmost occurrences of its label. This is done by simply traversing the tree from the bottom to the top and propagating the information.
%By storing the size of the subtrees of each node of the suffix tree, which can be precomputed in linear time and space as well, one can also output the total number of occurrences of $p[i..j]$ in $p$ in constant time.
We also preprocess the suffix trees in linear time so that they can fing the lowest common ancestor between two nodes in constant time~\cite{bender2000lca}. Given two substrings $(i,j)$ and $(k,l)$ the length of the label of their common ancestor is their longest common prefix. Analogously, by having the same suffix tree built for the reversed string, we can compute the longest common suffix.
\end{proof}

\begin{corollary}[{of Corollary~\ref{pmgapped:cor:arithmetic_progression} and Proposition~\ref{prop:suffix_tree}}]
\label{cor:imp}
One can preprocess a string $p$ of length $m$ in $O(m)$ time and space to maintain the following queries in constant time: Given a substring $(i,j)$, such that $j-i\ge m/2$, one can output the arithmetic progression of the occurrences of $p[i...j]$ in $p$ in constant time.
\end{corollary}



\subsection{Grammars}
\begin{definition}[Straight-line program~\cite{tit/KiefferY00}]
A \emph{straight-line program} (SLP)  is a context-free grammar (CFG) consisting of a set of non-terminals $X_1, \ldots, X_q$, a set of terminals, an initial symbol $X_q$, and a set of productions, satisfying the following properties:
\begin{itemize}
\item A production consists of a left-hand side and a right-hand side, where the left-hand side is a non-terminal $X_i$ and the right-hand side is a sequence $X_jX_k$, where $j,k < i$, or a terminal;
\item Every non-terminal is on the left-hand side of exactly one production.
\end{itemize}
\end{definition}


The \emph{expansion} $\str{S}$ of a sequence of terminals and non-terminals $S$ is the string that is obtained by iteratively replacing non-terminals by the right-hand sides in the respective productions, until only terminals remain. We say that $G$ \emph{represents} the expansion $\str{G}$ of its initial symbol.

\begin{definition}[Parse tree]
 The \emph{parse tree} of a SLP is defined as follows: 
\begin{itemize}
\item The root is labeled by the initial symbol;
\item Each internal node is labeled by a non-terminal;
\item If $S$ is the expansion of the initial symbol, then the $i^{\text{th}}$ leaf of the parse tree is labeled by a terminal $S[i]$;
\item A node labeled with a non-terminal $A$ that is associated with a production $A\rightarrow BC$ has two children labeled by $B$ and $C$, respectively.
\end{itemize}
\end{definition}

The \emph{size} of a grammar is the total size of all right-hand sides of all productions. The \emph{height} of a grammar is the height of the parse tree.  

%\begin{definition}[Primary occurrence]
%Let $A$ be a non-terminal of $G$ associated with a production $A \rightarrow BC$. We say that a occurrence $i$ of $p$ in $\str{A}$ is \emph{primary} if $i \le |\str{B}| \le i + |p|-1$.   
%\end{definition}

\section{\texorpdfstring{\kApproxLCS}{LCS with approximately k mismatches}}\label{sec:klcs}
In this section, we prove Theorem~\ref{th:klcs_upper}. Let us first introduce a decision variant of the \kApproxLCS problem. 

\begin{problem}\label{pr:LCS'k-decision}
Two strings $X, Y$ of length at most $n$, integers $k, \ell$, and a constant $\eps > 0$ are given. We must return:
%
\begin{enumerate}
\item YES if $\ell \le \lcsk(X,Y)$;
\item Anything if $\lcsk(X,Y) < \ell \le \mathrm{LCS}_{(1+\eps)k}(X,Y)$; 
\item NO if $\mathrm{LCS}_{(1+\eps)k}(X,Y) < \ell$.
\end{enumerate}
%
If we return YES, we must also give a \emph{witness pair} of length-$\ell$ substrings $S_1$ and $S_2$ of $X$ and $Y$, respectively, such that $\HD(S_1,S_2)\le (1+\eps)k$. 
\end{problem}

The decision variant of the \kApproxLCS problem can be reduced to the following $(c,r)$-\NN problem. 

\begin{problem}\label{pr:NN}
In the $(c,r)$-\NN problem with failure probability~$f$, the aim is, given a set $P$ of $n$ points in $\mathbb{R}^d$, to construct a data structure supporting the following queries: given any point $q\in \mathbb{R}^d$, if there exists $p \in P$ such that $\norm{p-q} \leq r$, then return some point $p' \in P$ such that $\norm{p'-q} \leq cr$ with probability at least $1-f$.
\end{problem}

Using the reduction, we will show our first solution to the \kApproxLCS decision problem based on the result of Andoni and Razenshteyn~\cite{DBLP:conf/stoc/AndoniR15}, who showed that for any constant $f$, there is a data structure for the $(c,r)$-\NN problem that has $\Oh(n^{1+\rho+o(1)}+d \cdot n)$ size, $\Oh(d\cdot n^{\rho+o(1)})$ query time, and $\Oh(d \cdot n^{1+\rho+o(1)})$ preprocessing time, where $\rho = 1/(2c^2-1)$.

\begin{lemma}\label{lm:opt_NN}
Assume an alphabet of constant size $\sigma$. The decision variant of the \kApproxLCS problem can be solved in space $\Oh(n^{1+ 1/(1+2\eps) + o(1)})$ and time $\Oh(n^{1+ 1/(1+2\eps) + o(1)})$. The answer is correct with constant probability. 
\end{lemma}
\begin{proof}
Let $P$ be the set of all length-$\ell$ substrings of $X$ and $Q$ be the set of all length-$\ell$ substrings of $Y$,
all encoded in binary using the morphism $\mu$ (see Section~\ref{lcs:sec:prelim}). We start by applying the dimension reduction procedure of Corollary~\ref{cor:dim_reduction} to $P$ and $Q$ with $\alpha = 1/(\log\log n)^{\Theta(1)}$ and $\beta = 2$ to obtain sets $P'$ and $Q'$. We can implement the procedure in $\Oh(\sigma n \log^2 n (\log\log n)^{\Theta(1)}) = \Oh(n \log^{2+o(1)} n)$ time by encoding $X, Y$ using $\mu$ and running the FFT algorithm~\cite{FischerPaterson} for each of the $\Oh(\log^{1+o(1)} n)$ rows of the matrix and $\mu(X), \mu(Y)$. 

To solve the decision variant of \kApproxLCS, we build the data structure of Andoni and Razenshteyn~\cite{DBLP:conf/stoc/AndoniR15} for $(\sqrt{(1+\eps)(1-\alpha)}, \sqrt{(1+\alpha)k})$-\NN over~$Q'$. We make a query for each string in $P'$. If, queried for $\sk_{\alpha}(S_1)\in P'$, where $S_1$ is a length-$\ell$ substring of $X$, the data structure outputs $\sk_{\alpha}(S_2)\in Q'$, where $S_2$ is a length-$\ell$ substring of $Y$, then we compute $\norm{\sk_{\alpha}(S_1)- \sk_{\alpha}(S_2)}^2$. If it is at most $(1+\eps)k$, we output YES and the witness pair $(S_1,S_2)$ of substrings. As the length of vectors in $P'$, $Q'$ is $d = \Oh(\log^{1+o(1)} n)$, we obtain the desired complexity. 

To show that the algorithm is correct, suppose that there are length-$\ell$ substrings $S_1$ and $S_2$ of $X$ and $Y$, respectively, with $\HD(S_1, S_2) \le k$. By Corollary~\ref{cor:dim_reduction}, $\norm{\sk_\alpha(S_1),\sk_\alpha(S_2)}\le \sqrt{(1+\alpha)k}$ holds with probability at least $1-1/n$. Then, when querying for $\sk_\alpha(S_1)$, with constant probability the data structure will output a string $\sk_\alpha(S'_2)$ such that $\norm{\sk_\alpha(S_1) - \sk_\alpha(S'_2)}^2 \le (1+\eps)(1-\alpha^2) k \le (1+\eps) k$. Then, our algorithm will return YES. 

On the other hand, if we output YES with a witness pair $(S_1,S_2)$, then $\norm{\sk_\alpha(S_1) - \sk_\alpha(S_2)}^2\le (1+\eps)k$ implies $\HD(S_1,S_2)\le (1+\eps)k$ with high probability by Corollary~\ref{cor:dim_reduction}.
\end{proof}

While this solution is very fast, it uses quite a lot of space. Furthermore, the data structure of~\cite{DBLP:conf/stoc/AndoniR15} that we use as a black box applies highly non-trivial techniques. To overcome these two disadvantages, we will show a different solution based on a careful implementation of ideas first introduced in~\cite{substringNN} that showed a data structure for approximate text indexing with mismatches. In~\cite{DBLP:journals/algorithmica/KociumakaRS19}, the authors developed these ideas further to show an algorithm that solves the \kApproxLCS problem in $\Oh(n^{1+1/(1+\eps)} )$ space and $\Oh (n^{1+1/(1+\eps)} \log^2 n)$ time for $\eps\in(0,2)$ with constant error probability. In this work, we significantly improve and simplify the approach to show the following result:  

\begin{theorem}\label{th:LCS'k-decision}
Assume an alphabet of arbitrary size $\sigma = n^{\Oh(1)}$. The decision variant of \kApproxLCS can be solved in $\Oh(n^{1+1/(1+\eps)} \log^2 n)$ time and $\Oh(n)$ space. The answer is correct with constant probability. 
\end{theorem}

Let us defer the proof of the theorem until Section~\ref{lcs:sec:decision} and start by explaining how we use Lemma~\ref{lm:opt_NN} and Theorem~\ref{th:LCS'k-decision} and the \twentyquestions game to show Theorem~\ref{th:klcs_upper}.

\begin{proof}[Proof of Theorem~\ref{th:klcs_upper}]
We will rely on the modified version of the \twentyquestions game that we
  described in Section~\ref{lcs:sec:20questions}. In our case, $A = \lcsk(X,Y)$ and
  $B = \lcske(X,Y)$. For Carole, we use either the algorithm of
  Lemma~\ref{lm:opt_NN}, or the algorithm of Theorem~\ref{th:LCS'k-decision}, 
  with an additional procedure verifying the witness pair $(S_1,S_2)$ character by character to check that it indeed satisfies $\HD(S_1,S_2)\le (1+\eps)k$.
  We output the longest pair of (honest) witness
  substrings found across all iterations. We will return a correct answer
  assuming that the fraction of errors is $\rho <\frac13$. Recall that the algorithm solves the decision variant of the \kApproxLCS problem incorrectly with probability not exceeding a constant $\delta$, and we can ensure $\delta < \frac13$ by repeating it a constant number of times. It means that Carole can answer an individual question erroneously with probability less than~$\frac13$. Therefore, for a sufficiently large constant in the number of queries $Q = \Theta(\log n)$, the fraction of erroneous answers is $\rho < \frac13$ with high probability by Chernoff--Hoeffding bounds. The claim of the theorem follows immediately from Lemma~\ref{lm:opt_NN} and Theorem~\ref{th:LCS'k-decision}.
\end{proof}

\subsection{Proof of Theorem~\ref{th:LCS'k-decision}}\label{lcs:sec:decision}
We first give an algorithm for the decision version of the \kApproxLCS problem that uses $\Oh(n \log n)$ space and $\Oh(n^{1+1/(1+\eps)} \log n + \sigma n \log^2 n)$ time, and then we improve the space and time complexity. 

We assume to have fixed a Karp--Rabin fingerprinting function $\varphi$ for a prime $q = \Omega(\max\{n^5, \sigma\})$ and an integer $r \in \mathbb{Z}_q$. With error probability inverse polynomial in $n$, we can find such $q$ in $\Oh(\log^{\Oh(1)}n)$ time;
see~\cite{DBLP:journals/moc/TaoCH12,agrawal2004primes}. 

Let~$\Projections$ be the set of all projections of strings of length $\ell$ onto a single position, i.e., the value $\pi_i(S)$ of the $i$-th projection on a string $S$ of length $\ell$ is simply its $i$-th character $S[i]$. More generally, for a length-$\ell$ string $S$ and a function $h=(\pi_{a_1},\ldots,\pi_{a_m}) \in \Projections^m$, we define $h(S)$ as $S[a_{1}] S[a_{2}] \cdots S[a_{m}]$.

Let $p_1 = 1 - k / \ell$ and $p_2 = 1 - (1+\eps) k / \ell$. We assume that $(1+\eps) k< \ell$ in order to guarantee $p_1>p_2>0$; the problem is trivial if $(1+\eps)k \ge \ell$. 
Further, let $m = \ceil { \log_{p_2}{\tfrac{1}{n}} }$.

We choose a set $\Hashes$ of $L = \Theta(n^{1/(1+\eps)})$ hash functions in $\Projections^m$ uniformly at random. Let $\Collisions^{\Hashes}_{\ell}$ be the mutliset of all collisions of length-$\ell$ substrings of $X$ and $Y$ under the functions from $\Hashes$, i.e. $\Collisions^{\Hashes}_{\ell} = \{(X[i,i+\ell-1], Y[j,j+\ell-1],h) : \varphi(h(X[i,i+\ell-1])) = \varphi(h(Y[j,j+\ell-1])), 1 \le i \le |X|-\ell, 1 \le j \le |Y|-\ell\}$. 

We will perform two tests. The first test chooses an arbitrary subset $\Collisions'\subseteq \Collisions^{\Hashes}_{\ell}$ of size $|\Collisions'|=\min\{4nL, |\Collisions^{\Hashes}_{\ell}|\}$ and, for each collision $(S_1, S_2, h)\in \Collisions'$, computes $\norm{\sk_\eps(S_1) - \sk_\eps (S_2)}^2$. If this value is at most $(1+\eps) k$, then the algorithm returns YES and the pair $(S_1, S_2)$ as a witness. The second test chooses a collision $(S_1, S_2, h) \in \Collisions^{\Hashes}_{\ell}$ uniformly at random and computes the Hamming distance between $S_1$ and $S_2$ character by character in $\Oh(\ell) = \Oh(n)$ time. If the Hamming distance is at most $(1+\eps)k$, the algorithm returns YES and the witness pair $(S_1, S_2)$. Otherwise, the algorithm returns NO. See Algorithm~\ref{alg:LSH}.

\begin{algorithm}[ht]
\caption{\kApproxLCS (decision variant)}
\begin{algorithmic}[1]
\State Choose a set $\Hashes$ of $L$ functions from $\Projections^m$ uniformly at random
\State $\Collisions^{\Hashes}_{\ell}\!=\!\{(S_1,S_2, h) : S_1, S_2\text{---length-$\ell$ substrings of $X,Y$ resp. and }\varphi(h(S_1)) = \varphi(h(S_2))\}$\label{ln:hashes}

\State Choose an arbitrary subset $\Collisions' \subseteq \Collisions^{\Hashes}_{\ell}$ of size $\min\{4nL,|\Collisions^{\Hashes}_{\ell}|\}$
\State Compute $\sk_\eps(\cdot)$ sketches for all length-$\ell$ substrings of $X, Y$  \label{ln:sketches}
\For {$(S_1, S_2, h) \in \Collisions'$} \label{ln:test1}
	\If {$\norm{\sk_\eps (S_1) - \sk_\eps (S_2)}^2 \le (1+\eps) k$} \label{ln:Hamming_dist}\Return $(\text{YES},(S_1,S_2))$
	\EndIf
\EndFor

\State Draw a collision $(S_1, S_2, h) \in \Collisions^{\Hashes}_{\ell}$ uniformly at random \label{ln:test2}
\If {$\HD (S_1, S_2) \le (1+\eps) k$} \label{ln:test3}\Return $(\text{YES},(S_1,S_2))$
\EndIf\\
\Return NO
\end{algorithmic}
\label{alg:LSH}
\end{algorithm}

We must explain how we compute $\Collisions^{\Hashes}_{\ell}$ and choose the collisions that we test. We consider each hash function $h \in \Hashes$ in turn. Let $h = (\pi_{a_1},\ldots,\pi_{a_m})$. Recall that for a string $S$ of length~$\ell$ we define $h(S)$ as $S[a_{1}] S[a_{2}] \cdots S[a_{m}]$. Consequently, $\varphi(h(S)) =  (\sum_{i=1}^m r^{i-1} S[a_{i}]) \bmod q$. We create a vector $U$ of length $\ell$ where each entry is initialised with $0$. For each $i$, we add $r^{i-1} \bmod q$ to the $a_{i}$-th entry of $U$. Finally, we run the FFT algorithm~\cite{FischerPaterson} for $U$ and $X, Y$ in the field $\mathbb{Z}_q$, and sort the resulting values. We obtain a list of sorted values that we can use to generate the collisions. Namely, consider some fixed value $z$. Assume that there are $x$ substrings of $X$ and $y$ substrings of $Y$ of length $\ell$ such that the fingerprint of their projection is equal to $z$. The value $z$ then gives $xy$ collisions, and we can generate each one of them in constant time. This explains how to choose the subset $\Collisions'$ in $\Oh(nL \log n)$ time.

To draw a collision from $\Collisions^{\Hashes}_{\ell}$ uniformly at random, we could simply compute the total number of collisions across all functions $h \in \Hashes$, draw a number in $[1,  |\Collisions^{\Hashes}_{\ell}|]$, and generate the corresponding collision. However, this would require to generate the collisions twice. Instead, we use the weighted reservoir sampling algorithm~\cite{reservoir}. We divide all collisions into subsets according to the values of fingerprints. We assume that the weighted reservoir sampling algorithm receives the fingerprint values one-by-one, as well as the number of corresponding collisions. At all times, the algorithm maintains a ``reservoir'' containing one fingerprint value and a random collision corresponding to this value. When a new value $z$ with $xy$ collisions arrives, the algorithm replaces the value in the reservoir with $z$ and a random collision with some probability. Note that to select a random collision it suffices to choose a pair from $[1,x] \times [1,y]$ uniformly at random. It is guaranteed that if for a value $z$ we have $xy$ collisions, the algorithm will select $z$ with probability $xy/|\Collisions^{\Hashes}_{\ell}|$. Consequently, after processing all values, the reservoir will contain a collision chosen from $\Collisions^{\Hashes}_{\ell}$ uniformly at random.

\begin{lemma}\label{lm:complexity}
Algorithm~\ref{alg:LSH} uses $\Oh(n^{1+1/(1+\eps)} \log n + \sigma n \log^2 n)$ time and $\Oh(n \log n)$ space. 
\end{lemma}
\begin{proof}
Computing the sketches (Line~\ref{ln:sketches}) takes $\Oh(\sigma n \log^2 n)$ time and $\Oh(n \log n)$ space. Computing the collisions and choosing the collisions to test takes $\Oh(n^{1+1/(1+\eps)} \log n)$ time and $\Oh(n)$ space in total. Testing $\min \{4nL, |\Collisions^{\Hashes}_{\ell}|\}$ collisions (Line~\ref{ln:test1}) takes $\Oh(n^{1+1/(1+\eps)} \log n)$ time and constant space. Computing the Hamming distance for a random collision (Line~\ref{ln:test3}) takes $\Oh(\ell) = \Oh(n)$ time and constant space.
\end{proof}

\begin{lemma}\label{lm:hash_function_exists}
Let $S_1$ and $S_2$ be two length-$\ell$ substrings of $X$ and $Y$, respectively, with $\HD(S_1, S_2) \le k$. 
If $L=\Theta(n^{1/(1+\eps)})$ is large enough, then, with probability at least $3/4$, there exists a function $h\in \Hashes$ such that $h(S_1)=h(S_2)$.
\end{lemma}
\begin{proof}
Consider a function $h=(\pi_{a_1},\ldots,\pi_{a_m})$ drawn from $\Projections^m$ uniformly at random. The probability of $h(S_1)=h(S_2)$ is at least $p_1^m$.
Due to $p_1 \le 1$, we have \[p_1^m = p_1^{\ceil{\log_{p_2}\frac1n}} \ge p_1^{1+\log_{p_2}\frac1n}=p_1\cdot n^{-\frac{\log p_1}{\log p_2}}.\]
Moreover, $p_1 = 1-\frac{k}{\ell}$ and $(1+\eps)k < \ell$ yield $p_1 > 1-\frac{1}{1+\eps}=\frac{\eps}{1+\eps}$,
whereas Bernoulli's inequality implies $p_2 = 1-(1+\eps)\frac{k}{\ell} \le (1-\frac{k}{\ell})^{1+\eps}=p_1^{1+\eps}$,
i.e., $\log p_2  \le (1+\eps)\log p_1$.
Therefore, \[p_1^m \ge p_1\cdot n^{-\frac{\log p_1}{\log p_2}}\ge\tfrac{\eps}{1+\eps}\cdot n^{-\frac{1}{1+\eps}}.\]
Hence, we can choose the constant in $L=|\Hashes|$ so that the claim of the lemma holds.
\end{proof}

\begin{lemma}\label{lm:bad_collisions}
If $|\Collisions^{\Hashes}_{\ell}| > 4 nL$ and $(S_1,S_2, h)$ is a uniformly random element of $\Collisions^{\Hashes}_{\ell}$, then $\Prob[\HD(S_1,S_2) \ge (1+\eps) k] \le \frac12$.
\end{lemma}
\begin{proof}
Consider length-$\ell$ substrings $S_1, S_2$ of $X, Y$, respectively, such that $\HD(S_1, S_2) \ge (1+\eps)k$, and a hash function $h$. Let us bound the probability of $(S_1,S_2, h) \in \Collisions^{\Hashes}_{\ell}$. There two possible cases: either $h(S_1) \neq h(S_2)$ but $\varphi(h(S_1)) = \varphi(h(S_2))$, or $h(S_1) = h(S_2)$. The probability of the first event is bounded by the collision probability of Karp--Rabin fingerprints, which is at most $1/n$. Let us now bound the probability of the second event. Since $\HD (S_1,S_2)\ge (1+\eps)k$, we have $\Prob [h (S_1) = h (S_2)] \le p_2^{m} \le 1/n$,  
where the last inequality follows from the definition of $m$. Therefore, the probability that for some function $h \in \Hashes$ we have $\varphi(h(S_1)) = \varphi(h(S_2))$ is at most $2/n$. 

In total, we have $n^2 |\Hashes|$ possible triples $(S_1, S_2 ,h)$ so by linearity of expectation, we conclude that the expected number of such triples is at most $\frac{2}{n} n^2 L =2n L$. Therefore the probability to hit a triple $(S_1, S_2, h)$ such that $\HD(S_1, S_2) \ge (1+\eps)k$ when drawing from $\Collisions^{\Hashes}_{\ell}$ uniformly at random is at most $2nL / |\Collisions^{\Hashes}_{\ell}| \le 2nL / 4nL = 1/2$.
\end{proof}

Below, we combine the previous results to prove that, with constant probability, Algorithm~\ref{alg:LSH} correctly solves
the decision variant of the \kApproxLCS problem.
Note that we can reduce the error probability to an arbitrarily small constant $\delta>0$: it suffices to repeat the algorithm a constant number of times. 

\begin{corollary}
With non-zero constant probability, Algorithm~\ref{alg:LSH} solves the decision variant of \kApproxLCS correctly.
\end{corollary}
\begin{proof}
Suppose first that $\ell \le \lcsk(X,Y)$, which means that there are two length-$\ell$ substrings $S_1, S_2$ of $X, Y$ such that $\HD(S_1, S_2) \le k$. By Lemma~\ref{lm:hash_function_exists}, with probability at least $3/4$, there exists a function $h\in \Hashes$ such that $h(S_1)=h(S_2)$.
In other words, $(S_1, S_2, h) \in \Collisions^{\Hashes}_{\ell}$ with probability at least $\frac34$. If $|\Collisions^{\Hashes}_{\ell}| < 4nL$, we will find this triple and it will pass the test with probability at least $1-n^{-6}$.
If $|\Collisions^{\Hashes}_{\ell}| \ge 4nL$, then by Lemma~\ref{lm:bad_collisions} the Hamming distance between $S_1, S_2$, where $(S_1, S_2, h)$ was drawn from $\Collisions^{\Hashes}_{\ell}$ uniformly at random, is at most $(1+\eps)k$ with probability $\ge 1/2$, and therefore this pair will pass the test with probability $\ge 1/2$. It follows that in this case the algorithm outputs YES with constant probability.

Suppose now that $\ell > \lcske(X,Y)$. In this case, the Hamming distance between any pair of length-$\ell$ substrings of $X$ and $Y$ is at least $(1+\eps)k$, so none of them will ever pass the second test and none of them will pass the first test with constant probability.
\end{proof}

We now improve the space of the algorithm to linear. Note that the only reason why we needed $\Oh(n \log n)$ space is that we precompute and store the sketches for the Hamming distance. Below we explain how to overcome this technicality.

First, we do not precompute the sketches. Second, we process the collisions in $\Collisions'$ in batches of size $n$. Consider one of the batches, $\mathcal{B}$. For each collision $(S_1,S_2, h) \in \mathcal{B}$ we must compute $\norm{\sk_\eps(S_1) - \sk_\eps(S_2)}^2$. 
We initialize a counter for every collision, setting it to zero initially. The number of rounds in the algorithm will be equal to the length of the sketches, and, in round $i$, the counter for a collision $(S_1, S_2, h) \in \mathcal{B}$ will contain the squared $L_2$ distance between the length-$i$ prefixes of $\sk_\eps(S_1)$ and $\sk_\eps(S_2)$. In more detail, let $\mathcal{S}$ be the set of all substrings of $X, Y$ that participate in the collisions in $\mathcal{B}$. Recall that all these substrings have length $\ell$. At round $i$, we compute the $i$-th coordinate of the sketches of the substrings in $\mathcal{S}$. By definition, the $i$-th coordinate is the dot product of the $i$-th row of $c \cdot M$, where $c$ and $M$ are as in Corollary~\ref{cor:dim_reduction}, and a substring encoded using $\mu$. Hence, we can compute the coordinate using the FFT algorithm~\cite{FischerPaterson} in $\Oh(\sigma n \log n)$ time and $\Oh(n)$ space. When we have the coordinate $i$ computed, we update the counters for the collisions and repeat.

At any time, the algorithm uses $\Oh(n)$ space.
Compared to the time consumption proven in Lemma~\ref{lm:complexity}, the algorithm spends an additional $\Oh(\sigma n^{1+1/(1+\eps)} \log^2 n)$ time for computing the coordinates of the sketches.
Therefore, in total the algorithm uses $\Oh(\sigma n^{1+1/(1+\eps)} \log^2 n) = \Oh(n^{1+1/(1+\eps)} \log^2 n)$ time and $\Oh(n)$ space. 
For constant-size alphabets, this completes the proof of Theorem~\ref{th:LCS'k-decision}. For alphabets of arbitrary size, we replace the sketches from Section~\ref{lcs:sec:prelim} with the sketches defined in~\cite{DBLP:journals/algorithmica/KociumakaRS19} to achieve the desired complexity.
We note that we could use the sketches~\cite{DBLP:journals/algorithmica/KociumakaRS19} for small-size alphabets as well, but their lengths hide a large constant. 






\section{Experiments}\label{sec:implem}
We now present results of experimental evaluation of the second solution presented in Theorem~\ref{th:klcs_upper}.

\textit{Methodology and test environment.} The baselines and our solution are written in C++11 and compiled with optimizations using gcc 7.4.0. The experimental results were generated on an Intel Xeon E5-2630 CPU using 128 GiB RAM. To ensure the reproducibility of our results, our complete experimental setup, including data files, is available at \url{https://github.com/fnareoh/LCS\_Approx\_k\_mis}.

\textit{Baseline.} The only other solution to the \kApproxLCS problem was presented in~\cite{DBLP:journals/algorithmica/KociumakaRS19}, however, it has a worse complexity and is likely to be unpractical because it uses a very complex class of hash functions. We therefore chose to compare our algorithm against algorithms for the \kLCS problem. To the best of our knowledge, none of the existing algorithms has been implemented. We implemented the solution to \kLCS by Flouri et al., which we refer to as FGKU~\cite{DBLP:journals/ipl/FlouriGKU15}. (The other algorithms seem to be too complex to be efficient in practice.) The main idea of the algorithm of Flouri et al.\ is that if we know that the longest common substring with $k$ mismatches is obtained by a substring of $X$ that starts at a position $p$ and a substring of $Y$ that starts at a position $p+i$, then we can find it by scanning $X$ and $Y[i,|Y|]$ in linear time; see Algorithm~\ref{alg:FGKU} for details.

\begin{algorithm}[ht]
\caption{FGKU algorithm}
\begin{algorithmic}[1]
\State $n \gets  |X|$, $m \gets  |Y|$
\State $l \gets  0$, $r_1 \gets  0$, $r_2 \gets  0$
\For {$d \gets-m+1$ to $n-1$} 
	\State $i \gets  \max(-d,0)+d$, $j \gets  \max(-d,0)$
	\State $Q \gets  \emptyset$, $s \gets  0$, $p \gets  0$ 
	\While {$p \leq \min(n-i,m-j)-1$}
	\If {$X[i+p] \neq Y[j+p]$}
		\If {$|Q| = k$}
			\State $s \gets  \min Q + 1$
			\State {\footnotesize DEQUEUE}($Q$)
		\EndIf
		\State {\footnotesize ENQUEUE}($Q,p$)
	\EndIf
	\State $p \gets p+1$
	\If{$p-s > l$}
		\State $l \gets p-s$, $r_1 \gets i+s$, $r_2 \gets j+s$
	\EndIf
	\EndWhile
\EndFor
\end{algorithmic}
\label{alg:FGKU}
\end{algorithm}

\textit{Details of implementation.}
We made several adjustments to the theoretical algorithm we described. First, we use the fact that $A = \mathrm{LCS}(X,Y)+k \le \lcsk(X,Y) \le B = (k+1)\cdot\mathrm{LCS}(X,Y)+k$ to bound the interval in the \twentyquestions game. We also treated the number of questions in the  \twentyquestions game and $L$, the size of the set of hash functions $\Hashes$, as parameters that trade time for accuracy, and put the number of questions to $2 \log (B-A)$ in the \twentyquestions game and $L = n^{1/(1+\eps)}/16$. In Line~\ref{ln:Hamming_dist} of Algorithm~\ref{alg:LSH}, we used sketches to estimate the Hamming distance. In practice, we computed the Hamming distance via character-by-character comparison when $\ell$ is small compared to $k$ and via kangaroo jumps otherwise~\cite{10.1145/8307.8309}. Also, when the length $\ell$ in Algorithm~\ref{alg:LSH} is smaller than $2 \log n$, we compute the hash values of the $\ell$-length substrings of $S_1$ and $S_2$ naively, instead of using the FFT algorithm~\cite{FischerPaterson}. 

\begin{figure}[ht!]
\centering
    \begin{subfigure}{.5\textwidth}
        \centering
        \captionsetup{justification=centering}
        \figLCS{0.45}{figs/random_10.png}
        \caption{Random, $k = 10$}
    \end{subfigure}%
    \begin{subfigure}{0.5\textwidth}
        \centering
        \captionsetup{justification=centering}
        \figLCS{0.45}{figs/e_coli_10.png}
        \caption{E. coli, $k = 10$}
    \end{subfigure}
\caption{Comparison of the FGKU algorithm versus our algorithm for $k = 10$ and different values of $\eps$. Large standard deviation for length $60000$ is caused by an outlier with very long longest common substring with $k$ mismatches.}
\label{fig:runtime_10}
\end{figure}

\begin{figure}[ht!]
\centering
    \begin{subfigure}{.5\textwidth}
        \centering
        \captionsetup{justification=centering}
        \figLCS{0.45}{figs/random_25.png}
        \caption{Random, $k = 25$}
    \end{subfigure}%
    \begin{subfigure}{0.5\textwidth}
        \centering
        \captionsetup{justification=centering}
        \figLCS{0.45}{figs/e_coli_25.png}
        \caption{E. coli, $k = 25$}
    \end{subfigure}     
\caption{Comparison of the FGKU algorithm versus our algorithm for $k = 25$ and different values of $\eps$.}
\label{fig:runtime_25}
\end{figure}
 
\begin{figure}[ht!]
\centering   
    \begin{subfigure}{.5\textwidth}
        \centering
        \captionsetup{justification=centering}
        \figLCS{0.45}{figs/random_50.png}
        \caption{Random, $k = 50$}
    \end{subfigure}%
    \begin{subfigure}{0.5\textwidth}
        \centering
        \captionsetup{justification=centering}
        \figLCS{0.45}{figs/e_coli_50.png}
        \caption{E. coli, $k = 50$}
    \end{subfigure}        
\caption{Comparison of the FGKU algorithm versus our algorithm for $k = 50$ and different values of $\eps$.}
\label{fig:runtime_50}
\end{figure}

\textit{Data sets and results.}
We considered $k \in \{10, 25, 50\}$ and $\eps \in \{1.0, 1.25, 1.5, 1.75, 2.0\}$. We tested the algorithms on pairs of random strings (each character is selected independently and uniformly from a four-character alphabet $\{A, T, G, C\}$) and on pairs of strings extracted at random from the E. coli genome. The lengths of the strings in each pair are equal and vary from $0$ to $60000$ with a step of $5000$. All timings reported are averaged over ten runs. Figures~\ref{fig:runtime_10}-~\ref{fig:runtime_50} show the results for $k = 10, 25, 50$. We note that for $\eps = 1$ and $k = 10, 25$, the standard deviation of the running time on the E. coli data set is quite large, which is probably caused by our choice of the method to compute the Hamming distance between substrings, but for all other parameter combinations it is within the standard range. We can see that the time decreases when $\eps$ grows, which is coherent with the theoretical complexity. 


As for the accuracy, note that our algorithm cannot return a pair of strings at Hamming distance more than $(1+\eps) k$, and so the only risk is returning strings which are too short. Consequently, we measured the accuracy of our implementation by the ratio of the length $\lcsak(X, Y)$ returned by our algorithm divided by $\lcsk(X, Y)$ computed by the dynamic programming. We estimate $r_{\min}(\eps, k) = \min_{X,Y}(\lcsak(X,Y)/\lcsk(X,Y))$ and $r_{\max}(\eps, k) = \max_{X,Y}(\lcsak(X,Y)/\lcsk(X,Y))$
by computing $\lcsak$ and $\lcsk$ for $10$ pairs of strings for each length from $5000$ to $60000$ with step of $5000$, as well as the error rate, i.e. the percentage of experiments where $\lcsak(X,Y)$ is shorter than $\lcsk(X,Y)$ (see Table~\ref{tb:eps}). Not surprisingly, $r_{\min}$ and~$r_{\max}$ grow as $k$ and $\eps$ grow, while the error rate drops. Even though there is no theoretical upper bound on $r_{\max}$, the latter is at most $2.24$ at all times. We also note that even in the cases when the error rate is non-negligible, $\lcsak \ge 0.86 \cdot \lcsk$, in other words, our algorithm returns a reasonable approximation of $\lcsk$.


\newcolumntype{?}{!{\vrule width 1pt}}
\newcommand{\err}{\mathrm{err}}
\begin{center}
\begin{table}[ht!]
\center
\begin{tabular}{| c ? c | c | c | c| c| c ? c | c | c | c | c | c |}
\hline
 & \multicolumn{6}{c?}{Random} & \multicolumn{6}{c|}{E. coli} \\ 
\hline
 & \multicolumn{2}{c|}{$k = 10$} & \multicolumn{2}{|c|}{$k = 25$} & \multicolumn{2}{|c?}{$k = 50$} & \multicolumn{2}{c|}{$k = 10$} & \multicolumn{2}{c|}{$k = 25$} & \multicolumn{2}{c|}{$k = 50$} \\ 
\hline
\hline

\multirow{ 2}{*}{$\eps = 1.0$} & 0.95 & 1.41 & 1.12 & 1.46 & 1.27 & 1.54 & 0.89 & 1.34 & 0.94 & 1.48 & 0.97 & 1.59\\ 
\cline{2-13}
& \multicolumn{2}{c|}{$\err = 3\%$}  & \multicolumn{2}{c|}{$\err = 0\%$}   & \multicolumn{2}{c?}{$\err = 0\%$}   & \multicolumn{2}{c|}{$\err = 33\%$}   & \multicolumn{2}{c|}{$\err = 13\%$}  & \multicolumn{2}{c|}{$\err = 3\%$}\\ 
\hline
\multirow{ 2}{*}{$\eps = 1.25$}  & 0.97 & 1.47 & 1.15 & 1.63 & 1.44 & 1.78 & 0.88  & 1.48 & 0.98 & 1.56 & 0.99 & 1.73\\ 
\cline{2-13}
& \multicolumn{2}{c|}{$\err = 1\%$}  & \multicolumn{2}{c|}{$\err = 0\%$}   & \multicolumn{2}{c?}{$\err = 0\%$}   & \multicolumn{2}{c|}{$\err = 28\%$}   & \multicolumn{2}{c|}{$\err = 5\%$}  & \multicolumn{2}{c|}{$\err = 3\%$}\\ 
\hline
\multirow{ 2}{*}{$\eps = 1.5$}  & 1.05 & 1.57 & 1.37 & 1.76 & 1.55 & 1.91  & 0.88 & 1.45 & 0.96 & 1.67 & 0.99 & 1.89\\ 
\cline{2-13}
& \multicolumn{2}{c|}{$\err = 0\%$}  & \multicolumn{2}{c|}{$\err = 0\%$}   & \multicolumn{2}{c?}{$\err = 0\%$}   & \multicolumn{2}{c|}{$\err = 17\%$}   & \multicolumn{2}{c|}{$\err = 3\%$}  & \multicolumn{2}{c|}{$\err = 3\%$}\\  
\hline
\multirow{ 2}{*}{$\eps = 1.75$} & 1.02 & 1.69 & 1.46 & 1.86 & 1.72 & 2.12 & 0.88 & 1.58 & 0.95 & 1.84 & 1.02 & 2.15\\ 
\cline{2-13}
& \multicolumn{2}{c|}{$\err = 0\%$}  & \multicolumn{2}{c|}{$\err = 0\%$}   & \multicolumn{2}{c?}{$\err = 0\%$}   & \multicolumn{2}{c|}{$\err = 17\%$}   & \multicolumn{2}{c|}{$\err = 2\%$}  & \multicolumn{2}{c|}{$\err = 0\%$}\\ 
\hline
\multirow{ 2}{*}{$\eps = 2.0$}  & 1.10 & 1.72 & 1.59 & 2.00 & 1.89 & 2.24 & 0.91  & 1.77 & 1.01 & 2.10 & 1.00 & 2.19\\ 
\cline{2-13}
& \multicolumn{2}{c|}{$\err = 0\%$}  & \multicolumn{2}{c|}{$\err = 0\%$}   & \multicolumn{2}{c?}{$\err = 0\%$}   & \multicolumn{2}{c|}{$\err = 9\%$}   & \multicolumn{2}{c|}{$\err = 0\%$}  & \multicolumn{2}{c|}{$\err = 1\%$}\\ 
\hline
\end{tabular} 
\caption{Accuracy of the \kApproxLCS algorithm. For each $k$ and $\eps$, we show $r_{\min}(\eps, k)$, $r_{\max}(\eps, k)$, as well as the error rate.}
\label{tb:eps}
\end{table}
\end{center}





\bibliography{main}

\end{document}
