% !TEX root = main.tex
In this section, we give a description of our new streaming algorithms for regular expression membership and pattern matching. The structure of the algorithms is very similar, the main difference is how we define a witness (see Definitions~\ref{def:witness-memb} and~\ref{def:witness-matching}). For this reason, we describe the algorithms in parallel. Recall that~$T$ is a string of length $n$ over an alphabet $\Sigma  = \{1, 2, \ldots, \sigma\}$, where $\sigma = n^{\Oh(1)}$, and $A_1, A_2, \ldots, A_d$ are the atomic strings for the regular expression $R$. Recall also that $\Pi$ is the set of canonical prefixes of the atomic strings, defined as $\Pi = \{A_i[1 \dd \min\{2^j, |A_i|\}] : 1\le  i \le d, 0 \le j \le \lceil \log |A_i|\rceil\}$. 

We make use of the following corollary of Fact~\ref{fact:fine_wilf}:

\begin{corollary}[Of Fact~\ref{fact:fine_wilf}]\label{cor:primitive}
For a primitive string $X$ of length $x$, a string $D = XX$ can contain only two occurrences of string $X$, $D[1\dd x]$ and $D[x+1\dd 2x]$.
\end{corollary}

\paragraph{Preprocessing.} 
We start by deleting all transitions $(u,v)$ from $T_C(R)$ that are labelled by atomic strings of lengths larger than $n$. 

Let $\mathcal{F}$ contain all atomic strings $A$ such that there is an $\eps$-transitions path from the endpoint of the transition labelled by $A$ to the final state of $T_C(R)$. In addition, for each atomic string $A$, compute the subset $A_{i_1}, A_{i_2}, \ldots, A_{i_j}$ of atomic strings such that there is an $\eps$-transitions path from the endpoint of the transition labelled by $A_{i_{j'}}$, $1 \le j' \le j$, to the starting point of the transition labelled by $A$.

For each periodic $P \in \Pi$, consider a string $\Delta(P) = P[|P|-\rho+1 \dd ]$, where $\rho$ is the period of $P$. During the preprocessing step, the algorithm computes the generator set of anchors $\A$ (Definition~\ref{def:anchors}) for the string~$W = (\Delta(P))^\infty$. 

Define the \emph{overlap} of two strings $X$ and $Y$ as the maximal length of a suffix of $X$ that equals a prefix of $Y$, $\Pi(P)$ to be the set of all canonical prefixes such that their overlap  with $W$ is at least $2 \rho$, and $\Overlap(P) = \bigcup_{P' \in  \Pi(P)} \{\ell: \ell \text{ is the overlap of } P' \text{ and }  W\}$. The algorithm computes $\Pi(P)$ and $\Overlap(P)$ during the preprocessing step as well. 

For regular expression pattern matching, it also computes the smallest integer $\mu(P)$ such that $p = \mu(P) \cdot \rho \in \occ(P, W)$ and $p$ is a witness for $P$ in $W$ (in the sense of Definition~\ref{def:witness-matching}). 

Finally, the algorithm creates a directed graph $G(P) = (V,E)$ from the compact Thompson automaton~$T_C(R)$. Consider again the string $W = (\Delta(P))^\infty$. For each canonical prefix $P' \in \Pi(P)$, the algorithm creates a node $v \in V$ corresponding to a pair $(P', r)$, where $r$ is the remainder of  the overlap of $P'$ and $W$ modulo $\rho$. Additionally, for every fragment $W[i \dd j] = P'' \in \Pi$ which is anchored by an anchor $a \in \A^\ast(W)$, it creates a node $v \in V$ corresponding to a pair $(P'', j \pmod{\rho})$ (for two identical prefix-remainder pairs, it creates just one node). 

Consider two nodes $v',v'' \in V$. Suppose that $v'$ corresponds to $(P',r')$ and $v''$ to $(P'',r'')$, where $P',P''$ are canonical prefixes of atomic strings $A',A''$, respectively, and $r',r''$ are remainders modulo $\rho$. For $0 < \ell \le 10 \rho$, the algorithm adds an edge $(v',v'')$ of length $\ell$ to $E$ if there is a walk in $T_C(R)$ from the ending state of the transition labelled by $A'$ to the ending state   of the transition labelled by $A''$ such that the concatenation of the labels in this walk equals to a string $L = \Delta(P)[r' \dd] \Delta(P)^\alpha \Delta(P)[\dd r'']$, where the integer power $\alpha$ is chosen so that $|L| = \ell$ (in other words, the concatenation equals to a fragment of $W$ with the offsets defined by $v'$ and $v'$ and of an appropriate length, if such a fragment does not exist, the algorithm does not create the edge). 

It might seem that the resulting graph is infinite, but as we show below, this is not the case.

\begin{claim}\label{claim:shifts}
Consider all occurrences $W[\ell_i \dd r_i]$, $i \in \Zp$, of a string $X$ in $W$. The size of the set $\{r_i \pmod \rho : W[\ell_i \dd r_i] \text{ is anchored by } a\in \A^\ast(W), i \in \Zp\}$ is $\Oh(\log \rho)$.
\end{claim}
\begin{proof}
We consider two cases: $|X| \ge 2\rho$ and $|X| < 2\rho$. In the first case, by Fact~\ref{fact:fine_wilf}, if there is at least one occurrence of $X$ in $W$, then $X$ is periodic with period $\rho$. By Corollary~\ref{cor:primitive}, we have $r_i = q \pmod{\rho}$ for some fixed~$q$ and all $i \in \Zp$. The claim follows from Lemma~\ref{lm:anchors}. 

In the second case, for all $a \in \A^\ast(W)$ such that $a \ge 2\rho$, and for all strings $U \in \Sigma^\ast, V \in \Sigma^\rho$ such that $V \neq W[1 \dd \rho]$, all occurrences of $X$ that contain $|U|+|V|+a$ are contained in $(UVW)[|U|+|V|+a-2\rho+1 \dd |U|+|V|+a-2\rho-1] = W[a-2\rho+1 \dd a-2\rho-1]$. It follows that for all $a,a' \in \A^\ast(W)$ such that $a,a' \ge 2\rho$ and $a = a' \pmod {\rho}$, the sets $\{r_i \pmod {\rho} : W[\ell_i \dd r_i] \text{ is anchored by } a, i \in \Zp\}$ and $\{r_i \pmod {\rho} : W[\ell_i \dd r_i] \text{ is anchored by } a', i \in \Zp\}$ are equal. Moreover, each of them contains only a constant number of elements. Therefore, the size of the set $\{r_i \pmod  {\rho} : W[\ell_i \dd r_i] \text{ is anchored by } a\in \A^\ast(W), a \ge 2\rho, i \in \Zp\}$ is $\Oh(\log \rho)$ by Lemma~\ref{lm:anchors}. It remains to estimate the size of the analogous sets for anchors smaller than $2\rho$. The number of such anchors is $\Oh(\log \rho)$ by Lemma~\ref{lm:anchors}, and each of them can anchor only a constant number of occurrences of $X$. The claim follows.
\end{proof}

\begin{corollary}\label{cor:size_of_G}
$G(P)$ contains $|V| = \Oh(d \log^2 n)$ nodes and $|E| = \Oh(d^2 \log^4 n)$ edges, and can be constructed in $\Oh(\rho \cdot d^3 \log^4 n)$ time.
\end{corollary}
\begin{proof}
The size of $\Pi$, and consequently $\Pi(P)$, is $\Oh(d \log n)$.
For each string $P' \in \Pi(P)$, the remainder $r$ of the overlap of $P'$ and $W$ modulo $\rho$ is defined in a unique way.
By Claim~\ref{claim:shifts}, for each string $P'' \in \Pi$ there are $\Oh(\log \rho)$ different remainders $r$ modulo $\rho$ such that there is an anchored occurrence of $P''$ ending at a position $p = r \pmod \rho$.
Thus $|V| = \Oh(d \log^2 n)$.

Observe that the interval $[0, 10\rho]$ can contain only a constant number of values $\ell$ such that the power $\alpha$ is an integer.
Hence for each pair of nodes we have only a constant number of possible edges, so $|E| = \Oh(d^2 \log^4 n)$.
For each edge, we can check whether it exists in time $\Oh(\rho \cdot |T(R)|) = \Oh(|\rho| \cdot d)$.
\end{proof}

\begin{corollary}[Of Theorem~\ref{thm:detecting_walk_specific_weight}]\label{cor:path_thompson}
There exists an algorithm which, given the graph $G(P)$, its two nodes $v_1$ and $v_2$ and a number $x\leq n$,  decides if there is a walk from $v_1$ to $v_2$ of total weight $x$ in $\Oh(xd^3\polylog n)$ time and $\Oh(d^3\polylog n)$ space and succeeds with high probability.
\end{corollary}
\begin{proof}
 Recall that $\rho \leq n$. 
 We substitute the bounds from Corollary~\ref{cor:size_of_G} into Theorem~\ref{thm:detecting_walk_specific_weight}.
 Then we repeat the algorithm of Theorem~\ref{thm:detecting_walk_specific_weight} $2c \lceil \log n \rceil$ times and output the majority answer to obtain a success probability of at~least~$1-1/n^c$.
\end{proof}


\paragraph{Main phase.}
During the main phase of the algorithm, we run the pattern matching algorithm (Theorem~\ref{th:pattern_matching}) for every $P \in \Pi$. For every non-periodic $P \in \Pi$, we store (at most) two latest witnesses. For every periodic $P \in \Pi$, we run the pattern matching algorithm for $\Delta(P) = P[|P|-\rho+1\dd]$ and $T$, where $\rho$ is the period of $P$. 

\begin{definition}
We say that a fragment $T[i \dd j]$ is a \emph{streak} of a string $X$ if $T[i \dd j] = X^k$ for some integer~$k \ge 1$ and it is maximal, i.e. it cannot be extended neither to the left nor to the right. 
\end{definition}

The pattern matching algorithm detects streaks of $\Delta(P)$ in the arrived prefix of $T$. Every witness $r$ for $P$ belongs to $\occ(P,T)$ and by Observation~\ref{obs:few_progressions} ends in such a streak.  At any moment of the algorithm, we store (at most) two latest streaks and a compact representation of witnesses in $\occ(P,T)$ that end in it. For membership testing we assume that the witnesses are defined as in Definition~\ref{def:witness-memb}, and for pattern matching as in Definition~\ref{def:witness-matching}. Perhaps a bit counter-intuitively, the representation contains witnesses from $\occ(P,T)$ and witnesses for other canonical prefixes as well, the reason for it will become clear later. Formally, the representation of a streak~$S = T[i \dd j]$ consists of the following elements, where $\rho = |\Delta(P)|$:

\begin{enumerate}
\item For each $P' \in \Pi(P)$ and its overlap $\ell$ with $W$, the representation contains $p = i+\ell-1$ if $p \in \occ(P',T)$ and is a witness;
\item All witnesses in $\occ(P,T)$ that belong to the interval $[i \dd  i+12 \rho-1]$;
\item For every $\ell \in \Overlap(P)$, all witnesses in $\occ(P,T)$ that belong to the interval $[i+\ell \dd  (i+\ell-1)+8 \rho]$;
\item For every $P' \in \Pi$, all witnesses $r \in \occ(P',T)$ such that $T[r-|P'|+1\dd r]$ is a fragment of $S$ and is anchored by an anchor in $\A^\ast(F) \cap [i,i+4\rho]$. 
\end{enumerate}

By Observation~\ref{obs:few_progressions} and Lemma~\ref{lm:anchors}, the compact representation of witnesses in a streak has size $\Oh(d \log^2n)$. The algorithm uses the compact representations of witnesses to decide whether a newly detected occurrence $r \in \occ(P,T)$ for some $P \in \Pi$ is a witness:

\begin{lemma}\label{lm:restore_witnesses}
Let $d<n$. Assume that $T[r]$ is the latest arrived character of the text and  that $r \in \occ(P,T)$. Assume that for each non-periodic $P'\in \Pi$ we store two latest witnesses in $\occ(P',T)$, and for each periodic $P' \in \Pi$ we store the two latest streaks of $\Delta(P')$ and compact representations of the witnesses in them. In addition, assume that we store the set of all atomic strings $A$ such that $(r-1)$ is a witness for $\occ(A,T)$. One can decide whether~$r$ is a witness for $P$ in $\Oh(nd^4 \polylog n)$ time and $\Oh(d^3\polylog n)$ (extra) space with high probability.
\end{lemma} 
\begin{proof}
Let $P = A[1 \dd \min\{2^k,|A|\}]$, where $A$ is an atomic string. Consider two cases: $k = 0$ and $k \ge 1$.

\underline{Case 1: $k = 0$.} If $k = 0$, let  $A_{i_1}, A_{i_2}, \ldots, A_{i_j}$ be the atomic strings such that there is an $\eps$-transitions path from the endpoint of the transition labelled by $A_{i_j'}$, $1 \le j' \le j$, to the starting point of the transition labelled by~$A$. The position $r$ is a witness for $P$ iff for some $1 \le j' \le j$, the position $(r-1)$ is a witness for $A_{i_{j'}}$. We can decide whether this holds in $\Oh(d)$ time.

\underline{Case 2: $k \ge 1$.} If $k \ge 1$, the position $r$ is a witness for $P$ iff $r-2^{k-1}$ is a witness for $P[1\dd 2^{k-1}]$. For brevity, denote $r' = r-2^{k-1}$, $P' = P[1\dd 2^{k-1}]$, and $\rho = |\Delta(P')|$. If $P'$ is non-periodic and $r' \in \occ(P',T)$, the algorithm stores it explicitly by Observation~\ref{obs:few_occ}.
Otherwise, by Observation~\ref{obs:few_progressions}, $r'$ belongs to one of the two latest streaks of $\Delta(P')$, let it be a fragment $S = T[i \dd i+ \ell \cdot \rho-1]$. Suppose that $r'$ is a witness for $P'$. Let us first explain the solution for the regular expression membership problem, and then we will show how to modify it for the regular expression pattern matching problem. 

In the membership problem, if $r'$ is a witness for $P'$, then $T[1\dd r'] P[2^{k-1}+1\dd]$ is a partial\sloppy occurrence of $R$ and there is a partition of $T[1 \dd r'] = T[\ell_1 \dd r_1] T[\ell_2 \dd r_2] \ldots T[\ell_m \dd r_m]$, where each $T[\ell_{m'} \dd r_{m'}]$, $1 \le m' < m$, is an atomic string, and $T[\ell_m \dd r_m] = P'$. Let $T[\ell_{m'} \dd r_{m'}]$ be the fragment containing~$i$. We consider two subcases: $r_{m'}-i+1 > 2\rho$ and $r_{m'}-i+1 \le 2\rho$. 

\underline{Case 2(a): $r_{m'}-i+1 > 2\rho$.} We claim that in this subcase $r_{m'}-i+1$ equals the overlap $\ell$ of $T[\ell_{m'} \dd r_{m'}]$ and~$W = (\Delta(P'))^\infty$. By definition, $r_{m'}-i+1 \le \ell$. Suppose that $r_{m'}-i+1 < \ell$. If $\ell-(r_{m'}-i+1)$ is a multiple of~$\rho$, then we obtain that $T[r_{m'}-\rho\dd r_{m'}-1] = \Delta(P')$, a contradiction with the definition of $S$. If~$\ell-(r_{m'}-i+1)$ is not a multiple of $\rho$, then there is an occurrence of $\Delta(P')$ in the prefix $(\Delta(P'))^2$ of $S$ that does not end at positions $\rho$ or $2\rho$. By Corollary~\ref{cor:primitive}, we obtain a contradiction. Therefore, $r_{m'}-i+1 = \ell$ and the compact representation of witnesses in $S$ stores $r_{m'} \in \occ(T[\ell_{m'} \dd r_{m'}], T)$. 

If $m' = m$ or $r_m-r_{m'} \le 8 \rho$, then we are done: if $r'$ is a witness, it must be stored explicitly, and we can check whether it is the case in $\Oh(d \log^2n)$ time. Otherwise, we use the following claim:

\begin{claim}\label{claim:subseq}
There is a sequence $m' = m_{0} < m_{1} < m_{2}  < \cdots < m_{q} = m$ such that each $T[\ell_{m_{q'}} \dd r_{m_{q'}}]$, $1 \le q' < q$, is either anchored by an anchor $a \in \A^\ast(S)$, or has length at least $2\rho$, and for each $1 \le q' \le q$, $\ell_{m_{q'}}- r_{m_{q'-1}} \le 10 \rho$. 
\end{claim}
\begin{proof}
The sequence is built as follows. Let $m_{0} = m'$ and $m_{q'}$ be the latest index added to the sequence. If there is an index $m''$ such that $T[\ell_{m''} \dd r_{m''}]$ has length at least $2\rho$ or $m'' = m$ and $\ell_{m''}- r_{q'} \le 10 \rho$, then set~$m_{q'+1} = m''$ and continue. Otherwise, let $m''$ be the smallest index such that $\ell_{m''}- r_{m_{q'}} \ge 8 \rho$. Note that we also have~$\ell_{m''}- r_{m_{q'}} \le 10 \rho$ (otherwise, the length of $T[\ell_{m''-1} \dd r_{m''-1}]$ would have been larger than $2\rho$). By Lemma~\ref{lm:anchors}, there is $m_{q'} < \tilde{m} \le m''$ such that $T[\ell_{\tilde{m}} \dd r_{\tilde{m}}]$ is anchored by an anchor $a \in \A^\ast(S) \cap [r_{m''} - 4 \rho+1, r_{m''}]$. We set $m_{q'+1} = \tilde{m}$ and continue. 
\end{proof}

Let $v'$ be the node in $G(P)$ corresponding to $(T[\ell_{m'} \dd r_{m'}], r_{m'}-i+1 \pmod \rho)$, and $v$ be the node corresponding to $(P', r_m-i+1 \pmod \rho)$. We have that $j'$ is a witness iff there exists the sequence $m' = m_{0} < m_{1} < m_{2}  < \cdots < m_{q} = m$ as above iff there is a walk from $v$ to $v'$ of length $|T[r_{m'}+1 \dd r_{m}]|\leq n$, which we can check in $\Oh(nd^4 \polylog n)$time and $\Oh(d^3\polylog n)$ extra space with high probability via Corollary~\ref{cor:path_thompson} (we must check whether this condition is verified for each of the $\Oh(d \log^2 n)$ witnesses stored in the compact representation of $S$). \label{it:case2a}

\underline{Case 2(b): $r_{m'}-i+1 \le 2\rho$.} Consider now the second subcase. If $r_m \le 12 \rho$, then we are done: if $r'$ is a witness, it must belong to the compact representation of witnesses in $S$, which can be verified in $\Oh(d \log^2 n)$ time. Otherwise, $r_{m} - r_{m'} \ge 8\rho$.  By Lemma~\ref{lm:anchors_catch_partial_match}\ref{it:anchor_gen} there is $p$, $m' < p \le m$, such that $T[\ell_{p} \dd r_{p}]$ is anchored by an anchor $\A^\ast(F) \cap [i,i+4 \rho-1]$, and therefore $T[\ell_{p} \dd r_{p}]$ is stored in the compact representation of witnesses in $S$. Analogously to Case 2(a), we can show equivalence of the following conditions: $r'$ is a witness; there is a walk in $G(P)$ from the node corresponding to $(T[\ell_{p} \dd r_{p}], r_p-i+1 \pmod \rho)$ to the node corresponding to~$(T[\ell_{m} \dd r_{m}], r_m-i+1 \pmod \rho)$ of length $|T[r_{p}+1 \dd r_{m-1}]|$. We can therefore decide whether $r'$ is a witness via Corollary~\ref{cor:path_thompson} in $\Oh(nd^4 \polylog n)$ time and $\Oh(d^3\polylog n)$ space with high probability.

We now explain how to modify the argument so that it can be used for regular expression pattern matching. Note that in pattern matching, if $r'$ is a witness for $P'$, then there is some position $\ell'$, $1 \le \ell' \le r'$ such that~$r' \in \occ(P',T)$ is a witness. The position $\ell'$ can be inside the streak $S$, i.e. $m'$ can be undefined, making it impossible to apply the argument above. However, we can easily check if this is the case using the integer $\mu(P')$ we computed during the preprocessing step: if $\mu(P') \cdot \rho \ge (r'-i+1)$, then $r'$ is a witness and we are done, and otherwise $\ell' \le i$ (if $r'$ is a witness), $m'$ is defined, and we can apply the argument above.
\end{proof}

\begin{thm}\label{th:memb}
Given a streaming text $T$ of length $n$ and a regular expression $R$ of size $d$. There is a randomised algorithm that solves the membership and the pattern matching problems for $T$ and $R$ in $\Oh(d^3\polylog n)$ space and $\Oh(nd^5\polylog n)$ time per character of the text. The algorithm succeeds with high probability.
\end{thm}
\begin{proof}
If $d \ge n$, we can use Claim~\ref{claim:few_atomic_strings}. Below we assume that $d < n$. Recall that we do not account for the time used during the preprocessing step (but one can note that it is polynomial in $d$ and the total length of the atomic strings of $R$). The information computed during this step, including the graphs $G(P)$ for each $P \in \Pi$, takes $\Oh(d^3 \log^5 n)$ space. 

During the main step, we use Lemma~\ref{lm:restore_witnesses} to maintain the compact representations of the streaks of $\Delta(P)$ for each $P \in \Pi$, and to decide, eventually, whether $T$ matches the regular expression $R$. Whenever an instance of the pattern matching algorithm detects an occurrence of $\Delta(P)$, we decide in constant time whether this occurrence extends the latest streak of $\Delta(P)$ or starts a new one. If the number of streaks becomes equal to three, we discard the oldest streak. 

When an instance of the pattern matching algorithm detects $r \in \occ(P,T)$ for some $P \in \Pi$, we must decide whether $r$ is a witness and whether we must store it in the compact representation of the streaks containing~$r$. We apply Lemma~\ref{lm:restore_witnesses} to decide whether $r$ is a witness in $\Oh(n d^4 \polylog n)$ total time and $\Oh(d^3\polylog n)$ space and then in $\Oh(d \log n)$ time whether $r$ must be added to the compact representations of the streaks containing~$r$. Note that a position $r$ can belong to $\occ(P,T)$ for $\Oh(d\log n)$ canonical prefixes $P \in \Pi$, and therefore in the worst case we spend $\Oh(n d^5 \polylog n)$ to process $r$. The compact representations of the streaks take $\Oh(d^2 \log^3 n)$ space. 

Recall that $\mathcal{F}$ contains all atomic strings $A$ such that there is an $\eps$-transitions path from the endpoint of the transition labelled by $A$ to the final state of $T_C(R)$. 
In the regular expression pattern matching problem, we report all positions $r$ such that $r$ is a witness in $ \occ(A,T)$ for some $A\in \mathcal{F}$. In the regular expression membership problem, $T \in L(R)$ if $n$ is a witness for $\occ(A,T)$ for some $A\in \mathcal{F}$.
\end{proof}
