% !TEX root = main.tex
In this section we define the notion of witnesses that will allow us to store all occurrences (partial or not) of regular expressions detected by the algorithm efficiently. We make use of the following well-known fact:

\begin{fact}[Fine and Wilf's periodicity lemma~\cite{fw:65}]\label{fact:fine_wilf}
If a string $X$ has two periods of length~$p$ and $q$ and $p+q \le |X|$, then $X$ also has a period of length $\gcd(p,q)$.
\end{fact}

\begin{corollary}[Of Fact~\ref{fact:fine_wilf}]\label{cor:}
For a primitive string $X$ of length $x$, a string $D = XX$ can contain only two occurrences of string $X$, $D[1\dd x]$ and $D[x+1\dd 2x]$.
\end{corollary}

\subsection{Anchors}
\begin{definition}[Anchors]\label{def:anchors}
Consider a periodic string $W$ with period $\rho$. Set the anchor $a_0 = 2\rho$ and the period $\rho_0 = \rho$. Suppose that $(a_r, \rho_r)_{r = 0}^{q-1}$ are defined. We define $(a_{q}, \rho_{q})$ recursively. Consider the set $S$ of fragments $W[i \dd j]$ of $W$ satisfying the following properties:
\begin{enumerate}
\item $i \le a_{q-1} \le j$ and $W[i \dd j]$ is periodic with period $\pi < \rho_{q-1}$;
\item $i+4\pi-1 \le a_{q-1}$ (there are at least four repetitions of the string period of $W[i \dd j]$ before $a_{q-1}$);
\item $a_{q-1} \le j-4\pi-r$, where $r = (j-i+1) \pmod \pi$ (there are at least four repetitions of the string period of $W[i \dd j]$ after $a_{q-1}$).
\end{enumerate}
Let $i_q = \min\{i : W[i\dd j] \in S\}$ and $j_q = \max\{j : W[i_q\dd j] \in S\}$. If $W[i_{q} \dd j_{q}]$ is undefined, recursion stops. Otherwise, $\rho_q$ is defined to be the period of $W[i_q \dd j_q]$ and $a_{q} := i_{q}+2\rho_{q}-1$. 

Let $\A(W) = \{a_0, a_1, \ldots, a_{Q}\}$, where $(a_Q,\rho_Q)$ is the last defined anchor-period pair. We call $\A$ the \emph{generator set of anchors} of $W$. Define $\A^\ast(W) = \left[\bigcup_{\Delta \in \mathbb{Z}_+} (\A(W) + \Delta \rho)\right] \cap [1, |W|]$. We refer to $\A^\ast$ simply as the \emph{set of anchors} of $W$. 
\end{definition}

(In the next section we will slightly abuse notation and extend the notion of the set of anchors to infinite strings in a natural way, i.e. for an infinite string $W$ with period $\rho$ the set $\A^\ast(W) = \left[\bigcup_{\Delta \in \mathbb{Z}_+} (\A(W) + \Delta \rho)\right]$.) 
We first show that the generator set of anchors has logarithmic size:

\begin{lemma}\label{lm:anchors}
Let $W$ be a periodic string with period $\rho$. We have $|\A(W)| = \Oh(\log \rho)$. 
\end{lemma}
\begin{proof}
Let $\A(W) = \{a_0, a_1, \ldots, a_Q\}$. For each $0 \le q \le Q$, let $W[i_q \dd j_q]$ be the fragment associated with $a_q$, and $\rho_q$ be its period. (In particular, for $q = 0$ we have $i_0 = 1$, $j_0 = |W|$, $\rho_0 = \rho$.) We show that for each $1 \le q \le Q$ we have $|W[i_q \dd j_q]| \le 2 \rho_{q-1}$. This implies, in particular, that $|W[i_q\dd a_{q-1}]| \le  2 \rho_{q-1}$ and therefore $\rho_q \leq \rho_{q-1} / 2$. The lemma follows immediately. 

Fix $0 \le q \le Q$. Assume by contradiction that $|W[i_q \dd j_q]| > 2 \rho_{q-1}$. We then have that $W[i_q \dd j_q]$ has periods $\rho_{q-1}$ and $\rho_q$ and $\rho_{q-1}+\rho_q < |W[i_q \dd j_q]|$. Hence, $W[i_q \dd j_q]$ is periodic with period $\pi = \gcd(\rho_{q-1},\rho_q)$ by Fact~\ref{fact:fine_wilf}. The substring $W[i_q \dd j_q]$ contains a full copy of $W[i_{q-1} \dd i_{q-1}+\rho_{q-1}-1]$. Therefore, $W[i_{q-1} \dd i_{q-1}+\rho_{q-1}-1]$ has a period $\pi$, which implies that it equals $(W[i_{q-1} \dd i_{q-1}+\pi-1])^{\rho_{q-1}/\pi}$, i.e. the string period of $W[i_{q-1} \dd j_{q-1}]$ is not primitive, a contradiction. 
\end{proof}


\begin{definition}
Let $W$ be a periodic string with period $\rho$. We say that a fragment $F = W[i \dd j]$ is anchored by an anchor $a \in \A^\ast(W)$ if $i \leq a \leq j$ and for any strings $U \in \Sigma^\ast, V \in \Sigma^\rho$ such that $V \neq W[1 \dd \rho]$ there are at most eight occurrences of $F$ in $UV(W[1\dd j])$ containing the anchor (i.e., containing $|U|+|V|+a$). 
\end{definition}

\begin{lemma}\label{lm:anchors_catch_partial_match}
Let $W$ be a periodic string with period $\rho$. Consider a fragment $W[\ell \dd r]$ of length at least $4 \rho$ and a partitioning $W[\ell \dd r] = W[\ell_0 \dd r_0] W[\ell_1 \dd r_1] \ldots W[\ell_k \dd r_k]$. 
\begin{enumerate}[label=\textrm{(\alph*)}]
\item There exists $0 \le k' \le k$ such that $W[\ell_{k'} \dd r_{k'}]$ is anchored by an anchor $a \in \A^\ast(W) \cap [r-4\rho+1, r]$.\label{it:anchor_gen}
\item If, in addition, $1 \le \ell \le 2 \rho$, there exists $0 \le k' \le k$ such that $W[\ell_{k'} \dd r_{k'}]$ is anchored by an anchor $a \in \A^\ast(W) \cap [\ell, \ell + 4\rho-1]$.\label{it:anchor_beg}
\end{enumerate}
\end{lemma}
\begin{proof}
The high-level idea of the proof of~\ref{it:anchor_gen} and~\ref{it:anchor_beg} is as follows. Let $\A(W)= \{a_0,a_1, \ldots a_Q \}$. For every $0 \le q \le Q$ and an integer $\Delta > 0$ to be determined later, define $a_q^\Delta := a_q +\Delta \rho$. Let $F_q=W[\ell_{k_q},r_{k_q}]$ be the fragment that contains $a_q^{\Delta}$. We show that either $F_q$ is anchored by $a_q^{\Delta}$ or $q < Q$, which yields the lemma. We exploit two auxiliary claims:

\begin{claim}\label{claim:many_reps}
Let $\pi$ be the period of $F_q$, $0 \le q \le Q$. If $F_q$ is not anchored by $a_q^\Delta$, then there exist $U \in \Sigma^\ast, V \in \Sigma^\rho$ with $V \neq W[1 \dd \rho]$ such that there is a fragment $S[p \dd t]$ of the string $S = UV(W[\dd r_q])$ satisfying the following properties:
\begin{enumerate}
\item $p \le |U|+|V|+a_q^\Delta \le t$ (the fragment contains the anchor);
\item $S[p \dd t]$ is periodic with period $\pi$;
\item $p+7\cdot \pi \le |U|+|V|+a_q^\Delta$ (there are at least seven repetitions of the string period of the fragment before the anchor);
\item $|U|+|V|+a_q^\Delta \le t-6\pi-r$, where $r = t-p+1 \pmod \pi$ (there are at least six repetitions of the string period of the fragment after the anchor). 
\end{enumerate}
\end{claim}
\begin{proof}
Let $U \in \Sigma^\ast, V \in \Sigma^\rho$ with $V \neq W[1 \dd \rho]$ such that there are least eight occurrences of $F_q$ in $S = UV(W[1 \dd r_q])$ containing $a := |U|+|V|+a_q^\Delta$. Let the last eight occurrences be $S[p_k \dd t_k]$, $1 \le k \le 8$. As all occurrences contain $a$, the length of $S[p_1 \dd t_8]$ is at most $2|F_q|$. As a corollary from Fact~\ref{fact:fine_wilf}, we obtain that $S[p_1 \dd t_8]$ is periodic with period $\pi$, $p_k = p_1 + (k-1) \pi$, and $t_k = t_1 + (k-1) \pi$. As $p_8 = p_1+7\pi \le a$,  we have $S[p_1 \dd a] \ge 7 \pi$. On the other hand, $t_2-(|F|\pmod \pi)+1 > t_1 \ge a$. Therefore, $S[a+1 \dd t_8] \ge S[t_2-r+1 \dd t_8] \ge 6\pi$. By taking $p = p_1$ and $t = t_8$, we obtain the claim. 
\end{proof}

\begin{claim}\label{claim:next_level}
Assume that $0 \le q$ and that the period of $F_q$ is $\pi < \rho_q$. If $F_q$ is not anchored by $a_q^\Delta$, then $q < Q$. 
\end{claim}
\begin{proof}
By Claim~\ref{claim:many_reps}, there exist $U \in \Sigma^\ast, V \in \Sigma^\rho$ with $V \neq W[1 \dd \rho]$ such that there is a fragment $S[p \dd t]$ of the string $S = UV(W[\dd r_q])$ periodic with period $\pi$ that contains $a: = |U|+|V|+a_q^\Delta$ and such that there are at least six repetitions of the string period before and after $a$.

Let us show that $a-2\rho_q+1 \le p < t \le a+2\rho_q$. Suppose by contradiction that $p < a-2\rho_q+1$. We have that $S[a-2\rho_q+1 \dd a] = W[a_q^\Delta-2\rho_q+1 \dd a_q^\Delta]$ (note that by definition $a_q^\Delta \ge 2 \rho_q$ for any $\Delta$). Furthermore, $W[a_q^\Delta-2\rho_q+1 \dd a_q^\Delta]$ has periods $\rho_q$ (by definition of $i_q$ and $a_q^\Delta$) and $\pi$ (by the assumption). By $Fact~\ref{fact:fine_wilf}$, $W[a_q^\Delta-2\rho_q+1 \dd a_q^\Delta]$ has a period $\gcd(\rho_q, \pi)$. As $W[a_q^\Delta-2\rho_q+1 \dd a_q^\Delta]$ contains a full copy of the string period of $W[i_q \dd j_q]$, we obtain that it is not primitive, a contradiction. To show that $t \le a+2\rho_q$, note that $S[a+1 \dd a+2\rho_q] = W[a_q^\Delta+1 \dd a_q^\Delta+2\rho_q]$, $a_0^\Delta+2\rho \le |W|$ and, for $q\ge 1$, $a_q^\Delta+2\rho_q \le a_{q-1}^\Delta$ by definition of $i_q$ and $a_q$. Therefore, for all $q \ge 0$, $W[a_q^\Delta+1 \dd a_q^\Delta+2\rho_q]$ is periodic with period $\rho_q$. The rest of the argument is analogous. 

From $a-2\rho_q+1 \le p < t \le a+2\rho_q$ and the fact that $S[p \dd t]$ contains at least six repetitions of its string period before and after $a$, we obtain that $i_{q+1}, j_{q+1}$ and hence $a_{q+1}, \rho_{q+1}$ are well-defined, which completes the proof of the claim.  
\end{proof}

We are now ready to show~\ref{it:anchor_gen} and~\ref{it:anchor_beg}. 

\begin{enumerate}[label=\textrm{(\alph*)}]
\item Let $\Delta \le \lfloor |W|/\rho\rfloor-2$ be the smallest integer such that $a_0+\Delta \rho \ge r-4\rho$. Note that $\Delta$ is well-defined. 
 Consider an anchor $a_0^\Delta = a_0+\Delta \rho \in \A^\ast(W) \cap [r-4\rho+1, r]$. Let $F_0 = W[\ell_{k_0} \dd r_{k_0}]$ be the fragment that contains $a_0^\Delta$ and let $\pi$ be its period. If $F_0$ is anchored by $a_0^\Delta$, we are done. Otherwise, by Claim~\ref{claim:many_reps}, there exist $U \in \Sigma^\ast, V \in \Sigma^\rho$ with $V \neq W[1 \dd \rho]$ such that there is a fragment $S[p \dd t]$ of the string $S = UV(W[\dd r_{k_0}])$ periodic with period $\pi$ that contains $a: = |U|+|V|+a_0^\Delta$ and such that there are at least six repetitions of the string period of $F_0$ before and after $a$. As we have $r_{k_0}-a_0^\Delta+1 \le r-a_0^\Delta+1 \le 3\rho$, there is $\pi \le \rho/2$. It follows that $i_1, j_1$ and hence $a_1, \rho_1$ are well-defined, i.e. $0 < Q$. 

We now show that for arbitrary $q \ge 1$ either $a_q^\Delta = a_q + \Delta \rho$ anchors $F_q$ or the period $\pi$ of $F_q$ is smaller than $\rho_q$, which by Claim~\ref{claim:next_level} implies that $q < Q$. Importantly, by our choice of $\Delta$, $F_q$ is well-defined.  If $F_q$ is anchored by $a_q^\Delta$, we are done.  

Otherwise, by Claim~\ref{claim:many_reps}, there exist $U \in \Sigma^\ast, V \in \Sigma^\rho$ with $V \neq W[1 \dd \rho]$ such that there is a fragment $S[p \dd t]$ of the string $S = UV(W[\dd r_{k_q}])$ periodic with period $\pi$ that contains $|U|+|V|+a_q^\Delta$ and such that there are at least six repetitions of the string period $F_q[1\dd \pi]$ of $F_q$ before and after $|U|+|V|+a_q^\Delta$. Recall that $a_q^\Delta = (i_q+2\rho_q) + \Delta \rho$. First, we have that $\pi \neq \rho_q$, otherwise we could have extended $W[i_q \dd j_q]$ to the left. Second, let us show that the case $\pi > \rho_q$ is impossible. Suppose otherwise. If $t-4\pi+1 \le |U|+|V|+a_{q-1}^\Delta$, then $W[a_q^\Delta+1 \dd a_{q-1}^\Delta]$ contains $F_q[1\dd 2\pi]$, and therefore $F_q[1\dd 2\pi]$ has a period $\rho_q$. By Fact~\ref{fact:fine_wilf}, the string period $F_q[1\dd \pi]$ of $F_q$ is not primitive, a contradiction. Otherwise, $a_{q-1}^\Delta$ is contained in $W[p-(|U|+|V|) \dd t-(|U|+|V|)]$, which has period $\pi < \rho_{q-1}$. In addition, there are at least four repetitions of the string period of $W[p-(|U|+|V|) \dd t-(|U|+|V|)]$ before and after $a_{q-1}^\Delta$, and $p-(|U|+|V|) < a_q^\Delta < i_q$\todog{$p-(|U|+|V|) < a_q^\Delta - 2\rho_q = i_q$ ?}. A contradiction with the choice of $W[i_q \dd j_q]$. It finally follows that $\pi < \rho_q$ and therefore by Claim~\ref{claim:next_level}, $q < Q$.


\item Let $\Delta \le \lfloor |W|/\rho\rfloor-2$ be the smallest integer such that $a_0+(\Delta-2) \rho \ge \ell$. (Note that $\Delta$ is well defined and $1 \le \Delta \le 2$). Consider an anchor $a_0^\Delta = a_0+\Delta \rho \in A^\ast(W) \cap [\ell, \ell + 4\rho-1]$. We first show that either $F_0$ is caught by $a_0^\Delta$, or $0 < Q$. If $F_0$ is anchored by $a_0^\Delta$, we are done. Otherwise, let $\pi$ be the period of $F_0$. We claim that $\pi < \rho = \rho_0$. As $F_0$ is not anchored, by Claim~\ref{claim:many_reps} there exist $U \in \Sigma^\ast, V \in \Sigma^\rho$ with $V \neq W[1 \dd \rho]$ such that there is a fragment $S[p \dd t]$ of the string $S = UV(W[\dd r_{k_0}])$ periodic with period $\pi$ that contains $a: = |U|+|V|+a_0^\Delta$ and such that there are at least six repetitions of the string period of $F_0$ before and after $a$. We can immediately rule out the case $\pi = \rho$ as $a \leq 4\rho$ and $V \neq W[1 \dd \rho]$. Consider now the case $\pi > \rho_0$. Consider the suffix $S[a+1\dd] = W[a_0^\Delta+1 \dd r_{k_0}]$. It contains an occurrence of $F_0[1\dd 2 \pi]$. By Fact~\ref{fact:fine_wilf}, $F_0[1\dd 2 \pi]$ has a period $\gcd(\pi, \rho)$, and therefore the string period of $F_0$ is not primitive, a contradiction. For $q \ge 1$, $F_q$ is well-defined by our choice of $\Delta$. 
The rest of the argument repeats the argument in the proof of~\ref{it:anchor_gen}.
\end{enumerate}
\end{proof}
