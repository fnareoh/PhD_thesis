\usepackage{amsfonts,amssymb,amsmath} %symboles maths
%\usepackage{fullpage} %marges
%\usepackage[margin=1in]{geometry} %clash
\usepackage{tabularx} %depassement des tableau hors page
\usepackage{float} %position des figures
\usepackage{cite} %bibtex
\usepackage[utf8]{inputenc}
\usepackage{todonotes}
\usepackage{changes}
\usepackage{multirow}
%\usepackage{subfigure}
%\usepackage{authblk}
\usepackage{tikz} %figures
\usetikzlibrary{automata,positioning,calc}
\usetikzlibrary{decorations.pathreplacing}
%\usepackage[noend,boxed]{algorithm2e}
\usepackage{enumitem}

\usepackage{hyperref} %liens web
\hypersetup{
    colorlinks=false,
    linkcolor=blue,
    urlcolor=cyan,
}


\newcommand{\colorTatiana}{yellow}
\newcommand{\colorGarance}{magenta}
\newcommand{\colorPawel}{red!70}
\newcommand{\colorBartek}{blue!40}

\definechangesauthor[color=\colorTatiana]{TA}
\definechangesauthor[color=\colorGarance]{GA}
\definechangesauthor[color=\colorBartek]{BA}
\definechangesauthor[color=\colorPawel]{PA}

\newcommand{\todot}[1]{\todo[color=\colorTatiana]{#1}}
\newcommand{\todoinlinet}[1]{\todo[color=\colorTatiana,inline]{#1}}
\newcommand{\todog}[1]{\todo[color=\colorGarance!40]{#1}}
\newcommand{\todoinlineg}[1]{\todo[color=\colorGarance!40,inline]{#1}}
\newcommand{\todop}[1]{\todo[color=\colorPawel]{#1}}
\newcommand{\todoinlinep}[1]{\todo[color=\colorPawel,inline]{#1}}
\newcommand{\todob}[1]{\todo[color=\colorBartek]{#1}}
\newcommand{\todoinlineb}[1]{\todo[color=\colorBartek,inline]{#1}}

\newcommand{\A}{\mathcal{A}}
\newcommand{\Oh}{\mathcal{O}}
\newcommand{\Ohtilde}{\tilde{\Oh}}
\newcommand{\RLE}{\textsf{RLE}}
\newcommand{\LCP}{\textsf{LCP}}
\newcommand{\poly}{\mathrm{poly}}
\newcommand{\per}{\mathrm{per}}
\def\polylog{\operatorname{polylog}}
\newcommand{\Mod}[1]{\ (\mathrm{mod}\ #1)}

\newcommand{\ed}{\mathsf{ed}}
\newcommand{\gr}{\mathsf{GR}}
\newcommand{\ga}{\mathsf{GA}}

\newtheorem{thm}{Theorem}[section] % reset theorem numbering for each section

\newtheorem{definition}[thm]{Definition}
\newtheorem{observation}[thm]{Observation}
%\newtheorem{lemma}[thm]{Lemma}
%\newtheorem{theorem}[thm]{Theorem}
%\newtheorem{corollary}[thm]{Corollary}
\newtheorem{example}[thm]{Example}
%\newtheorem{proposition}[thm]{Proposition}
%\newtheorem{fact}[thm]{Fact}
\newtheorem{claim}[thm]{Claim}

\makeatletter
\let\c@proposition\c@thm
\let\c@corollary\c@thm
\let\c@theorem\c@thm
\let\c@lemma\c@thm
\let\c@fact\c@thm
%\let\c@definition\c@thm
%\let\c@example\c@theorem
\makeatother

\renewcommand{\paragraph}[1]{\vspace{1mm}\noindent\textbf{#1}}

   \newcommand{\defproblem}[3]{
  \vspace{2mm}
\noindent\fbox{
  \begin{minipage}{0.96\textwidth}
  #1\\
  #2
  \end{minipage}
  }
  \vspace{2mm}
}

\usepackage{makecell} % NEw line in tabular cell

\renewcommand\theadalign{bc}
\renewcommand\theadfont{\bfseries}
\renewcommand\theadgape{\Gape[4pt]}
\renewcommand\cellgape{\Gape[4pt]}

%Tikzit for figures - it requires compiling with pdflatex
\usepackage{tikzit}
\input{figures/simple.tikzstyles}
\usepackage{caption}
\usepackage{subcaption}
