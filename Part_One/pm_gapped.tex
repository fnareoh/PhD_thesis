\newcommand{\inputpmgapped}[1]{\input{Part_One/pm_gapped/#1}}


%\renewcommand{\occ}[0]{\mathrm{output}}
\newcommand{\suffix}{\mathsf{suffix}}
\newcommand{\prefix}{\mathsf{prefix}}
\newcommand{\substringinfo}[1]{#1-substring information}
\newcommand{\boundaryinfo}[1]{#1-boundary information}
\newcommand{\suffixinfo}[1]{#1-suffix information}
\newcommand{\prefixinfo}[1]{#1-prefix information}

\contextbox{Publication}{
This chapter corresponds to the following publication:~\fullcite{}
}


\small{
    Originating from the work of Navarro and Thankanchan [TCS 2016], the problem of consecutive pattern matching is a variant of the fundamental pattern matching problem, where one is given a text and a pair of patterns $p_1,p_2$, and must compute consecutive occurrences of $p_1,p_2$ in the text. Assuming that the text is given as a straight-line program of size $g$, we develop an algorithm that computes all consecutive occurrences of $p_{1}, p_{2}$ in optimal $O(g+|p_1|+|p_2|+\occ)$ time. As a corollary, we also derive an algorithm that reports all co-occurrences separated by a distance $d \in [a,b]$ in $O(g+|p_1|+|p_2|+\occ)$ time and an algorithm that reports the top-$k$ closest co-occurrences in $O(g+|p_1|+|p_2|+k)$ time.
}

\inputpmgapped{intro}
\inputpmgapped{prelim}
\inputpmgapped{boundary}
\inputpmgapped{crossing}
\inputpmgapped{gapped}
\inputpmgapped{variants}
