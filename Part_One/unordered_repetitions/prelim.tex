\section{Preliminaries}
\label{sec:prelim}

\paragraph{Strings.}
A string of length $n$ is a sequence $T[1] \dots T[n]$ of characters from a finite alphabet $\Sigma$ of size $\sigma$. The substring $T[i..j]$ is the string $T[i] \cdots T[j]$, whereas the fragment $T[i..j]$ refers to the specific occurrence of $T[i..j]$ starting at position $i$ in $T$. If $i > j$, then $T[i..j]$ is the empty string.
A suffix of $T$ has the form $T[i..n]$.
We say that a fragment $T[i'..j']$ is properly contained in another fragment $T[i..j]$ if $i < i' \leq j' < j$.
A substring is properly contained in $T[i..j]$, if it equals a fragment that is properly contained in $T[i..j]$.
We write $T[i..j)$ a shortcut for $T[i..j-1]$. Similarly, we write $[i, j] = [i, j + 1)$ as a shortcut for the integer interval $\{i, \dots, j\}$. Given two positions $i \leq j$, their longest common extension (LCE) is the length of the longest common prefix between suffixes $T[i..n]$ and $T[j..n]$, formally defined as $\lce(i,j) = \lce(j,i) = \max\{\ell \in \{0, \dots, n - j + 1\} \mid T[i..i+\ell) = T[j..j+\ell)\}.$

\begin{definition}
A positive integer $p$ is a period of a string $T[1 .. n]$ if $T[i] = T[i+p]$ for every $i\in \{1, \dots n-p \}$. The smallest such $p$ is called the period of $T[1..n]$, and we call a string periodic if its period $p$ is at most $\frac{n}{2}$.
\end{definition}

\paragraph{Computational model.}
For a general unordered alphabet $\Sigma$, the only allowed operation on the characters is comparing for equality. In particular, there is no
linear order on the alphabet. Unless explicitly stated otherwise, we will only use such comparisons. A general ordered alphabet has a total order, such that comparisons of the type less-equals are possible. 

In the algorithmic part of the paper, we assume the standard unit-cost Word RAM model with words of length $\Omega(\log n)$,
but the algorithm is only allowed to access the input string $T[1..n]$ by comparisons $T[i]\stackrel{?}{=}T[j]$, which are assumed to take constant time.
We say that a string of length $n$ is over a linearly-sortable alphabet, if we can sort the $n$ symbols of the string in $\Oh(n)$ time. Note that whether or not an alphabet is linearly-sortable depends not only on the alphabet, but also on the string. For example, the alphabet $\Sigma=\{1,\dots,m^{\Oh(1)}\}$ is linearly-sortable for strings of length $n = \Omega(m)$ (e.g., using radix sort), but it is unknown whether it is linearly-sortable for all strings of length $n = o(m)$~\cite{HanT02}.
Our algorithm will internally use strings over linearly-sortable alphabets. We stress that in such strings the characters are not the
characters from the input string, but simply integers calculated by the algorithm.
Note that every linearly-sortable alphabet is also a general ordered alphabet.

\paragraph{Squares and runs.}
A square is a length-$2\ell$ fragment of period $\ell$. 
The following theorem is a classical result by Main and Lorentz~\cite{Main1984}.

\begin{theorem}
\label{lem:classical}
Testing square-freenes of $T[1..n]$ over a general alphabet can be implemented in $\Oh(n\log n)$ time and comparisons.
\end{theorem}

\noindent The proof of the above theorem is based on running a divide-and-conquer procedure
using the following lemma.

\begin{lemma}
\label{lem:conquer}
Given two strings $x$ and $y$ over a general alphabet, we can test if there is a square in $xy$ that is not fully contained in $x$ nor $y$ in $\Oh(|x|+|y|)$
time and comparisons.
\end{lemma}


A repetition is a length-$\ell$ fragment of period at most $\frac \ell 2$. 
A run is a maximal repetition. 
Formally, a repetition in $T[1..n]$ is a triple $\tuple{s, e, p}$ with $s,e \in [1, n]$ and $p \in [1, \frac{e - s + 1}{2}]$ such that $p$ is the smallest period of $T[s..e]$. 
A run is a repetition $\tuple{s, e, p}$ that cannot be extended to the left nor to the right with the same period, in other words
$s = 1$ or $T[s-1]\neq T[s-1+p]$ and $e = n$ or $T[e+1] \neq T[e+1-p]$.
The celebrated runs conjecture, proven by Bannai et al.~\cite{Bannai2017}, states
that the number of runs is any length-$n$ string is less than $n$.
Ellert and Fischer~\cite{Ellert2021} showed that all runs in a string over a general ordered alphabet can be computed in $\Oh(n)$ time.
As mentioned earlier, each run contains a square, and each square is contained in a run.
Thus, the string contains a square if and only if it contains a run, and it follows:
\begin{theorem}
\label{lem:fasterclassical}
Computing all runs (and thus testing square-freeness) of $T[1..n]$ over a general ordered alphabet can be implemented in $\Oh(n)$ time.
\end{theorem}

\paragraph{Lempel-Ziv factorisation.}
The unique LZ phrase starting at position $s$ of $T[1..n]$ is a fragment $T[s..e]$ such that $T[s..(e-1)]$ occurs at least twice in $T[1..(e-1)]$ and either $e = n$ or $T[s..e]$ occurs only once in $T[1..e]$.
The Lempel-Ziv factorisation of $T$ consists of $z$ phrases $f_{1},\ldots,f_{z}$ such that the concatenation $f_{1}\ldots f_{z}$ is equal to $T[1..n]$ and each $f_{i}$ is the unique LZ phrase starting at position $1 + \sum_{j=1}^{i - 1}\absolute{f_{j}}$.

\paragraph{Tries.}
Given a collection $\mathcal S = \{T_1, \dots, T_k\}$ of strings over some alphabet $\Sigma$, its trie is a rooted tree with edge labels from $\Sigma$. 
For any node $v$, the concatenation of the edge labels from the root to the node \emph{spells} a string. 
The string-depth of a node is the length of the string that it spells. 
No two nodes spell the same string, i.e., for any node, the labels of the edges to its children are pairwise distinct. Each leaf spells one of the $T_i$, and each $T_i$ is spelled by either an internal node or a leaf.

The compacted trie of $\mathcal S$ can be obtained from its (non-compacted) trie by contracting each path between a leaf or a branching node and its closest branching ancestor into a single edge (i.e., by contraction we eliminate all non-branching internal nodes). The label of the new edge is the concatenation of the edge labels of the contracted path in root to leaf direction. Since there are at most $k$ leaves and all internal nodes are branching, there are $O(k)$ nodes in the compacted trie. Each edge label is some substring $T_i[s..e]$ of the string collection, and we can avoid explicitly storing the label by instead storing the reference $(i,s,e)$.  Thus $O(k)$ words are sufficient for storing the compacted trie. Consider a string $T'$ that is spelled by a node of the non-compacted trie. We say that $T'$ is explicit, if and only if it is spelled by a node of the compacted trie. 
Otherwise $T'$ is implicit.

The suffix tree of a string $T[1..n]$ is the compacted trie containing exactly its suffixes, i.e., a trie over the string collection $\{T[i..n] \mid i \in \{1, \dots, n\}\}$. It is one of the most fundamental data structures in string algorithmics, and is widely used, e.g., for compression and indexing~\cite{Gusfield1997}. The suffix tree can be stored in $\Oh(n)$ words of memory, and for linearly-sortable alphabets it can be computed in $\Oh(n)$ time~\cite{Farach1997}.
The sparse suffix tree of $T$ for some set $B \subseteq \{1, \dots, n\}$ of sample positions is the compacted trie containing exactly the suffixes $\{T[i..n] \mid i \in B\}$. It can be stored in $\Oh(\absolute{B})$ words of memory.

We assume that $T$ is terminated by some special symbol $T[n] = \texttt\textdollar$ that occurs nowhere else in $T$. This ensures that each suffix is spelled by a leaf, and we label the leaves with the respective starting positions of the suffixes. Note that for any two leaves $i\neq j$, their lowest common ancestor (i.e., the deepest node that is an ancestor of both $i$ and $j$) spells a string of length $\lce(i, j)$.