 
\subsection{Final Improvement}
\label{sec:finalimprov}

\inputsquare{figures/final}

For our final improvement we need to replace the LCE queries implemented by \cref{lem:LCE_undordered} with our own
mechanism.
The goal will remain the same, that is, given a parameter $\Delta$ and estimate $\tilde{\sigma}$ of the alphabet size,
find a $\Delta$-approximate LZ factorisation of any fragment $T[x..y]$ in $\Oh(m\tilde{\sigma}\log m/\sqrt{\Delta})$ time, where $m=|T[x..y]|$ (with $m=\Theta(\Delta^2)$, as otherwise we are not required to detect anything).

As in the previous section, the algorithm might detect that the size of the alphabet is larger
than $\tilde{\sigma}$, and in such case we revert to the divide-and-conquer algorithm. Let $\tau=\lfloor\sqrt{\Delta}\rfloor$.

Initially, we only consider some fragments of $T[x..y]$. We say that $T[i\cdot \tau^{2}..i\cdot \tau^{2} + \tau)$ is a dense fragment.
We start by remapping the characters in all dense fragments that intersect $T[x..y]$ to a linearly-sortable alphabet. This can be done
in $\Oh(\tilde{\sigma})$ time for each position by maintaining a list of the already seen distinct characters. For each position in a dense fragment,
we iterate over the characters in the list, and possibly append a new character to the list if it is not present. As soon as the size of
the list exceeds $\tilde{\sigma}$, we terminate the procedure and revert to the divide-and-conquer algorithm. Otherwise, we replace each character by its position in the list.
Overall, there are $\Oh(m/\sqrt{\Delta})$ positions in the dense fragments of $T[x..y]$, and the remapping takes $\Oh(m\tilde{\sigma}/\sqrt{\Delta})$ time.

Next, we construct two generalised suffix trees~\cite{Gusfield1997}, the first one of all dense fragments, and the second one of their reversals. 
(The generalised suffix tree of a collection of strings is the compacted trie that contains all suffixes of all strings in the collection.)
Again, because we
now work with a linearly-sortable alphabet this takes only $\Oh(m/\sqrt{\Delta})$ time~\cite{Farach1997}. We consider fragments
of the form $T[i\cdot \tau .. (i+1)\cdot \tau)$ having non-empty intersection with $T[x..y]$. We call such fragments chunks.
We note that there are $\Oh(m/\sqrt{\Delta})$ chunks, and their total length is $\Oh(m)$.
For each chunk, we find its longest prefix $T[i\cdot \tau.. i\cdot \tau+\ell_{i})$ and
longest suffix $T[(i+1)\cdot \tau-r_{i} .. (i+1)\cdot \tau)$ that occur in one of the dense fragments.
\cref{fig:improv:1} visualizes the dense fragments, chunks, and longest prefixes and suffixes.
This can be done efficiently by following the heavy path decomposition of the generalised suffix tree of all dense fragments
and their reversals, respectively. On each current heavy path, we just naively match the characters as long as possible.
In case
of a mismatch, we spend $\Oh(\tilde{\sigma})$ time to descend to the appropriate subtree, which happens at most $\Oh(\log m)$ times due to the heavy path decomposition.
After having found $\ell_{i}$ and $r_{i}$, we test square-freenes of $T[i\cdot \tau.. i\cdot \tau+\ell_{i})$
and $T[(i+1)\cdot \tau-r_{i} .. (i+1)\cdot \tau)$. Because they both occur in dense fragments,
and we have remapped the alphabet of all dense fragments, we can use \cref{lem:fasterclassical} to implement
this in $\Oh(\ell_{i}+r_{i})$ time.
Thus, the total time per chunk is thus $\Oh(\tilde{\sigma}\log m)$ plus $\Oh(\ell_{i}+r_{i})$. The former sums up to
$\Oh(m\tilde{\sigma}\log m/\sqrt{\Delta})$, and we will later show that the latter can be amortised by deactivating blocks on the lower levels.

The situation so far is that we have remapped the alphabet of all dense fragments to linearly-sortable, and for every chunk we know its longest
prefix and suffix that occur in one of the dense fragments. We concatenate 
all fragments of the form $T[i\cdot \tau-r_{i-1} .. i\cdot \tau+\ell_{i})$ (intersected with $T[x..y]$) while adding distinct separators
in between to form a new string $T'$. 
We stress that, because we have remapped the alphabet of all dense fragments, and the found
longest prefix and suffix of each chunk also occur in some dense fragment,  $T'$ is over linearly-sortable alphabet. 
Thus, we can build the suffix tree $ST$ of $T'$ in $\Oh(|T'|)$ time~\cite{Farach1997}. 
A visualization of $T'$ is provided in \cref{fig:improv:2}

Let $\mathcal{D}=\{D_{1},D_{2},\ldots\}$ be the set of distinct dense fragments.
We would like to construct the set of all occurrences of the strings in $\mathcal{D}$ in $T[x..y]$.
Using the suffix tree of $T'$ we can retrieve all occurrences
of every $D_{j}$ in $T'$. We observe that, because of how we have defined $T[i\cdot \tau.. i\cdot \tau+\ell_{i})$ and
$T[(i+1)\cdot \tau-r_{i} .. (i+1)\cdot \tau)$, this will in fact give us all occurrences of every $D_{j}$ in the original $T[x..y]$.
To implement this efficiently, we proceed as follows. First, for every $i$ we traverse $ST$ starting from its root to find the (explicit or implicit)
node corresponding to the dense fragment $T[i\cdot \tau^{2}.. i\cdot \tau^{2} + \tau)$.
This takes only $\Oh(m\tilde{\sigma}/\sqrt{\Delta})$ time.
Then, all leaves in every subtree rooted at such a node correspond to occurrences of some $D_{j}$, and can be
reported by traversing the subtree in time proportional to its size, so at most $\Oh(|T'|)$ in total.
Finally, remapping the occurrences back to $T[x..y]$ can be done in constant time per occurrence by precomputing,
for every position in $T'$, its corresponding position in $T[x..y$], which can be done in $\Oh(|T'|)$ time when
constructing $T'$. Thus, in $\Oh(|T'|)$ time, we obtain the set $S$ of starting positions of all occurrences
of the strings in $\mathcal{D}$. We summarize the properties of $S$ below.

\begin{proposition}
$S$ admits the following properties:
\begin{enumerate}
\item For every $i\in [x,y]$ such that $i = 0 \pmod {\tau^{2}}$, $i\in S$.
\item For every $i\in [x,y-\tau]$, $i\in S$ if and only if $T[i..i+\tau) \in \mathcal{D}$.
\item $|S| \leq |T'|$.
\end{enumerate}
\end{proposition}

We now define a parsing of $T[x..y]\$ $ based on $S$.
Let $i_{1}<i_{2}<\ldots i_{k}$ be all the positions in $S$, that is, $(i_{j},i_{j+1}) \cap S = \emptyset $ for every $j=1,2,\ldots,k-1$.
For every $j=1,2,\ldots,k-1$, we create the phrase $T[i_{j}.. i_{j+1}+\tau)$. We add the last phrase $T[i_{k}.. y]\$$. We stress that
consecutive phrases overlap by $\tau$ characters, and each phrase begins with a length-$\tau$ fragment starting at a position in $S$.
This, together with property 2 of $S$, implies the following property.

\begin{observation}
\label{obs:prefixfree}
The set of distinct phrases is prefix-free.
\end{observation}

We would like to construct the compacted trie $\Tphrase$ of all such phrases, so that (in particular) we identify identical
phrases. 
We first notice that each phrase begins with a fragment $T[i_{j}..i_{j}+\tau)$ that has its corresponding
occurrence in $T'$. We note that, given a set of positions $P$ in $T$, we can find their corresponding
positions in $T'$ (if they exist) by sorting and scanning in $\Oh(|P|+|T'|)$ time.



Thus, we can
assume that for each $i_{j}$ we know its corresponding position $i'_{j}$ in $T'$.
Next, for each node of $ST$ we precompute its unique ancestor at string depth $\tau$ in $\Oh(|T'|)$ time.
Then, for every fragment $T[i_{j}..i_{j}+\tau)$ we can access its corresponding (implicit or explicit)
node of $ST$. This allows us to partition all phrases according to their prefixes of length $\tau$.
In fact, this gives us the top part of $\Tphrase$ containing all such prefixes in $\Oh(m/\sqrt{\Delta})$ time,
and for each phrase we can assume that we know the node of $\Tphrase$ corresponding to its length-$\tau$
prefix. 

To build the remaining part of $\Tphrase$, we partition the phrases into short and long.
$T[i_{j}.. i_{j+1}+\tau)$ is short when $i_{j+1} \leq i_{j}+\tau$ (meaning that its length is at most $2\tau$), and long otherwise.

We begin with constructing the compacted trie $\Tphrase'$ of all short phrases.
This can be done similarly to constructing the top part of $\Tphrase$, except that now
the fragments have possibly different lengths. However, every short phrase $T[i_{j}..i_{j+1}+\tau)$
occurs in $T'$ as $T'[i'_{j}..i'_{j+1}+\tau)$. We claim that the nodes of $ST$ corresponding
to every $T'[i'_{j}..i'_{j+1}+\tau)$ can be found in $\Oh(|T'|)$ time. This can be done
by traversing $ST$ in the depth-first order while maintaining a stack of all explicit nodes with string depth at least $\tau$
on the current path. Then, when visiting the leaf
corresponding to the suffix of $T'$ starting at position $i'_{j}$, we iterate over the current
stack to find the sought node. This takes at most $\Oh(|T'[i_{j}+\tau..i_{j+1}+\tau]|)$ time,
which sums up to $\Oh(|T'|)$. Having found the node of $ST$ corresponding to $T[i_{j}..i_{j+1}+\tau)$,
we extract $\Tphrase'$ from $ST$ in $\Oh(|T'|)$ time.

With $\Tphrase'$ in hand, we construct the whole $\Tphrase$ as follows. We begin with taking the union of $\Tphrase'$ and
the already obtained top part of $\Tphrase$, this can be obtained in $\Oh(|T'|)$ time. For each long
phrase $T[i_{j}.. i_{j+1}+\tau)$, we know the node corresponding to $T[i_{j}..i_{j}+\tau)$
and would like to insert the whole string $T[i_{j}.. i_{j+1}+\tau)$ into $\Tphrase$.
We perform the insertions in increasing order of $i_j$ (this will be crucial for amortising the time later).
This is implemented with a dynamic heavy path decomposition similarly as in \cref{sec:suffixtree},
however with one important change. Namely, we fix a heavy path decomposition of the
part of $\Tphrase$ corresponding to the union of $\Tphrase'$ and the top part of $\Tphrase$, and maintain a dynamic heavy
path decomposition of every subtree hanging off from this part. Thanks to this change,
the time to maintain the dynamic trie and all heavy path decompositions is $\Oh(m\log m/\sqrt{\Delta})$,
as there are only $\Oh(m/\sqrt{\Delta})$ long phrases. Next, for each long phrase
$T[i_{j}..i_{j + 1}+\tau)$, we begin the insertion at the already known node corresponding to $T[i_{j}..i_j+\tau)$,
and continue the insertion by following the heavy paths, first in the static heavy path decomposition
in the part of $\Tphrase$ corresponding to $\Tphrase'$, second in the dynamic heavy path
decomposition in the appropriate subtree.
On each heavy path, we naively match the characters as long as possible.
The time to insert a single phrase $T[i_{j}.. i_{j+1}+\tau)$ is $\Oh(\log m)$ (twice)
plus the length of the longest prefix
of $T[i_{j}+\tau.. i_{j+1}+\tau)$ equal to a prefix of $T[i_{j'}+\tau.. i_{j'+1}+\tau)$, for some $j'<j$.
The former sums up to another $\Oh(m\log m/\sqrt{\Delta})$, and we will
later show that the latter can be amortised by deactivating blocks on the lower levels.

$\Tphrase$ allows us to form metacharacters corresponding to the phrases, and transform $T[x..y]$ into a string $\Tparse$
of length $\Oh(|T'|)$ consisting of these metacharacters.
We build a suffix tree $\Sparse$ over this
string over linearly-sortable metacharacters in $\Oh(|T'|)$ time. Next, we convert it into the sparse suffix tree
$\Sparse'$ of all suffixes $T[i_{j}..y]$ as follows. Consider an explicit node $u\in\Sparse$ with children $v_{1},v_{2},\ldots,v_{d}$,
$d\geq 2$. We first compute the subtree $\mathcal{T}_{u}$ of $\Tphrase$ induced by the leaves corresponding to the first metacharacters
on the edges $(u,v_{i})$, for $i=1,2,\ldots,d$, and connect every $v_{i}$ to the appropriate leaf of $\mathcal{T}_{u}$.
This can be implemented in $\Oh(d)$ time, assuming constant-time lowest common ancestor queries on $\Tphrase$~\cite{BenderF00}
and processing the leaves from left to right with a stack, similarly as in the Cartesian tree construction algorithm~\cite{Vuillemin80}.
We note that the order on the leaves is the same as the order on the metacharacters, and hence no extra sorting is necessary.
Overall, this sums up to $\Oh(|T'|)$.
Next, we observe that, unless $u$ is the root of $\Sparse$, all metacharacters on the edges $(u,v_{i})$ correspond
to strings starting with the same prefix of length $\tau$. We obtain the subtree $\mathcal{T}_{u}'$ by truncating
this prefix (or taking $\mathcal{T}_{u}$ if $u$ is the root). Finally, we identify the root of $\mathcal{T}_{u}'$
with $u$, and every child $v_{i}$ with its corresponding leaf of $\mathcal{T}_{u}'$.
Because we truncate the overlapping prefixes of length $\tau$, after this procedure is executed on every node
of $\Sparse$ we obtain a tree $\Sparse'$ with the property that each leaf corresponds to a suffix
$T[i_{j}..y]$. Also, by \cref{obs:prefixfree}, the edges outgoing from every node start with different characters
as required.

By following an argument from the proof of \cref{lem:lz-log-star},
$\Sparse'$ allows us to determine, for every suffix $T[i_{j}..y]$, its longest prefix equal to a prefix of some $T[i'..y]$
with $i' < i_{j}$, as long as its length is at least $\tau$. Indeed, in such case we must have $i'\in S$ by property 2,
so in fact $i'=i_{j'}$ and it is enough to maximise the length of the common prefix
with all earlier positions in $S$, which can be done using $\Sparse'$. Thus, we either know that the length of this longest
prefix is less than $\tau$, or know its exact value (and the corresponding position $i'\in S$).

\begin{lemma}
\label{lem:synchro-facto}
For any parameter $\Delta \in [1,m]$ and estimate $\tilde{\sigma}$ of the alphabet size, a $(\Delta+\tau)$-approximate LZ factorisation of any
fragment $T[x..y]$ can be computed in $\Oh(m/\sqrt{\Delta})$ time with $m=|T[x..y]|$ (assuming the preprocessing
described earlier in this section).
\end{lemma}

\begin{proof}
Let $e \in [x,y]$ and suppose we have already constructed the factorisation of $T[x..e-1]$ and are now trying to construct the next phrase.
Let $e'$ be the next multiple of $\tau^{2}$, we have that $e'-e < \tau^{2}\leq \Delta$ and $T[e'.. e'+\tau)$ is a dense fragment.
Thus, by property 1 we have $e'\in S$.

The first possibility is that the longest common prefix between $T[e'..y]$ and any suffix starting at an earlier position is shorter
than $\tau$. In this case, we can simply set the head of the new phrase to be $T[e..e'+\tau)$ and the tail to be empty.
Otherwise, we know the length $\ell$ of this longest prefix by the preprocessing described above.
We set the head of the new phrase to be $T[e..e')$ and the tail to be $T[e'..e''+\ell)$. This takes constant time per phrase, and each
phrase is of length at least $\tau$, giving the claimed overall time complexity. It remains to argue correctness of every step.

Let $T[e..s]$ be the longest LZ phrase starting at position $e$, to show that we obtain a valid $(\Delta+\tau)$-approximate 
phrase it suffices to show that $s \leq  e'+ \max(\tau,\ell)$.
Let the previous occurrence of $T[e..s)$ be at position $p<e$. If $s-e' < \tau$ then there is nothing to prove.
Otherwise, $T[e'..s)$ is a string of length at least $\tau$ that also occurs starting earlier at position $p+e'-e < e'$.
Thus, we will correctly determine that $\ell \geq \tau$, and find a previous occurrence of the string maximising
the value of $\ell$. In particular, we will have $\ell \geq s-e'$ as required.
\end{proof}

To achieve the bound of \cref{thm:upperbound2}, we now proceed as in \cref{sec:improved}, except that instead
of \cref{lem:compute3} we use \cref{lem:synchro-facto}. For every $T[x..y]$ with $m=|T[x..y]|$
this takes $\Oh(m\tilde{\sigma}\log m/\sqrt{\Delta})$ time plus the time used for computing the longest
prefix and suffix of each chunk (the latter also accounts for constructing the suffix tree $ST$ and other steps that have
been estimated as taking $\Oh(|T'|)$ in the above reasoning)
plus the time for inserting $T[i_{j}+\tau.. i_{j+1}+\tau)$ into $\Tphrase$ when $i_{j+1}\geq i_{j}+\tau$.

We observe that we can deactivate any block pair fully contained in $T[i\cdot \tau.. i\cdot \tau+\ell_{i})$ and $T[(i+1)\cdot \tau-r_{i} .. (i+1)\cdot \tau)$,
as we have already checked that these fragments are square-free.
Also, we can deactivate any block pair fully contained in 
the longest prefix of $T[i_{j}+\tau.. i_{j+1}+\tau)$ equal to $T[i_{j'}+\tau.. i_{j'+1}+\tau)$, for some $j'<j$,
because such fragment cannot contain the leftmost occurrence of a square. 

There are $\Oh(m / \sqrt{\Delta})$ chunks and long phrases. If a chunk or a long phrase
contributes $x = \Omega(\sqrt[4]{\Delta})$ to the total time, then we explicitly deactivate the block pairs in phase $t + 3$
that are entirely contained in the corresponding fragment. Block pairs in phase $t + 3$ are of length $\Oh(\sqrt[4]{\Delta})$,
and thus we deactivate $\Omega(x)$ positions.
Therefore, the time spent on such chunks and long phrases in all phases sums to $\Oh(n)$.
The remaining chunks and long phrases contribute $\Oh(\sqrt[4]{\Delta})$ to the total time,
and there are $\Oh(m / \sqrt{\Delta})$ of them, which adds up to $\Oh(m / \sqrt[4]{\Delta})$.
In every phase, this is $\Oh(n/\sqrt[4]{\Delta})$, so $\Oh(n)$ overall by \cref{lem:polylog}.
