 
\section{Boundary Information}
For the duration of this section, fix a pattern $p$ of length $m$. 
For a string $s$, let $\prefix(s)$ be the longest prefix of $s$ which is a suffix of $p$ and $\suffix(s)$ the longest suffix of $s$ which is a prefix of $p$. 
%
If $s$ occurs in $p$ at position $i$, meaning $s=p[i...i+|p|-1]$, then we define \emph{\substringinfo{$p$}} for $s$ as $(i,i+|p|-1)$, and otherwise \substringinfo{$p$} for $s$ is undefined. 
%
For a string $s$, a \emph{\boundaryinfo{$p$}} is defined as follows:
\begin{enumerate}
\item If $s$ occurs in $p$, then it is simply \substringinfo{$p$} for $s$;
\item Otherwise, it is two substrings of $p$, $u_s$ and $v_s$ such that $\prefix(s)$ is a prefix of $u_s v_s$ which in turn is a prefix of $s$ (\emph{\prefixinfo{$p$}}), and two substrings of $p$, $x_s, y_s$ such that $\suffix(s)$ is a suffix of $x_s y_s$ which in turn is a suffix of $s$ (\emph{\suffixinfo{$p$}}). (See Fig.~\ref{fig:pboundary}.)
\end{enumerate}

\inputpmgapped{figures/fig_boundary}

%This concept was first introduced by Ganardi and Gawrychowski~\cite{DBLP:conf/soda/GanardiG22} to approximate $\prefix(s)$ and $\suffix(s)$. 
For a string $s$, multiple \emph{\boundaryinfo{$p$}} can be constructed. One way to construct one is recursively: for two strings $s, t$, assume to be given \boundaryinfo{$p$} for $s, t$, Algorithm~\ref{alg:boundary} (first described in~\cite{DBLP:conf/soda/GanardiG22})  correctly constructs a \boundaryinfo{$p$} for $c=st$.

\begin{algorithm}[!ht]
\centering
\begin{enumerate}
\item \label{case:s_is_substring} If $s$ is a substring of $p$ and $t$ is not,  then the \suffixinfo{$p$} of $c$ is the \suffixinfo{$p$} of $t$ and we define the \prefixinfo{$p$} for $c$ as follows:
        \begin{enumerate} 
        \item  If $su_t$ is a substring of $p$, then $u_c=su_t$ and $v_c=v_t$;
        \item \label{case:t_is_substring} Otherwise, $u_c=s$ and $v_c=u_t$.
        \end{enumerate}
\item If $t$ is a substring of $p$ and $s$ is not, then the \prefixinfo{$p$} of $c$ is the \prefixinfo{$p$} of $s$ and we define the suffix information for $c$ as follows: 
        \begin{enumerate}
            \item \label{case:merge} If $y_s t$ is a substring of $p$ then $y_c=y_s t$ and $x_c=x_s$;
            \item \label{case:crop} Otherwise, $x_c=y_s$ and $y_c=t$;
        \end{enumerate}
\item \label{case:both_are_substring} If $s$ and $t$ are both substrings of $p$, and $c$ is a substring $p[i \dots j]$ of $p$, then the \boundaryinfo{$p$} is \substringinfo{$p$} for $c$, and we define it equal to $(i,j)$. Otherwise, we put $u_c=x_c=s$ and $v_c=y_c=t$. 

\item \label{case:none_is_substring} If neither $s$ nor $t$ is a substring of $p$, then one can take \prefixinfo{$p$} for $c$ equal to \prefixinfo{$p$} for $s$ and \suffixinfo{$p$} to \suffixinfo{$p$} for $t$. 
\end{enumerate}
\caption{A boundary information of $c = st$}
\label{alg:boundary}
\end{algorithm}

