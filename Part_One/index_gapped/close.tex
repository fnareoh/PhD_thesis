\section{Compressed Indexing for Close Co-occurrences}\label{indexgapped:sec:close}
In this section, we show our main result, Theorem~\ref{thm:close_co_occurrences}. Recall that $S$ is a string of length $N$ represented by an SLP $G$ of size $g$. We start by applying Lemma~\ref{lm:locally_consistent} to transform $G$ into an RLSLP $G'$ of size $g' = O(g \log N)$ and height $h = O(\log N)$ representing $S$. 

The query algorithm uses the following strategy: first, it identifies all non-terminals of $G'$ such that their expansion contains a $b$-close relevant co-occurrence, where a relevant co-occurrence is defined similarly to a relevant occurrence: 

\begin{definition}[Relevant co-occurrence]
Let $A$ be a non-terminal of $G'$. We say that a co-occurrence $(q_1,q_2)$ of $P_1, P_2$ in $\str{A}$ is \emph{relevant} if $q_1 \le |\str{\head(A)}| \le q_2 + |P_2|-1$.   
\end{definition}

Second, it retrieves all $b$-close relevant co-occurrences in each of those non-terminals, and finally, reports all $b$-close co-occurrences by traversing the (pruned) parse tree of $G'$, which is possible due to the following claim:

\begin{restatable}{claim}{relevantcoocc}
\label{claim:relevant_cons_occ}
Assume that $P_2$ is not a substring of $P_1$, and let $(q_1,q_2)$ be a co-occurrence of $P_1, P_2$ in a string $S$. In the parse tree of $G'$, there exists a unique node $u$ such that either
\begin{enumerate}
\item Its label $A$ is associated with a production $A \rightarrow BC$, and $(q_1-\off(u),q_2-\off(u))$ is a relevant co-occurrence of $P_1,P_2$ in $\str{A}$;
\item Its label $A$ is associated with a production $A \rightarrow B^k$, $q_1-\off(u)=q_1'+k' |\str{B}|$, $q_2-\off(u)=q_2'+k' |\str{B}|$ for some $0 \le k' \le k$, where $(q_1',q_2')$ is a relevant co-occurrence of $P_1, P_2$ in $\str{A}$. 
\end{enumerate}
\end{restatable}
\begin{proof}
Let $A$ be the label of the lowest node $u$ in the parse tree that contains leaves $S[q_1], S[q_1+1], \ldots, S[q_2+|P_2|-1]$ in its subtree. Because $P_2$ is not a substring of $P_1$, $A$ cannot be associated with a production $A \rightarrow a$. By definition, $S[\off(u)+1]$ is the leftmost leaf in the subtree of this node. 

Assume first that $A$ is associated with a production $A \rightarrow BC$. We then have that the subtree rooted at the left child of $u$ (labelled by $B$) does not contain $S[q_2+|P_2|-1]$ and the subtree rooted at the right child of $u$ (labelled by $C$) does not contain $S[q_1]$. As a consequence, $(q_1-\off(u),q_2-\off(u))$ is a relevant co-occurrence of $P_1,P_2$ in $\str{A}$. 

Consider now the case where $A$ is associated with a production $A \rightarrow B^k$. The leaves labelled by $S[q_1]$ and $S[q_2+|P_2|-1]$ belong to the subtrees rooted at different children of $A$. If $S[q_1]$ belongs to the subtree rooted at the $k'$-th child of $A$, then $(q_1-\off(u)-|\str{B}| \cdot (k'-1),q_2-\off(u)-|\str{B}| \cdot (k'-1))$ is a relevant co-occurrence of $P_1,P_2$ in $\str{A}$. 
\end{proof}

\subsection{Combinatorial observations}
Informally, we define a set of $O(g^2)$ strings and show that for any patterns $P_1,P_2$ there are two strings $S_1,S_2$ in the set with the following property: whenever the expansion of a non-terminal $A$ in $G'$ contains a pair of occurrences $P_1,P_2$ forming a relevant co-occurrence, there are occurrences of $S_1,S_2$ in the proximity. This will allow us to preprocess the non-terminals of $G'$ for occurrences of the strings in the set and use them to detect $b$-close relevant co-occurrences of $P_1,P_2$. 

Consider two tries, \Tpre\ and \Tsuf: For each production of $G'$ of the form $A\rightarrow BC$, we store $\str{C}$ in  \Tsuf\ and $\rev{\str{B}}$ in \Tpre. For each production of the form $A\rightarrow B^k$, we store $\str{B}$, $\str{B^2}$, $\str{B^{k-2}}$, and $\str{B^{k-1}}$ in \Tsuf\ and the reverses of those strings in \Tpre. For $j \in \{1,2\}$ and $s \in \splits(G', P_j)$ define $S_j(s) = \rev{U} V$, where $U$ is the label of the locus of $\rev{P_j[\dots s]}$ in \Tpre\ and $V$ is the label of the locus of $P_j(s \dots]$ in \Tsuf. Let $l_j(s) = |\rev{U}|$ and $\Delta_j(s) = l_j(s)-s$.  

Consider a non-terminal $A$ such that its expansion $\str{A}$ contains a relevant co-occurrence $(q_1,q_2)$ of $P_1,P_2$. 

\begin{claim}\label{claim:q2}
There exists $s \in \splits(G', P_2)$ such that $p_2 = q_2-\Delta_2(s)$ is an occurrence of $S_2(s)$ in $\str{A}$ and $[p_2,p_2+|S_2(s)|) \supseteq [q_2,q_2+|P_2|)$.
\end{claim}
\begin{proof}
Below we show that there exists a descendant $A'$ of $A$ and a split $s \in \splits(G', P_2)$ such that either $\rev{P_2[\dots s]}$ is a prefix of $\rev{\head(A')}$ and $P_2(s \dots ]$ is a prefix of $\str{\tail(A')}$, or $A'$ is associated with a rule $A' \rightarrow (B')^k$, $\rev{P_2[\dots s]}$ is a prefix of $\rev{\str{(B')^2}}$ and $P_2(s\dots ]$ is a prefix of $\str{(B')^{k-2}}$. The claim follows by the definition of \Tpre, \Tsuf, and $S_2(s)$. 

If $q_2$ is relevant in $\str{A}$, there exists a split $s \in \splits(G', P_2)$ such that $\rev{P_2[\dots s]}$ is a prefix of $\rev{\head(A)}$ and $P_2(s \dots ]$ is a prefix of $\str{\tail(A)}$ by definition. If $q_2$ is not relevant, then $q_2 \ge |\str{\head(A)}|$ by the definition of a co-occurrence. By Claim~\ref{claim:primary_occurrence}, there is a descendant $A'$ of $A$ corresponding to a substring $\str{A}[\ell \dots r]$ for which either $(q_2-\ell)$ is relevant (and then we can repeat the argument above), or $A'$ is associated with a rule $A' \rightarrow (B')^k$ and $(q_2-\ell)-k' \cdot |\str{B'}|$ is relevant, for some $0 \le k' \le k$. Consider the latter case. If $A'=A$, then $k'=1$, as otherwise $q_1 < q_2' = q_2-|\str{B'}| < q_2$ is an occurrence of $P_2$ in $\str{A}$ contradicting the definition of a co-occurrence (recall that $(q_1,q_2)$ is a relevant co-occurrence and hence by definition $q_1 < |\str{\head(A)}|$), and therefore $s = |\str{(B')^2}|-q_2+\ell \in \splits(G', P_2)$, $\rev{P_2[ \dots s]}$ is a prefix of $\rev{(B')^2}$ and $P_2(s \dots ]$ is a prefix of $\str{(B')^{k-2}}$. If $A' \neq A$, then we can analogously conclude that $k'=0$, which implies $s = |\str{B'}|-q_2+\ell \in \splits(G', P_2)$, $\rev{P_2[ \dots s]}$ is a prefix of $\rev{B'}$ and $P_2(s \dots ]$ is a prefix of $\str{(B')^{k-1}}$.
\end{proof}

As the definition of a co-occurrence is not symmetric, $q_1$ does not enjoy the same property. However, a similar claim can be shown:

\begin{lemma}\label{lm:q1}
There exists $s \in \splits(G', P_1)$ and an occurrence $p_1$ of $S_1(s)$ in $\str{A}$ such that $[p_1,p_1+|S_1(s)|) \supseteq [q_1,q_1+|P_1|)$ and at least one of the following holds:
\begin{enumerate}
\item $q_1-\Delta_1(s)$ is an occurrence of $S_1(s)$;
\item $q_2$ is a relevant occurrence of $P_2$ in $\str{A}$, the period of $S_1(s)$ equals the period $\pi_1$ of $P_1$, and there exists an integer $k$ such that $p_1 = q_1-\Delta_1(s)-\pi_1 \cdot k$ and $q_2+\pi_1-1 \le p_1 +|S_1(s)|-1 \le q_2+|P_2|-1$.
\end{enumerate}
\end{lemma}
\begin{proof}
If $q_1$ is a relevant occurrence of $P_1$ in $A$ with a split $s \in \splits(G', P_1)$, then $\rev{P_1[\dots s]}$ is a prefix of $\rev{\str{\head(A)}}$ and $P_1(s \dots ]$ is a prefix of $\str{\tail(A)}$ and therefore the first case holds by the definition of \Tpre\ and \Tsuf. 

Otherwise, by Claim~\ref{claim:primary_occurrence}, there is a descendant $A'$ of $\head(A)$ corresponding to a substring $\str{A}[\ell \dots r]$ for which either $(q_1-\ell)$ is relevant (and then we can repeat the argument above), or $A'$ is associated with a rule $A' \rightarrow (B')^k$ and $(q_1-\ell)-k' \cdot |\str{B'}|$, for some $0 \le k' \le k$, is a relevant occurrence of $P_1$ in $\str{A'}$ with a split $s \in \splits(G',P_1)$. Consider the latter case. We must have (1) $q_1+|P_1|-1+|\str{B'}| \ge r$ or (2) $q_1+|\str{B'}|-1 \ge q_2$, because if both inequalities do not hold, then $q_1 < q_1+|\str{B'}| \le q_2$ is an occurrence of $P_1$ in $\str{A}$, which contradicts the definition of a co-occurrence. Additionally, if (1) holds, then by definition there exists a split $s' \in \splits(G', P_1)$ (which might be different from the split $s$ above) such that $\rev{P_1[ \dots s']}$ is a prefix of $\rev{\str{(B')^{r-1}}}$ and $P_1(s' \dots ]$ is a prefix of $\str{B'}$ and we fall into the first case of the lemma. 

From now on, assume that (2) holds and (1) does not. Since $q_1+|\str{B'}| \le r \le |\str{\head(A)|}$ and $(q_1,q_2)$ is a relevant co-occurrence, $q_2$ must be a relevant occurrence of $P_2$ in $\str{A}$. If $|P_1|-s \le |\str{(B')^2}|$, then $\rev{P_1[ \dots s]}$ is a prefix of $\rev{\str{B'}}$ and $P_1(s \dots ]$ is a prefix of $\str{(B')^2}$ and therefore $q_1-\Delta_1(s)$ is an occurrence of $S_1(s)$. Otherwise, by Fine and Wilf's periodicity lemma~\cite{fine1965uniqueness}, the periods of $\str{A'}$, $P_1$, and $S_1(s)$ are equal, since $P_1$ and hence $S_1(s)$ span at least two periods of $\str{A'}$. By periodicity, $S_1(s)$ occurs at positions $q_1-\Delta_1(s)-|\str{B'}| \cdot k$ of $\str{A}$. Let $p_1$ be the leftmost of these positions which satisfies $p_1+|S_1(s)|-1 \ge q_1+|P_1|-1$. This position is well-defined as (1) does not hold, and furthermore $[q_1,q_1+|P_1|) \subseteq [p_1,p_1+|S_1(s)|)$ as $s \le l_1(s)$ and $|S_1(s)|-l_1(s) \ge |P_1|-s$. We have $p_1 = q_1-\Delta_1(s)-\pi_1 \cdot k''$ for some integer $k''$ (as $|\str{B'}|$ is a multiple of $\pi_1$), and 
$q_2 + \pi_1-1 \le q_1+2|\str{B'}|-1 \le q_1+|P_1|-1 \le p_1+|S_1(s)|-1 \le r < q_2+|P_2|-1$, 
where the last inequality holds as $q_2$ is a relevant occurrence in $\str{A}$. The claim of the lemma follows.
\end{proof}

\inputindexgapped{figures/lemma18.tex}

We summarize Claim~\ref{claim:q2} and Lemma~\ref{lm:q1}:

\begin{corollary}\label{cor:q1_and_q2}
Let $(q_1,q_2)$ be a co-occurrence of $P_1,P_2$ in the expansion of a non-terminal $A$. There exist splits $s_1 \in \splits(G',P_1), s_2 \in \splits(G',P_2)$ and occurrences $p_1$ of $S_1(s_1)$ and $p_2$ of $S_2(s)$, where $[p_1,p_1+|S_1(s_1)|) \supseteq [q_1,q_1+|P_1|)$ and $[p_2,p_2+|S_2(s_2)|) \supseteq [q_2,q_2+|P_2|)$, such that at least one of the following holds:
\begin{enumerate}
\item \label{it:aperiodic} The occurrence $p_1$ is either relevant or $p_1+|S_1(s_1)|-1 \le |\str{\head(A)}|$. The occurrence $p_2$ is either relevant or $p_2 > |\str{\head(A)}|$. Additionally, $p_1 = q_1 - \Delta_1(s_1)$ and $p_2 = q_2 - \Delta_2(s_2)$. 
\item \label{it:periodic} The occurrence $p_2$ is relevant and $p_1 \le |\str{\head(A)}|$. Additionally, $p_2 = q_2 - \Delta_2(s_2)$, the period of $S_1(s)$ equals the period $\pi_1$ of $P_1$, and there exists an integer $k$ such that $p_1 = q_1-\Delta_1(s_1)-\pi_1 \cdot k$ and $p_2+\pi_1-1 \le p_1 +|S_1(s_1)|-1 \le p_2+|S_2(s_2)|-1$.
\end{enumerate}
\end{corollary}

The reverse observation holds as well:

\begin{observation}\label{obs:rev_q1_and_q2}
If $p_j$ is an occurrence of $S_j(s)$ in $\str{A}$, $j = 1,2$, then $q_j = p_j+\Delta_j(s)$ is an occurrence of $P_j$. Furthermore, if $S_1(s)$ is periodic with period $\pi_1$, then $q_1 + \pi_1 \cdot k$, $0 \le k \le \lfloor (|S_1(s)|-q_1-|P_1|)/\pi_1\rfloor$, are occurrences of $P_1$ in $\str{A}$. 
\end{observation}

Finally, the following trivial observation will be important for upper bounding the time complexity of our query algorithm:

\begin{observation}\label{obs:close}
If a string contains a pair of occurrences $(q_1,q_2)$ of $P_1$ and $P_2$ such that $0 \leq q_2-q_1\leq b$, then it contains a $b$-close co-occurrence of $P_1$ and $P_2$.
\end{observation}

\subsection{Index}
The first part of the index is the data structure of Theorem~\ref{th:occurrences} and the index of Christiansen et al.~\cite{talg/ChristiansenEKN21}:

\begin{fact}[{\cite[Introduction and Theorem 6.12]{talg/ChristiansenEKN21}}]
There is a $O(g \log^2 N)$-space data structure that can find the $\occ$ occurrences of any pattern $P[1\dots m]$ in $S$ in time $O(m + \occ)$.
\end{fact}

The second part of the index are the tries \Tpre\ and \Tsuf, augmented as explained below. Consider a quadruple $(u_1,u_2,v_1,v_2)$, where $u_1$ and $u_2$ are nodes of \Tpre\ and $v_1$ and $v_2$ are nodes of \Tsuf. Let $U_1, U_2, V_1, V_2$ be the labels of $u_1, u_2, v_1, v_2$, respectively. Define $S_1=\rev{U_1}V_1$ and $S_2=\rev{U_2}V_2$, and let $l_1 = |\rev{U_1}|$ and $l_2 = |\rev{U_2}|$.  

First, we store a binary search tree $\mathcal{T}_1(u_1, u_2, v_1, v_2)$ that for each non-terminal $A$ contains at most six integers $d = p_2-p_1$, where $p_1, p_2$ are occurrences of $S_1,S_2$ in $\str{A}$, satisfying at least one of the below:
\begin{enumerate}
\item $p_1$ is the rightmost occurrence of $S_1$ such that $p_1+|S_1|-1 < |\str{\head(A)}|$ and $p_2$ is the leftmost occurrence of $S_2$ such that $p_2 \ge |\str{\head(A)}|$;
\item $p_1$ is a relevant occurrence of $S_1$ with a split $l_1$ and $p_2$ is the leftmost occurrence of $S_2$ such that $p_2 \ge |\str{\head(A)}|$;
\item $p_1$ is a relevant occurrence of $S_1$ with a split $l_1$, $p_2$ is a relevant occurrence of $S_2$ with a split $l_2$; 
\item $p_2$ is a relevant occurrence of $S_2$ with a split $l_2$ and $p_1$ is the rightmost occurrence of $S_1$ such that $p_1+|S_1|-1<p_2$;
\item $p_2$ is a relevant occurrence of $S_2$ with a split $l_2$ and $p_1$ is the leftmost or second leftmost occurrence of $S_1$ in $\str{\head(A)}$ 
such that $p_1 < p_2 \le p_1+|S_1|-1 < p_2+|S_2|-1$. 
\end{enumerate}

Second, we store a list of non-terminals $\mathcal{L}(u_2, v_2)$ such that their expansion contains a relevant occurrence of $S_2$ with a split $l_2$. Additionally, for every $k \in [0, \log N]$, we store, if defined:
\begin{enumerate}
\item The rightmost occurrence $p_1$ of $S_1$ in $S_2$ such that $p_1+(|S_1|-1) \le l_2-2^k$;
\item The leftmost occurrence $p'_1$ of $S_1$ in $S_2$ such that $p'_1 \le l_2-2^k \le p'_1+|S_1|-1$;
\item The rightmost occurrence $p''_1$ of $S_1$ in $S_2$ such that $p''_1 \le l_2-2^k \le p''_1+|S_1|-1$.
\end{enumerate}

Finally, we compute and memorize the period $\pi_1$ of $S_1$. If the period is well-defined (i.e., $S_1$ is periodic), we build a binary search tree $\mathcal{T}_2(u_1, u_2, v_1, v_2)$. Consider a non-terminal~$A$ containing a relevant occurrence $p_2$ of $S_2$ with a split $l_2$.
Let $p_1$ be the leftmost occurrence of $S_1$ such that $p_1 \le  p_2 \le p_1+|S_1|-1 \le p_2 + |S_2|-1$ and $p_1'$ the rightmost. If $p_1$ and $p_1'$ exist ($p_1$ might be equal to $p_1'$) and $p_1'+|S_1|-1 \ge p_2 + \pi_1-1$, we add an integer $(p_1'-p_1)/\pi_1$ to the tree and associate it with $A$. We also memorize a number $\ov(S_1,S_2) = p_2-p_1'$, which does not depend on $A$ by Corollary~\ref{cor:arithmetic_progression} and therefore is well-defined (it corresponds to the longest prefix of $S_2$ periodic with period $\pi_1$).

\begin{claim}
The data structure occupies $O(g^5 \log^5 N)$ space.
\end{claim}
\begin{proof}
The data structure of Theorem~\ref{th:occurrences} occupies $O(g^2\log^4 N)$ space. The index of Christiansen et al. occupies $O(g \log^2 N)$ space. The tries, by Lemma~\ref{lm:tries}, use $O(g') = O(g \log N)$ space. There are $O((g')^4)$ quadruples $(u_1,u_2,v_1,v_2)$ and for each of them the trees take $O(g')$ space. The arrays of occurrences of $S_1$ in $S_2$ use $O(\log N)$ space. Therefore, overall the data structure uses $O(g^5 \log^5 N)$ space.
\end{proof}

\subsection{Query}
Recall that a query consists of two strings $P_1, P_2$ of length at most $m$ each and an integer $b$, and we must find all $b$-close co-occurrences of $P_1,P_2$ in $S$, let $\occ$ be their number. 

We start by checking whether $P_2$ occurs in $P_1$ using a linear-time and constant-space pattern matching algorithm such as~\cite{constantspacepm}. If it is, let $q_2$ be the position of the first occurrence. If $q_2 > b$, then there are no $b$-close co-occurrences of $P_1,P_2$ in $S$. Otherwise, to find all $b$-close co-occurrences of $P_1, P_2$ in $S$ (that \emph{always} consist of an occurrence of $P_1$ in $S$ and the first occurrence of $P_2$ in $P_1$), it suffices to find all occurrences of $P_1$ in $S$, which we do using the index of Christiansen et al.~\cite{talg/ChristiansenEKN21} in time $O(|P_1|+\occ) = O(m+\occ)$. 

From now on, assume that $P_2$ is not a substring of $P_1$. Let $\mathcal{N}$ be the set of all non-terminals in $G'$ such that their expansion contains a relevant $b$-close co-occurrence of $P_1, P_2$. By Claim~\ref{claim:relevant_cons_occ}, $|\mathcal{N}| \le \occ$. 

\begin{lemma}\label{lm:non-term_close_co_occ}
Assume that $P_2$ is not a substring of $P_1$. One can retrieve in $O(m+(1+\occ)\log^3 N)$ time a set $\mathcal{N}' \supset \mathcal{N}$, $|\mathcal{N}'| = O(\occ \log N)$. 
\end{lemma}
\begin{proof}
We start by computing $\splits(G',P_1)$ and $\splits(G',P_2)$ via Lemma~\ref{lm:locally_consistent} in $O((|P_1|+|P_2|) \log N) = O(m \log N)$ time (or providing a certificate that either $P_1$ or $P_2$ does not occur in $S$, in which case there are no co-occurrences of $P_1,P_2$ in $S$ and we are done). Recall that $|\splits(G',P_1)|, |\splits(G',P_2)| \in O(\log N)$. For each fixed pair of splits $s_1 \in \splits (G',P_1)$, $s_2 \in \splits (G',P_2)$ and $j \in \{1,2\}$, we compute the interval of strings in \Tpre\ prefixed by $\rev{P_j[\dots s_j]}$, which corresponds to the locus $u_j$ of $\rev{P_j[\dots s_j]}$ in \Tpre\, and the interval of strings in \Tsuf\ prefixed by $P_j(s_j \dots ]$, which corresponds to the locus $v_j$ of $P_j(s_j \dots ]$ in \Tsuf. Computing the intervals takes $O(m+\log^2 N)$ time for all the splits by Lemma~\ref{lm:tries}. Consider the strings $S_1=\rev{U_1}V_1$ and $S_2=\rev{U_2}V_2$, where $U_1, U_2, V_1, V_2$ are the labels of $u_1, v_1, u_2, v_2$, respectively. Let $l_1 = |\rev{U_1}|$, $\Delta_1 = l_1-s_1$, $l_2 = |\rev{U_2}|$, $\Delta_2 = l_2-s_2$, and $\Delta = \Delta_1-\Delta_2$. 

Consider a relevant co-occurrence $(q_1,q_2)$ of $P_1,P_2$ in the expansion of a non-terminal~$A$. By Corollary~\ref{cor:q1_and_q2}, $q_1,q_2$ imply existence of occurrences $p_1,p_2$ of $S_1,S_2$ such that $[p_1,p_1+|S_1|) \supseteq [q_1,q_1+|P_1|)$ and $[p_2,p_2+|S_2|) \supseteq [q_2,q_2+|P_2|)$. 
Our index must treat both cases of Corollary~\ref{cor:q1_and_q2}. We consider eight subcases defined in Fig.~\ref{fig:q1_and_q2}, which describe all possible locations of $p_1$ and $p_2$. 


\inputindexgapped{figures/cases_co_occ}  

\underline{Subcases~\psubref{subfig:both_irrelevant}-\psubref{subfig:q2_relevant}}. To retrieve the non-terminals, we query $\mathcal{T}_1(u_1, u_2,v_1, v_2)$ to find all integers that belong to the range $[\Delta,\Delta+b]$ (and the corresponding non-terminals). Recall that, for each non-terminal $A$, the tree stores an integer $d = p_2-p_1$, where $p_1$ is the starting position of an occurrence of $S_1$ in $\str{A}$ and $p_2$ of $S_2$. By Observation~\ref{obs:rev_q1_and_q2}, $p_1+\Delta_1$ is an occurrence of $P_1$ and $p_2+\Delta_2$ is an occurrence of $P_2$. The distance between them is in $[0,b]$ iff $d \in [\Delta,\Delta+b]$. By Observation~\ref{obs:close}, each retrieved non-terminal contains a close co-occurrence of $(q_1,q_2)$. On other other hand, if $\str{A}$ contains a co-occurrence $(q_1,q_2)$ corresponding to one Subcases~\psubref{subfig:both_irrelevant}-\psubref{subfig:q2_relevant}, then by Corollary~\ref{cor:q1_and_q2}, $p_1 = q_1-\Delta_1$ is an occurrence of $S_1$ and $p_2 = q_2-\Delta_2$ is an occurrence of $S_2$ and by construction $\mathcal{T}_1(u_1, u_2,v_1, v_2)$ stores an integer $d = p_2-p_1$. Therefore, the query retrieves all non-terminals corresponding to Subcases~\psubref{subfig:both_irrelevant}-\psubref{subfig:q2_relevant}. 

\underline{Subcases~\psubref{subfig:q1_included} and~\psubref{subfig:q1_included_per}}. We must decide whether an occurrence of $P_1$ in $S_2$ forms a $b$-close co-occurrence with the occurrence $\Delta_2$ of $P_2$ in $S_2$, and if so, report all non-terminals such that their expansion contains a relevant co-occurrence of $S_2$ with a split $l_2$, which are exactly the non-terminals stored in the list $\mathcal{L}(u_2, v_2)$. Let $k = \lceil \log(s_2) \rceil$. Recall that the index stores the following information for $k$: 
\begin{enumerate}
\item $p_1$, the rightmost occurrence of $S_1$ in $S_2$ such that $p_1+(|S_1|-1) \le l_2-2^k$;
\item $p'_1$, the leftmost occurrence of $S_1$ in $S_2$ such that $p_1' \le l_2-2^k \le p_1+(|S_1|-1)$;
\item $p''_1$, the rightmost occurrence of $S_1$ in $S_2$ such that $p_1'' \le l_2-2^k \le p''_1+(|S_1|-1)$.
\end{enumerate}
(See Fig.~\ref{fig:included}). By Observation~\ref{obs:rev_q1_and_q2}, the occurrence $p_1$ of $S_1$ induces an occurrence $q_1 = p_1+\Delta_1$ of $P_1$. Furthermore, if $S_1$ is periodic with period $\pi_1$, then $q_1+\pi_1\cdot k$, $0 \le k \le \lfloor (|S_1|-q_1-|P_1|)/\pi_1 \rfloor$, are also occurrences of $P_1$. One can decide whether the distance from any of these occurrences to $q_2$ is in $[0,b]$ in constant time, and if yes, then there $S_2$ contains a $b$-close co-occurrence of $P_1,P_2$ by Observation~\ref{obs:close}. Second, by Corollary~\ref{cor:arithmetic_progression}, if $S_1$ is not periodic, then there are no occurrences of $S_1$ between $p_1'$ and $p_1''$ and $p_1',p_1''$ by Observation~\ref{obs:rev_q1_and_q2} induce occurrences $p_1'+\Delta_1, p_1''+\Delta_1$ of $P_1$. Otherwise, there are occurrences of $P_1$ in every position $p_1'+\Delta_1+k\cdot\pi_1$, $0 \le k \le \lfloor (|S_1|+p_1''-|P_1|-p_1')/\pi_1 \rfloor$. Similarly, we can decide whether the distance from any of them to the occurrence $\Delta_2$ of $P_2$ in $S_2$ is in $[0,b]$ in constant time. Finally, let $q_1$ be the rightmost occurrence of $P_1$ in $S_2$ in the interval $[l_2-2^k+1, \Delta_2]$. We extract $S_2(l_2-2^k, \Delta_2 +|P_2|)$ via Fact~\ref{fact:prefsuf_extraction} and search for $q_1$ using a linear-time pattern matching algorithm for $P_1$, which takes $O(|P_1|+|P_2|) = O(m)$ time. If $0 \le \Delta_2-q_1\le b$, then there is a $b$-close co-occurrence of $P_1,P_2$ in $S_2$. Correctness follows from Corollary~\ref{cor:q1_and_q2}, Observation~\ref{obs:rev_q1_and_q2} and Observation~\ref{obs:close}. 

\begin{figure}[h!]
\inputindexgapped{figures/query_preceed_S1}
\caption{Query algorithm for Subcases~\psubref{subfig:q1_included} and~\psubref{subfig:q1_included_per}.}
\label{fig:included}
\end{figure}

{\underline{Subcase~\psubref{subfig:q1_overlap_per}}}. Let $\pi_1$ be the period of $S_1$. We retrieve the non-terminals associated with the integers $q \in \mathcal{T}_2(u_1, u_2,v_1, v_2)$ such that the intersection of an interval $I = [a,b]$ and $[\ell,q]$ is non-empty, where 
$$a=\lceil{(\Delta-\ov(S_1,S_2))/\pi_1\rceil} \text{, } b=\lfloor{(\Delta-\ov(S_1,S_2)+b)/ \pi_1 \rfloor} \text{ and } \ell = -\lfloor{(|S_1|-|P_1|-\Delta_1)/\pi_1\rfloor}$$
(See the description of the index for the definition of $\ov(S_1,S_2)$). As $\ell$ is fixed, we can implement the query via at most one binary tree search: If $b \le \ell$, the output is empty, if $a \le \ell \le b$, we must output all integers, and if $\ell \le a$, we must output all $q \ge b$. Let us now explain why the algorithm is correct. Consider a non-terminal $A$ for which $\mathcal{T}_2(u_1, u_2,v_1, v_2)$ stores an integer $q$. By construction, $\str{A}$ contains a relevant occurrence of $S_2$ with a split $l_2$. A position $p_1 = |\str{\head(A)}|-l_2-\ov(S_1,S_2)-q \cdot \pi_1$ is the leftmost occurrence of $S_1$ in $\str{A}$ such that $p_1 \le p_2 \le p_1+|S_1|-1$ and $p_2 = |\str{\head(A)}|-l_2-\ov(S_1,S_2)$ the rightmost. Consequently, there is an occurrence $q_1 = |\str{\head(A)}|-l_2-\ov(S_1,S_2)-q'\cdot \pi_1+\Delta_1$ of $P_1$ for each $-\lfloor{(|S_1|-|P_1|-\Delta_1)/\pi_1\rfloor} \le q' \le q$. The occurrence of $S_2$ implies that $q_2 = |\str{\head(A)}|-s_2$ is an occurrence of $P_2$. We have $0 \le q_2-q_1=q'\cdot\pi_1+\ov(S_1,S_2)-\Delta\le b$ iff $\Delta-\ov(S_1,S_2) \le  q' \cdot \pi_1 \le \Delta-\ov(S_1,S_2)+b$, which is equivalent to $[\ell,q] \cap I \neq \emptyset$. It follows that we retrieve every non-terminal corresponding to Subcase~\psubref{subfig:q1_overlap_per}. On the other hand, by Observation~\ref{obs:close}, the expansion of each retrieved non-terminal contains a $b$-close co-occurrence of $P_1, P_2$.

{\underline{Subcase~\psubref{subfig:q1_overlap}}}. We argue that we have already reported all non-terminals corresponding to this subcase and there is nothing left to do. Consider a non-terminal $A$ such that its expansion contains a relevant occurrence $p_2$ of $S_2$. If there are at most two occurrences $p_1$ of $S_1$ such that $p_1 \le p_2 \le p_1+|S_1|-1\le p_2+|S_2|-1$, we will treat them when we query $\mathcal{T}_1(u_1, u_2,v_1, v_2)$ (Subcases~\psubref{subfig:both_irrelevant}-\psubref{subfig:q2_relevant}). Otherwise, by Corollary~\ref{cor:arithmetic_progression}, $S_1$ is periodic and there is an occurrence $p'_1$ of $S_1$ such that $p'_1 \le p_2 < p_2 + \pi_1 \le p_1+|S_1|-1 < p_2 + |S_2|-1$. The non-terminals corresponding to this case are reported when we query $\mathcal{T}_2(u_1, u_2,v_1, v_2)$ (Subcase~\psubref{subfig:q1_overlap_per}).


{\underline{Time complexity}}. 
As shown above, the algorithm reports a set $\mathcal{N}' \supset \mathcal{N}$ of non-terminals and each non-terminal in $\mathcal{N}'$ contains a $b$-close co-occurrence. By Claim~\ref{claim:relevant_cons_occ} and since the height of $G'$ is $h = O(\log N)$, we have $|\mathcal{N}'| = O(\occ \log N)$. Furthermore, for a fixed pair of splits of $P_1,P_2$, each non-terminal in $\mathcal{N}'$ can be reported a constant number of times. Since $|\splits(G',P_1)| \cdot |\splits(G',P_2)| = O(\log^2 N)$, the total size of the output is $|\mathcal{N}'| \cdot O(\log^2 N) = O(\occ \cdot \log^3 N)$. We therefore obtain that the running time of the algorithm is $O(m+\log^3 N + \occ \log^3 N) = O(m+(1+\occ)\log^3 N)$ as desired. 
\end{proof}

Once we have retrieved the set $\mathcal{N}'$, we find all $b$-close relevant co-occurrences for each of the non-terminals in $\mathcal{N}'$ using Theorem~\ref{th:occurrences}. In fact, our algorithm acts naively and computes \emph{all} relevant co-occurrences for a non-terminal in $\mathcal{N}'$, and then selects those that are $b$-close. By case inspection, one can show that a relevant co-occurrence for a non-terminal $A$ always consists of an occurrence of $P_2$ that is either relevant or the leftmost in $\str{\tail(A)}$, and a preceding occurrence of $P_1$. Intuitively, this allows to compute all relevant co-occurrences efficiently and guarantees that their number is small. Formally, we show the following claim: 

\begin{restatable}{lemma}{computingrelevant}\label{lem:relevant_co_occurr_A}
Assume that $P_2$ is not a substring of $P_1$. After $O(m \log N + \log^2 N)$-time preprocessing, the data structure of Theorem~\ref{th:occurrences} allows to compute all $b$-close relevant co-occurrences of $P_1, P_2$ in the expansion of a given non-terminal $A$ in time $O(\log^{3} N \log\log N)$. 
\end{restatable}


A part of the index of Christiansen et al.~\cite{talg/ChristiansenEKN21} is a pruned copy of the parse tree of $G'$. They showed how to traverse the tree to report all occurrences of a pattern, given its relevant occurrences in the non-terminals. By using essentially the same algorithm, we can report all $b$-close co-occurrences in amortized constant time per co-occurrence, which concludes the proof of Theorem~\ref{thm:close_co_occurrences}. (For completeness, we provide all details of this step in Appendix~\ref{app:close}, Lemma~\ref{lem:close_co_occurr}.)  

