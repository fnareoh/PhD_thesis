% A command to draw a rival to avoid code duplication
\newcommand{\Rival}[6]{% x, y, lext, rext, length anchor, color
\node (s1) at (#1,#2) {};% bottom left corner of the box containing the rival
\draw ($(s1)+(0,0.5)$) -- ($(s1)+(#3,0.5)$);
\draw[fill=#6] ($(s1)+(#3,0.25)$) rectangle ++((#5,0.5) ;
\draw ($(s1)+(#3,0.5)+(#5,0)$) -- ($(s1)+(#3,0.5)+(#4,0)+(#5,0)$);
}

\newcommand{\Init}{
%rectangle
\node (a) at (0,0) {};
\node (b) at (13,0) {};
\draw[gray] (a) -- (b);
\draw[gray] (6,-0.2) -- (6,0.2);

%C1
\node(c1) at (1,0) {};
\node at ($(c1)+(0,1)$) {$C_1$};
\draw (c1) rectangle ($(c1)+(3,0.5)$);
\node (a1) at ($(c1)+(1,0)$) {};
\draw[fill=color1] ($(a1)$) rectangle ++(1,0.5) node [midway] {$a_1$};

%C2
\node (c2) at (8,0) {};
\node at ($(c2)+(4,1)$) {$C_2$};
\draw (c2) rectangle ($(c2)+(4,0.5)$);
\node (a2) at ($(c2)+(2,0)$) {};
\draw[fill=color2] ($(a2)$) rectangle ($(a2)+(1,0.5)$) node [midway] {$a_2$};
}

\begin{figure}
\centering
\captionsetup[subfigure]{justification=centering}
\begin{subfigure}{0.45\textwidth}
\centering
\begin{tikzpicture}[scale=0.5, every node/.style={scale=0.7}] % ignore
\Init

%left side
\Rival{0.8}{-1}{0.7}{0.3}{1}{color1}
\Rival{0}{-1.7}{0.4}{0.5}{1}{color2}

%rightside
\Rival{10.2}{-1}{0.3}{0.7}{1}{color1}
\Rival{9.6}{-1.7}{1.2}{0.5}{1}{color2}
\end{tikzpicture}
\caption{ignore}
\label{fig:rivals:ignore}
\end{subfigure}
\hfill
\begin{subfigure}{0.45\textwidth}
\centering
\begin{tikzpicture}[scale=0.5, every node/.style={scale=0.7}] % included
\Init
\Rival{8.2}{-1}{0.7}{0.3}{1}{color1}
\Rival{1.2}{-1}{0.4}{0.3}{1}{color2}
\node at (0,-1.7) {}; %hack so the graph align
\end{tikzpicture}
\subcaption[fig:rivals:ignore]{included}
\end{subfigure}

\begin{subfigure}{0.45\textwidth}
\centering
\begin{tikzpicture}[scale=0.5, every node/.style={scale=0.7}] % simple
\Init
\Rival{6.2}{-1}{0.7}{0.3}{1}{color1}
\Rival{3.2}{-1}{0.4}{0.3}{1}{color2}
\end{tikzpicture}
\subcaption[fig:rivals:simple]{simple}
\end{subfigure}
\hfill
\begin{subfigure}{0.45\textwidth}
\centering
\begin{tikzpicture}[scale=0.5, every node/.style={scale=0.7}] % annoying
\Init
\Rival{7.5}{-1}{0.7}{0.3}{1}{color1}
\Rival{1.6}{-1}{0.8}{0.8}{1}{color2}
\end{tikzpicture}
\subcaption[fig:rivals:annoying]{annoying}
\end{subfigure}

\caption{Different types of rivals illustrated. For simplicity, we illustrate the case where $C_1$ and $C_2$ are on each side of the boundary of $A$ and they do not overlap,but the definition is not limited to this case.}
\label{fig:rivals}
\end{figure}